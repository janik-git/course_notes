\chapter{MEAN FIELD LIMIT FOR SDE SYSTEM}
\section{Basics On Probability Theory}
This section is dedicated to a small review of basic concepts 
in probability theory in preparations of SDE's
\subsection{Probability Spaces and Random Variables}
\begin{definition}[$\sigma$-Algebra]
 Let $\Omega $  be a given set, then a $\sigma-$algebra $\mathcal{F}$ on $\Omega $ is a
 family of subsets of $\Omega $ s.t.
 \begin{enumerate}
   \item $\emptyset \in  \mathcal{F}$
   \item $F \in  \mathcal{F} \implies F^{c} \in  \mathcal{F} $
   \item If $A_{1},A_{2},\ldots \in \mathcal{F}$ countable, then 
     \begin{align*}
       A = \bigcup_{j=1}^{\infty} A_j \in \mathcal{F}
     .\end{align*}
 \end{enumerate}
\end{definition}
\begin{definition}[Measure Space]
 A tuple $(\Omega ,\mathcal{F})$  is called a measurable space. The elements of $\mathcal{F}$ are 
 called measurable sets 
\end{definition}
\begin{definition}[Probability Measure]
 A probability measure $\P$ on $(\Omega ,\mathcal{F})$  is a function 
 \begin{align*}
   \P \ : \ \mathcal{F} \to [0,1]
 .\end{align*}
 s.t.
 \begin{enumerate}
   \item $\P(\emptyset) = 0$ , $\P(\Omega ) = 1$
   \item If $A_{1},A_{2},\ldots \in \mathcal{F}$ s.t. $A_i \cap A_j = \emptyset \ \forall  i \neq j$  then
     \begin{align*}
       \P(\bigcup_{j=1}^{\infty} A_j ) = \sum_{j=1}^{\infty} \P(A_j) 
     .\end{align*}
 \end{enumerate}
\end{definition}
\begin{definition}[Probability Space]
 The triple $(\Omega ,\mathcal{F},\P)$  is called a probability space. $F \in  \mathcal{F}$ is called
 event. We say the probability space $(\Omega ,\mathcal{F},\P)$ is complete, if $\mathcal{F}$ contains all zero-measure sets i.e.
 if 
 \begin{align*}
  \inf \{\P(F) \ : \ F \in  \mathcal{F},G \subset  F\}  = 0
 .\end{align*}
 then $G \in  \mathcal{F}$ and $\P(G) = 0$. Without loss of generality we use in this lecture $(\Omega ,\mathcal{F},\P)$
 as complete probability space
\end{definition}
\begin{definition}[Almost Surely]
  If for some $F \in  \mathcal{F}$ it holds $\P(F) = 1$ the we say that $F$ happens with 
  probability 1 or almost surely (a.s.)
\end{definition}
\begin{remark}
 Let $\mathcal{H}$  be a family of subsets of $\Omega$, then there exists a smallest $\sigma-$algebra of 
 $\Omega$ called $\mathcal{U}_{\mathcal{H}}$ with 
 \begin{align*}
   \mathcal{U}_{\mathcal{H}} = \bigcap_{\substack{\mathcal{H} \subset \mathcal{U} \\ \mathcal{H} \ \sigma-\text{alg.}}} \mathcal{H}  
 .\end{align*}
\end{remark}
\begin{example}
  The $\sigma-$algebra generated by a topology $\tau $ of $\Omega$ , $\mathcal{U}_{\tau } \triangleq \mathcal{B}$ is called 
  the Borel $\sigma-$algebra, the elements $B \in  \mathcal{B}$ are called Borel sets.
\end{example}
\begin{definition}[Measurable Functions]
 Let $(\Omega ,\mathcal{F},\P)$  be a probability space, a function 
 \begin{align*}
  Y \ : \ \Omega  \to \mathbb{R}^{d} 
 .\end{align*}
 is called measurable if and only if 
 \begin{align*}
  Y^{-1}(B) \in  \mathcal{F} 
 .\end{align*}
 holds for all $B \in  \mathcal{B}$ or equivalent for all $B \in  \tau $
\end{definition}
\begin{example}
 Let $X : \Omega  \to  \mathbb{R}^{d} $  be a given function, then the $\sigma-$algebra $\mathcal{U}(X)$ generated by X is 
 \begin{align*}
  \mathcal{U}(X) = \{X^{-1}(B) \ : \ B \in  \mathcal{B} \}  
 .\end{align*}
\end{example}
\begin{lemma}[Doob-Dynkin]
 If $X,Y \ : \ \Omega  \to \mathbb{R}^{d} $  are given then $Y$ is $\mathcal{U}(X)$ measurable if and only if 
 there exists a Boreal measurable function $g \ : \ \mathbb{R}^{d} \to  \mathbb{R}^{d}  $ such that 
 \begin{align*}
  Y = g(x)
 .\end{align*}
\end{lemma}
\begin{exercise}
  Proof the above lemma
\end{exercise}
From now on we denote $(\Omega ,\mathcal{F},\P)$ as a given probability space.
\begin{definition}[Random Variable]
 A random variable $X \ : \ \Omega  \to \mathbb{R}^{d} $  is a $\mathcal{F}-$measurable function.
 Every random variable induces a probability measure or $\mathbb{R}^{d} $ 
 \begin{align*}
  \mu_X(B) = \P(X^{-1}(B) ) \quad \forall B \in  \mathcal{B}
 .\end{align*}
This measure is called the distribution of X
\end{definition}
\begin{definition}[Expectation and Variance]
 Let $X$ be a random variable, if 
 \begin{align*}
   \int_{\Omega } \abs{X(\omega )}d\P(\omega ) < \infty
 .\end{align*}
 then 
 \begin{align*}
   \E[X] = \int_{\Omega } X(\omega ) d\P(\omega ) =  \int_{\mathbb{R}^{d} }x d\mu_X(x)
 .\end{align*}
 is called the expectation of $X$ (w.r.t. $\P$) \\[1ex]
 \begin{align*}
   \V[X] = \int_{\Omega } \abs{X - \E[X]}^2 d\P(\omega )
 .\end{align*}
 is called variance and there exists the simple relation 
 \begin{align*}
   \V[X] = \E[\abs{X-\E[X]}^2] = \E[\abs{X}^2] - \E[X]^2
 .\end{align*}
\end{definition}
\begin{remark}
 If $f : \mathbb{R}^{d} \to  \mathbb{R} $ measurable and 
 \begin{align*}
   \int_{\Omega } \abs{f(X(\omega ))} d\P(\Omega ) <\infty
 .\end{align*}
 then 
 \begin{align*}
   \E[f(x)] = \int_{\Omega }f(X(\omega ))d\P(\omega ) = \int_{\mathbb{R}^{d} }f(x) d\mu_X(x)
 .\end{align*}
\end{remark}
\begin{definition}[$L^p$ spaces]
  Let $X : \Omega  \to  \mathbb{R}^{d} $  be a random variable and $p \in [1,\infty)$.
  With the norm 
  \begin{align*}
    \|X\|_p = \|X\|_{L^{p}(\P ) } = \left( \int_{\Omega} \abs{X(\omega )}^{p} d\P(\omega )  \right)^{\frac{1}{p}} 
  .\end{align*}
  If $p=\infty$ 
  \begin{align*}
    \|X\|_{\infty} = \inf \{N \in  \mathbb{R} : \abs{X(\omega )} \le  N \text{ a.s.}\}  
  .\end{align*}
  the space $L^{p}(\P ) = L^{p}(\Omega ) = \{X \ : \ \Omega  \to  \mathbb{R}^{d}  \ | \ \|X\|_p \le \infty \}    $ is a Banach space.
\end{definition}
\begin{remark}
 If $p=2$ then $L^{2}(\P) $ is a Hilbert space with inner product 
 \begin{align*}
   \braket{X,Y} = \E[X(\omega )*Y(\Omega )] = \int_{\Omega }X(\omega )*Y(\omega )d\P(\omega )
 .\end{align*}
\end{remark}
\begin{definition}[Distribution Functions]
 Note for $x,y \in  \mathbb{R}^{d} $  we write $x\le y$ if $x_i \le  y_i$ for $\forall i$
 \begin{enumerate}
   \item $X: (\Omega ,\mathcal{F},\P) \to \mathbb{R}^{d} $ is a random variable the ints distribution function $F_x : \mathbb{R}^{d} \to [0,1] $
     is defined by 
     \begin{align*}
      F_X(x) = \P(X\le x) \quad x \in  \mathbb{R}^{d} 
     .\end{align*}
    \item If $X_{1},\ldots ,X_m : \Omega \to \mathbb{R}^{d} $ are random variables, their joint distribution function is
      \begin{align*}
        F_{X_{1},\ldots ,X_m} &: (\mathbb{R}^{d} )^m \to [0,1]\\
        F_{X_{1},\ldots ,X_M} &= \P(X_{1}\le x_{1},\ldots ,X_m\le x_m) \quad \forall x_i \in \mathbb{R}^{d} 
      .\end{align*}
 \end{enumerate}
\end{definition}
\begin{definition}[Density Function Of X]
 If there exists a non-negative function $f(x) \in  L^{1}(\mathbb{R}^{d} ; \mathbb{R} ) $   such that 
 \begin{align*}
   F(x) = \int_{-\infty}^{x_{1}}  \ldots \int_{-\infty}^{x_n} f(y) dy \quad y = (y_{1},\ldots ,y_n)
 .\end{align*}
 then f is called density function of $X$ and 
 \begin{align*}
  \P(X^{-1}(B) ) = \int_B f(x) dx \quad \forall  B \in  \mathcal{B}
 .\end{align*}
\end{definition}
\begin{example}
 Let $X$ be random variable with density function  $x \in  \mathbb{R}$
 \begin{align*}
 f(x) = \frac{1}{\sqrt{2\pi \sigma ^2}}e^{-\frac{\abs{x-m}^2}{2\sigma ^2}}  
 .\end{align*}
 then we say that $X$ has a Gaussian (or Normal) distribution with mean m and variance $\sigma^2$ and write
 \begin{align*}
  X \sim \mathcal{N}(m,\sigma^2)
 .\end{align*}
 Obviously 
 \begin{align*}
   \int_{\mathbb{R}} xf(x) dx = m \quad \text{ and } \quad \int_{\mathbb{R}}\abs{x-m}^2f(x) dx = \sigma^2
  .\end{align*}
\end{example}
\begin{definition}[Independent Events]
  Events $A_{1},\ldots ,A_{n} \in  \mathcal{F}$ are called independent if $\forall 1 \le k_{1} < \ldots  < k_m \le  n$ it holds 
  \begin{align*}
    \P(A_{k_{1}}\cap A_{k_2} \cap \ldots \cap A_{k_m} )=\P(A_{k_{1}})\P(A_{k_{2}})\ldots \P(A_{k_m})
  .\end{align*}
\end{definition}
\begin{definition}[Independent $\sigma-$Algebra]
 Let $\mathcal{F}_j \subset  \mathcal{F}$   be $\sigma-$algebras for $j=1,2,\ldots $. Then we say $\mathcal{F}_j$ are independent if 
 for $\forall 1 \le k_{1}<k_{2}<\ldots <k_m$ and $\forall A_{k_j} \in  \mathcal{F}_{k_j}$ it holds
 \begin{align*}
  \P(A_{k_{1}}\cap A_{k_2} \cap \ldots \cap A_{k_m} )=\P(A_{k_{1}})\P(A_{k_{2}})\ldots \P(A_{k_m})
 .\end{align*}
\end{definition}
\begin{definition}[Independent Random Variables]
 We say random variables $X_{1},\ldots ,X_m \ : \ \Omega  \to \mathbb{R}^{d} $  are independent if 
 for $\forall  B_{1},\ldots ,B_{m} \subset  \mathcal{B}$ in $\mathbb{R}^{d} $ it holds 
 \begin{align*}
   \P(X_{j_{1}}\in B_{j_{1}},\ldots, X_{j_k}\in B_{j_k} ) = \P(X_{j_{1}} \in  B_{j_{1}})\ldots \P(X_{j_k} \in  B_{j_k})
 .\end{align*}
 which is equivalent to proving that $\mathcal{U}(X_{1}),\ldots ,\mathcal{U}(X_k)$ are independent
\end{definition}
\begin{theorem}
 $X_{1},\ldots ,X_m \ : \ \Omega  \to \mathbb{R}^{d} $ are independent if and only if 
 \begin{align*}
   F_{X_{1},\ldots ,X_m}(x_{1},\ldots ,x_m) =F_{X_{1}}(x_{1})\ldots F_{x_m}(x_m) \quad \forall  x_i \in \mathbb{R}^{d} 
 .\end{align*}
\end{theorem}
\newpage
\begin{theorem}
  If $X_{1},\ldots ,X_m \ : \ \Omega  \to \mathbb{R} $ are independent and $\E[\abs{X_i}] < \infty$ then 
  \begin{align*}
    \E[\abs{X_{1},\ldots ,X_m}]<\infty
  .\end{align*}
  and 
  \begin{align*}
    \E[X_{1}\ldots X_m] = \E[X_{1}]\ldots \E[X_m]
  .\end{align*}
\end{theorem}
\begin{theorem}
  $X_{1},\ldots ,X_m \ : \ \Omega  \to \mathbb{R} $ are independent and $\V[X_i] <\infty$ then 
  \begin{align*}
    \V[X_{1} + \ldots  + X_m] = \V[X_{1}] + \ldots  + \V[X_m]
  .\end{align*}
\end{theorem}
\begin{exercise}
 Proof the above theorems
\end{exercise}
\subsection{Borel Cantelli}
\begin{definition}
 Let $A_{1},\ldots ,A_m \in \mathcal{F}$   then the set 
 \begin{align*}
   \bigcap_{n=1}^{\infty}\bigcup_{m=n}^{\infty} A_m = \{\omega \in \Omega  \ : \ \omega  \text{ belongs to infinite many} A_{m}\text{'s}\}  
 .\end{align*}
 is called $A_m$ infinitely often or $A_m$ i.o.
\end{definition}
\begin{lemma}[Borel Cantelli]\label{borel_cantelli}
 If $\sum_{m=1}^{\infty} \P(A_m) < \infty  $  then $\P(A_ \text{ i.o. }) = 0$
\end{lemma}
\begin{proof}
 By definition we have 
 \begin{align*}
   \P(A_m \text{ i.o. }) \le  \P(\bigcup_{m=n}^{\infty} ) \le \sum_{m=n}^{\infty} \P(A_m)  \xrightarrow{m\to \infty} 0
 .\end{align*}
\end{proof}
\begin{definition}[Convergence In Probability]
  We say a sequence of random variables $(X_k)_{k=1}^{\infty} $  converges in probability to $X$ if 
  for $\forall  \epsilon > 0$
  \begin{align*}
    \lim_{k\to \infty} \P(\abs{X_k - X} > \epsilon ) = 0 
  .\end{align*}
\end{definition}
\begin{theorem}[Application Of Borel Cantelli]
 If $X_k \to  X$  in probability, then there exists 
 a subsequence $(X_{k_j})_{j=1}^{\infty} $ such that 
 \begin{align*}
   X_{k_j}(\omega ) \to X(\omega ) \text{ for almost every } \omega \in  \Omega 
 .\end{align*}
 This means that $\P(\abs{X_{k_j}-X}\to 0) = 1$
\end{theorem}
\begin{proof}
  For $\forall  j \ \exists k_j$  with $k_j < k_{j+1} \to  \infty$ s.t.
  \begin{align*}
    \P(\abs{X_{k_j} - X} > \frac{1}{j}) \le \frac{1}{j^2}
  .\end{align*}
  then 
  \begin{align*}
    \sum_{j=1}^{\infty} \P(\abs{X_{k_j}-X} > \frac{1}{j}) = \sum_{j=1}^{\infty} \frac{1}{j^2}   < \infty
  .\end{align*}
  Let $A_j = \{\omega  \ : \ \abs{X_{k_j}-X} > \frac{1}{j}\}  $ then by \nameref{borel_cantelli} we have $\P(A_j\text{ i.o.}) = 0$ s.t.
  \begin{align*}
    \forall  \omega  \in  \Omega  \ \exists J \text{ s.t. } \forall  j>J
  .\end{align*}
  it holds 
  \begin{align*}
    \abs{X_{k_j}(\omega ) - X(\omega )} \le  \frac{1}{j}
  .\end{align*}
\end{proof}
\subsection{Strong Law Of Large Numbers}
\begin{definition}
 A sequence of random variables $X_{1},\ldots ,X_n$  is called identically distributed if 
 \begin{align*}
   F_{X_{1}}(x)= F_{X_{2}}(x) = \ldots  = F_{X_n}(x) \quad \forall x \in  \mathbb{R}^{d}  
 .\end{align*}
 If additionally $X_{1},\ldots ,X_n$ are independent then we say they are identically-independent-distributed i.i.d
\end{definition}
\begin{theorem}[Strong Law Of Large Numbers]
 Let $X_{1},\ldots ,X_N$  be a sequence of i.i.d integrable random variables on the same probability
 space $(\Omega ,\mathcal{F},\P)$ then 
 \begin{align*}
   \P(\lim_{N \to  \infty} \frac{X_1 + \ldots  + X_N}{N} = \E[X_i]) = 1
 .\end{align*}
 where $\E[X_i] = \E[X_j]$ 
\end{theorem}
\begin{proof}
  Suppose for simplicity $\E[X^{4} ] < \infty$  for $\forall  i = 1,2,\ldots $.
  Then without loss of generality we may assume $\E[X_i] = 0$ otherwise we use $X_i - \E[X_i]$ as our new sequence.
  Consider 
  \begin{align*}
    \E[(\sum_{i=1}^{N} X_i )^{4} ] = \sum_{i,j,k,l}\E[X_iX_jX_kX_l]
  .\end{align*}
  If $i \neq j ,k,l$ then because of independence it follows that 
  \begin{align*}
    \E[X_iX_jX_kX_l] = \E[X_i]\E[X_jX_kX_l] = 0
  .\end{align*}
  Then 
  \begin{align*}
    \E[(\sum_{i=1}^{N}X_i )^{4} ] &= \sum_{i=1}^{N}\E[X_i^{4} ]  + 3 \sum_{i\neq j}\E[X_i^2X_j^2] \\
                                  &= N\E[X_{1}^{4} ] + 3(N^2-N)\E[X_{1}^2]^2 \\
                                  &\le  N^2C
  .\end{align*}
  Therefore for fixed $\epsilon > 0$
  \begin{align*}
    \P(\abs{\frac{1}{N} \sum_{i=1}^{N}X_i } \ge \epsilon) &= \P(\abs{\sum_{i=1}^{N}X_i }^{4}  \ge (\epsilon N)^{4} )\\
                                                          &\myS{Mrkv.}{\le} \frac{1}{(\epsilon N)^{4} } \E[\abs{\sum_{i=1}^{N} X_i }^{4} ]\\
                                                          &\le  \frac{C}{\epsilon^4}\frac{1}{N^2}
  .\end{align*}
  Then by \nameref{borel_cantelli} we get 
  \begin{align*}
    \P(\abs{\frac{1}{N} \sum_{i=1}^{N} X_i } \ge  \epsilon \text{ i.o.}) = 0
  .\end{align*}
  because 
  \begin{align*}
    \sum_{N=1}^{\infty} \P(A_N)  = \sum_{N=1}^{\infty} \frac{C}{\epsilon^4}\frac{1}{N^2}  < \infty
  .\end{align*}
  where  
  \begin{align*}
    A_N = \{\omega  \in  \Omega  \ : \ \abs{\frac{1}{N}\sum_{i=1}^{N}X_i } \ge  \epsilon\}  
  .\end{align*}
  Now we take $\epsilon = \frac{1}{k}$ then the above gives 
  \begin{align*}
    \lim_{N\to \infty} \sup \frac{1}{N} \sum_{i=1}^{N}  X_i(\omega ) \le \frac{1}{k}
  .\end{align*}
  holds except for $\omega  \in  B_k$ with $\P(B_k) = 0$. Let $B = \bigcup_{k=1}^{\infty} B_k $ then $\P(B) = 0$ and
  \begin{align*}
    \lim_{N \to \infty} \frac{1}{N} \sum_{i=1}^{N} X_i(\omega ) = 0  \text{ a.e.}
  .\end{align*}
\end{proof}
\subsection{Conditional Expectation}
\begin{definition}
  Let $Y$ be random variable, then $\E[X|  Y] $ is defined as a $\mathcal{U}(Y)-$measurable random variable
  s.t for $\forall  A \in \mathcal{U}(Y)$ it holds 
  \begin{align*}
    \int_A X d\P = \int_A \E[X|Y] d\P
  .\end{align*}
\end{definition}
\begin{definition}
 Let $(\Omega ,\mathcal{F},\P)$  be a probability space and $\mathcal{ U} \subset  \mathcal{F}$ be a $\sigma-$algebra,
 if $X  : \Omega  \to  \mathbb{R}^{d} $ is an integrable random variable then $\E[X |\mathcal{U}]$  is
 defined as a random variable on $\Omega$ s.t. $\E[X | \mathcal{U}]$ is $\mathcal{U}-$measurable and for $\forall A \in  \mathcal{U}$
 \begin{align*}
   \int_A X d\P  = \int_A \E[X | \mathcal{U}] d \P
 .\end{align*}
\end{definition}
\begin{exercise}
 Proof the following equalities  
 \begin{enumerate}
   \item $\E[X|Y] = \E[X | \mathcal{U}]$
   \item $\E[\E[X|\mathcal{U}]] = \E[X]$
   \item $\E[X] = \E[X | \mathcal{W}]$, where $\mathcal{W} = \{\emptyset,\Omega \}  $
 \end{enumerate}
\end{exercise}
\begin{remark}
 One can define the conditional probability similarly. Let $\mathcal{V} \subset  \mathcal{U}$  be a $\sigma-$algebra 
 then for $A \in  \mathcal{U}$ the conditional probability is defined as follows
 \begin{align*}
   \P(A | \mathcal{V}) = \E[\cha_A | \mathcal{V}]
 .\end{align*}
Note the equivalent notation $\chi_A \equiv \cha_A$
\end{remark}
\begin{theorem}
 Let $X$ be an integrable random variable, then for all $\sigma-$algebras $\mathcal{U} \subset  \mathcal{F}$  the 
 conditional expectation $\E[X | \mathcal{U}]$ exists and is unique up to $\mathcal{U}-$measurable sets of probability
 zero
\end{theorem}
\begin{proof}
 Omit 
\end{proof}
\begin{theorem}[Properties Of Conditional Expectation]
 \begin{enumerate}
   \item If $X$ is $\mathcal{U}-$measurable then $\E[X|\mathcal{U}] = X$ a.s.
   \item $\E[aX + bY|\mathcal{U}] = a\E[X|\mathcal{U}] + b \E[Y|\mathcal{Y}]$
   \item If $X$ is $\mathcal{U}-$measurable and $XY$ is integrable then 
     \begin{align*}
       \E[XY|\mathcal{U}] = X \E[Y|\mathcal{Y}]
     .\end{align*}
   \item If $X$ is independent of $\mathcal{U}$ then $\E[X|\mathcal{U}] = \E[X]$ a.s.
   \item If $\mathcal{W} \subset  \mathcal{U}$ are two $\sigma-$algebras then 
     \begin{align*}
       \E[X|\mathcal{W}] = \E[\E[X|\mathcal{U}]|\mathcal{W}] = \E[\E[X|\mathcal{W}]|\mathcal{U}] \text{ a.s.}
     .\end{align*}
   \item If $X\le Y$ a.s. then $\E[X|\mathcal{U}] \le \E[Y\mathcal{U}]$  a.s.
 \end{enumerate} 
\end{theorem}
\begin{exercise}
 Proof the above properties  
\end{exercise}
\begin{lemma}[Conditional Jensen's Inequality]
  Suppose $\phi  : \mathbb{R}\to \mathbb{R}$ is convex and $\E[\phi(x)] < \infty$ then
  \begin{align*}
    \phi(\E[X|\mathcal{U}]) \le \E[\phi(X)|\mathcal{U}]
  .\end{align*} 
\end{lemma}
\begin{exercise}
 Proof the above Lemma 
\end{exercise}
\subsection{Stochastic Processes And Brownian Motion}
\begin{definition}[Stochastic Process]
 A stochastic process is a parameterized collection of random variables 
 \begin{align*}
   (X(t))_{t \in [0,T]} \ : [0,T] \times \Omega  \ : \ \ (t,\omega ) \mapsto X(t,\omega )
 .\end{align*}
 For $\forall  \omega  \in  \Omega $ the map 
 \begin{align*}
   X(*,\omega ) \ : \ [0,T] \to \mathbb{R}^{d}  \ : \ t \mapsto X(t,\omega )
 .\end{align*}
 is called sample path
\end{definition}
\begin{definition}[Modification and Indistinguishable]
 Let $X(*)$  and $Y(*)$ be two stochastic processes, then we say they are modifications of each other if 
 \begin{align*}
   \P(X(t) = Y(t))  = 1 \qquad \forall t \in [0,T] 
 .\end{align*}
 We say they are indistinguishable if 
 \begin{align*}
   \P(X(t) = Y(t) \ \forall  t \in  [0,T])  = 1 
 .\end{align*}
\end{definition}
\begin{remark}
 Note that if two stochastic processes are indistinguishable then they are also always a modification of each other,
 the reverse is not always true.
\end{remark}
\begin{definition}[History]
 Let $X(t)$ be a real valued process. The $\sigma-$algebra 
 \begin{align*}
  \mathcal{U}(t) \coloneqq  \mathcal{U}(X(s) \ | \ 0\le s\le t)
 .\end{align*}
 is called the history of $X$ until time $t\ge 0$
\end{definition}
\begin{definition}[Martingale]
  Let $X(t)$ be a real valued process and $\E[\abs{X(t)}] < \infty$  for $\forall t \ge 0$
  \begin{enumerate}
    \item If $X(s) = \E[X(t)|\mathcal{U}(s)]$ a.s. $\forall  t \ge  s \ge  0$  then $X(*)$ is called a martingale
    \item If $X(s) \lesseqqgtr  \E[X(t)|\mathcal{U}(s)]$ a.s. $\forall  t \ge  s \ge  0$  then $X(*)$ is called a (super) sub-martingale
  \end{enumerate}
\end{definition}
\begin{lemma}
  Suppose $X(*)$ is a real-valued martingale and $\phi  : \mathbb{R} \to  \mathbb{R}$ a convex function.
  If $\E[\abs{\phi(X(t))}] < \infty $ for $\forall  t\ge 0$ then $\phi(X(*))$ is a sub-martingale
\end{lemma}
\begin{theorem}[Martingale-Inequalities]
 Assume $X(*)$  is a process with continuous sample paths a.s. 
 \begin{enumerate}
   \item If $X(*)$ is a sub-martingale then $\forall  \lambda > 0$ , $t \ge 0$ it holds 
     \begin{align*}
       \P(\max_{0\le s\le t} X(s) \ge \lambda ) \le  \frac{1}{\lambda }\E[X(t)^{+} ]
     .\end{align*}
    \item If $X(*)$ is a martingale and $1 < p < \infty$ then
      \begin{align*}
        \E[\max_{0\le s\le t} \abs{X(s) }^{p} ] \le (\frac{p}{p-1})^{p} \E[\abs{X(t)}^{p} ]
      .\end{align*}
 \end{enumerate}
\end{theorem}
\begin{proof}
 Omit 
\end{proof}
\subsection{Brownian Motion}
\begin{definition}[Brownian Motion]
 A real valued stochastic process $W(*)$ is called a Brownian motion 
 or Wiener process if 
 \begin{enumerate}
   \item $W(0) = 0$ a.s.
   \item $W(t)$ is continuous a.s.
   \item $W(t) - W(s) \sim \mathcal{N}(0,t-s)$ for $\forall t\ge s\ge 0$
   \item $\forall \ 0 < t_{1}<t_{2}<\ldots <t_n$ , $W(t_{1}),W(t_{2})-W(t_{1}),\ldots ,W(t_n)-W(t_{n-1})$ are independent 
 \end{enumerate}
\end{definition}
\begin{remark}
 One can derive directly that 
 \begin{align*}
   \E[W(t)] = 0 \quad \E[W^2(t)] = t \qquad \forall t \ge 0
 .\end{align*}
\end{remark}
Furthermore based on the above remark for $t\ge s$ 
\begin{align*}
  \E[W(t)W(s)] &= \E[(W(t)-W(s))(W(s))]+\E[(W(s)w(s))]\\
               &= \E[W(t)-W(s)]\E[W(s)] + \E[W(s)W(s)] \\
               &= s
.\end{align*}
which means generally 
\begin{align*}
  \E[W(t)W(s)] = t \land s
.\end{align*}
\begin{definition}
 An $\mathbb{R}^{d} $  valued process $W(*) = (W^{1}(*),\ldots ,W^{d}(*)  )$ is a $d-$dimensional Wiener process (or Brownian motion) if
 \begin{enumerate}
   \item $W^{k}(*) $ is a 1-$D$ Wiener process for $\forall  k =1 ,\ldots ,d$
   \item $\mathcal{U}(W^{k}(t) \ , \ t\ge 0 )$ $\sigma-$algebras are independent $k=1,\ldots ,d$
 \end{enumerate}
\end{definition}
\begin{remark}
 If $W(*)$  is  a $d-$Dimensional Brownian motion, then $W(t) \sim \mathcal{N}(0,t)$ and for any Borel set $A \subset  \mathbb{R}^{2} $
 \begin{align*}
  \P(W(t) \in  A) = \frac{1}{(2\pi t)^{\frac{n}{2}} } \int_A e^{-\frac{\abs{x}^2}{2t}} dx
 .\end{align*}
\end{remark}
\begin{theorem}
 If $X(*)$  is a given stochastic process with a.s. continuous sample paths and 
 \begin{align*}
   \E[\abs{X(t)-X(s)}^{\beta } ] \le  C \abs{t-s}^{1+\alpha } 
 .\end{align*}
 Then for $\forall 0 < \gamma  < \frac{\alpha }{\beta }$ and $T > 0$ a.s. $\omega$, there $\exists  K = K(\omega ,\gamma ,T)$ s.t.
 \begin{align*}
  \abs{X(t,\omega )-X(s,\omega )}\le K \abs{t-s}^{\gamma } \quad \forall  0 \le s, t \le T 
 .\end{align*}
\end{theorem}
\begin{proof}
 Omit 
\end{proof}
An application of this result on Brownian motion is interesting since 
\begin{align*}
  \E[\abs{W(t)-W(s)}^{2m} ] \le  C \abs{t-s}^{m}  
.\end{align*}
we get immediately 
\begin{align*}
  W(*,\omega ) \in  \mathcal{C}^{\gamma }([0,T])  \quad 0<\gamma <\frac{m-1}{2m} < \frac{1}{2} \ \forall  m \gg 1
.\end{align*}
This means that Brownian motions is a.s. path Hölder continuous up to exponent $\frac{1}{2}$
\begin{remark}
  One can also further prove that the path wise smoothness  of Brownian motion can not be better than Hölder  continuous. Namely 
  \begin{enumerate}
    \item $\forall  \gamma  \in  (\frac{1}{2},1]$  and a.s. $\omega , t \mapsto W(t,\omega )$ is nowhere Hölder  continuous with exponent $\gamma $
    \item $\forall $ a.s. $\omega  \in  \Omega $ the map $t \mapsto W(t,\omega )$ is nowhere differentiable and is of infinite variation on each subinterval.
  \end{enumerate}
\end{remark}
\begin{definition}[Markov Property]
 An $\mathbb{R}^{d}-$valued process $X(*)$ is said to have the Markov property, if $\forall  0\le s\le t$ and 
 $\forall B \subset  \mathbb{R}^{d} $ Borel. , it holds 
 \begin{align*}
   \P(X(t) \in  B | \mathcal{U}(s)) = \P(X(t) \in  B | X(s)) \text{ a.s.}
 .\end{align*}
\end{definition}
\begin{remark}
 The $d-$Dimensional Wiener Process $W(*)$  has Markov property and 
 \begin{align*}
  \P(W(t)\in B | W(s)) = \frac{1}{(2\pi(t-s))^{\frac{n}{2}} } \int_B e^{-\frac{\abs{x - W(s)}^2}{2(t-s)}}  dx \text{ a.s.}
 .\end{align*}
\end{remark}
\subsection{Convergence of Measure and Random Variables}
In the following we include a couple definitions for the convergence of measures and random variables
\begin{definition}[Weak convergence of measures]
 The following statements are equivalent   
 \begin{enumerate}
   \item $\mu_n \rightharpoonup \mu $
   \item For $\forall  f \in  \mathcal{C}_b(\mathbb{R}^{d} )$ it holds 
 \begin{align*}
  \int f d\mu_n \to \int  f d\mu 
 .\end{align*}
\item For $\forall  B \in \mathcal{B} $  
  \begin{align*}
    \mu_n(B) \to  \mu(B)
  .\end{align*}
\item For $\forall f \in  \mathcal{C}_b(\mathbb{R}^{d} )$ uniform continuous it holds 
  \begin{align*}
    \int  f d\mu_n \to  \int f d\mu 
  .\end{align*}
 \end{enumerate}
\end{definition}
\begin{definition}[Weak convergence of Random variable]
 The following statements are equivalent  
 \begin{enumerate}
   \item $X_n$ converges weakly in Law to $X$  
     \begin{align*}
      X_n \rightharpoonup X
     .\end{align*}
   \item For $\forall f \in  \mathcal{C}_b(\mathbb{R}^{d} )$ it holds 
     \begin{align*}
       \E[f(X_n)] \to \E[f(x)]
     .\end{align*}
 \end{enumerate}
 \begin{enumerate}
   \item $X_n$ converges to $X$ in probability
   \item For $\forall \epsilon  >0$ 
     \begin{align*}
       \P(\abs{X_n - X} >\epsilon ) \xrightarrow{n\to \infty} 0
     .\end{align*}
 \end{enumerate}
\end{definition}
\begin{exercise}
 Prove that 
 \begin{align*}
   X_n \to  X \ \text{a.s.} \implies \P(\abs{X_n - X} > \epsilon ) \xrightarrow{n\to \infty} 0 \implies X_n \xRightarrow{(D)} X
 .\end{align*}
\end{exercise}
\begin{definition}[Tightness]
 A set of probability measures $S \subset  \mathcal{P}(\mathbb{R}^{d} )$  is called tight, if 
 for $\forall \ \epsilon  > 0$ there exists $\exists \  K \subset  \mathbb{R}^{d} $ compact such that 
 \begin{align*}
   \sup_{\mu  \in  S} \mu(K^{c} )  \le  \epsilon 
 .\end{align*}
\end{definition}
\begin{theorem}[Prokhorov's theorem]
  A sequence of measures $(\mu_n)_{n \in  \mathbb{N}}$  is tight in $\mathcal{P}(\mathbb{R}^{d} )$ iff 
  any subsequence has a weakly convergences subsequence.
\end{theorem}
\begin{proof}
 Refer to literature 
\end{proof}

\section{It\^o Integral}
From now on we denote by $W(*)$ the $1-D$ Brownian motion on $(\Omega ,\mathcal{F},\P)$
\begin{definition}
  \hspace{0mm}\\
  \begin{enumerate}
    \item $\mathcal{W}(t) = \mathcal{U}(W(s) | 0 \le s \le t)$ is called the history up to t
    \item The $\sigma-$algebra 
      \begin{align*}
        \mathcal{W}^{+}(t) \coloneqq \mathcal{U}(W(s)-W(t) | s\ge t) 
      .\end{align*}
      is called the future of the Brownian motion beyond time $t$.
  \end{enumerate}
\end{definition}
\begin{definition}[Non-Anticipating Filtration]
 A family $\mathcal{F}(*)$  of $\sigma-$algebras is called non-anticipating (w.r.t $W(*)$) if 
 \begin{enumerate}
   \item $\mathcal{F}(t) \supseteq \mathcal{F}(s)$ for $\forall t \ge  s \ge 0$ 
   \item $\mathcal{F}(t) \supseteq \mathcal{W}(t)$ for $\forall t \ge 0$ 
   \item $\mathcal{F}(t)$ is independent of $\mathcal{W}^{+}(t) $ for $\forall t \ge 0$ 
 \end{enumerate}
\end{definition}
A primary example of this is 
\begin{align*}
  \mathcal{F}(t) \coloneqq \mathcal{U}(W(s) , 0\le s\le t, X_{0})
.\end{align*}
where $X_{0}$ is a random variable independent of $\mathcal{W}^{+}(0) $
\begin{definition}[Non-Anticipating Process]
 A real-valued stochastic process $G(*)$  is called non-anticipating (w.r.t. $\mathcal{F}(*)$) if 
 for $\forall t \ge  0$ , $G(t)$ is $\mathcal{F}(t)-$measurable.
\end{definition}
\vskip5mm
From now on we use $(\Omega,\mathcal{F},\mathcal{F}(t),\P)$ as a filtered probability space with right continuous filtration 
$\mathcal{F}(t) = \bigcap_{s \ge t} \mathcal{F}(s)$. Note we also use the convention that $\mathcal{F}(t)$ is complete .
\begin{definition}
  \hspace{0mm}\\
  \begin{enumerate}
    \item A stochastic process is adapted to $(\mathcal{F}(t))_{t\ge 0}$  if $X_t$ is $\mathcal{F}(t)$ measurable for $\forall  t \ge 0$
    \item A stochastic process is progressively measurable w.r.t. $\mathcal{F}(t)$ if
      \begin{align*}
        X(s,\omega ) \ : \ [0,t] \times  \Omega  \to  \mathbb{R}
      .\end{align*}
      is $\mathcal{B}([0,t]) \times  \mathcal{F}(t)$ measurable for $\forall  t > 0$.
  \end{enumerate}
\end{definition}
\begin{definition}
  We denote $\mathbb{L}^2([0,T])$  the space of all real-valued progressively measurable stochastic processes $G(*)$ s.t.
  \begin{align*}
    \E[\int_0^{T} G^2 dt ] < \infty
  .\end{align*}
  We denote $\mathbb{L}^{1}([0,T]) $ the space of all real-valued progressively measurable stochastic processes $F(*)$ s.t.
  \begin{align*}
    \E[\int_0^{T} \abs{F} dt ] < \infty
  .\end{align*}
\end{definition}
\begin{definition}[Step-Process]
  $G \in  \mathbb{L}^2([0,T])$  is called a step process if there exists a partition of the interval $[0,T]$ i.e.
  $P = \{(t_0,t_1,\ldots,t_m):0 = t_{0} < t_{1} < \ldots <t_m =T\}$ s.t. 
  \begin{align*}
    G(t) = G_k \quad \forall  t_k \le t < t_{k+1} \quad k=0,\ldots ,m-1,
  \end{align*}
  where $G_k$ is an $\mathcal{F}(t_k)$ measurable random variable.
\end{definition}
\begin{remark}
Note that the above definition directly yields the following representation for any step process $G \in  \mathbb{L}^2([0,T])$ 
\begin{align*}
  G(t,\omega ) = \sum_{k=0}^{m-1} G_k(\omega )*\cha_{[t_k,t_{k+1})}(t)
.\end{align*}
\end{remark}
\begin{definition}[(Simple) It\^o Integral]
  Let $G \in  \mathbb{L}^2([0,T])$  be a step process. Then we define 
  \begin{align*}
    \int_0^{T} G(t,\omega ) dW_t \coloneqq \sum_{k=0}^{m-1} G_k(\omega )*(W(t_{k+1},\omega )-W(t_k,\omega ))
  .\end{align*}
\end{definition}
\begin{prop}
  Let $G,H \in  \mathbb{L}^2([0,T])$  be two step processes, then for $\forall  a,b \in  \mathbb{R}$ it holds 
  \begin{enumerate}
    \item $\int_0^{T}(aG + bH)dW_t  = a \int_0^{T} G dW_t + b \int_0^{T} HdW_t  $
    \item $\E [\int_0^{T}GdW_t] = 0$.
  \end{enumerate}
\end{prop}
\begin{proof}
  (1). This case is easy. Set 
  \begin{align*}
    G(t) &= G_k \quad t_k \le t <t_{k+1} \quad k=0,\ldots,m_1 -1\\
    H(t) &= H_l \quad t_l \le t <t_{l+1} \quad l=0,\ldots,m_2 -1
  .\end{align*}
  Let $0 \le  t_{0}<t_{1}<\ldots \le t_n=T$ be the collection of $t_k$'s and $t_k$'s which together form a new partition
  of $[0,T]$ then obviously $G,H \in  \mathbb{L}^2([0,T])$ are again step processes on this new partition. We have 
  directly the linearity by definition on the It\^o integral for step processes
  \begin{align*}
    \int_0^{T} (G+H)d W_t = \sum_{j=0}^{n-1} (G_j+H_j)*(W(t_{j+1})-W(t_j))
  .\end{align*}
  (2). By definition we have 
  \begin{align*}
    \E[\int_0^{T} GdW_t ] = \E[\sum_{k=0}^{m-1}G_k(W(t_{k+1})-W(t_k)) ] = \sum_{k=0}^{m-1} \E[G_k(W(t_{k+1})- W(t_k))] 
  .\end{align*}
  Notice that $G_k$ by definition is $\mathcal{F}_{t_k}$ measurable and $W(t_{k+1}) - W(t_k)$ is measurable in $\mathcal{W}^{+}(t_k) $. Since
  $\mathcal{F}_{t_k}$ is independent of $\mathcal{W}^{+}(t_k) $, we can deduce that $G_k$ is independent of $W(t_{k+1}) - W(t_k)$ which implies 
  \begin{align*}
    \sum_{k=0}^{m-1} \E[G_k(W(t_{k+1})- W(t_k))]  = \sum_{k=0}^{m-1} \E[G_k]*\E[W(t_{k+1}) - W(t_k)]  =0
  .\end{align*}
\end{proof}
\begin{lemma}[(Simple) It\^o isometry]
  For step processes $G \in  \mathbb{L}^2([0,T])$  we have 
  \begin{align*}
    \E[(\int_0^{T} G dW_t )^2] = \E[\int_0^{T} G^2 dt ]
  .\end{align*}
\end{lemma}
\begin{proof}
  By definition we can write 
  \begin{align*}
    \E[\left( \int_0^{T} G dW_t  \right)^2 ] = \sum_{k,j=0}^{m-1} \E[G_kG_j(W(t_{k+1})-W(t_k))(W(t_{j+1})-W(t_j))] 
  .\end{align*}
  If $j < k$, then $W(t_{k+1}) -W(t_k)$ is independent of $G_kG_j(W(t_{j+1})-W(t_j))$. Therefore 
  \begin{align*}
    \sum_{j<k}\E[\ldots ] = 0 \quad \text{ and }  \quad \sum_{j>k}\E[\ldots ] = 0
  .\end{align*}
  Then we have 
  \begin{align*}
   & \E[\left( \int_0^{T} GdW_t  \right)^2 ] = \sum_{k=0}^{m-1} \E[G_k^2(W(t_{k+1})-W(t_k))^2]  \\
                                            &= \sum_{k=0}^{m-1} \E[G_k^2]\E[(W(t_{k+1})-W(t_k))^2]                               = \sum_{k=0}^{m-1} \E[G_k^2](t_{k+1}-t_k) 
                                          = \E[\int_0^{T} G^2dt ] 
  .\end{align*}
\end{proof}
For general $\mathbb{L}^2([0,T])$ processes we use approximation by step processes to define the It\^o integral 
\begin{lemma}
  If $G \in  \mathbb{L}^2([0,T])$  then there exists a sequence of bounded step processes  $G^{n} \in  \mathbb{L}^2([0,T])$  s.t. 
  \begin{align*}
   \E[\int_0^{T} \abs{G - G^{n} }^2 dt ] \xrightarrow{n\to \infty} 0
  .\end{align*}
\end{lemma}
  We roughly sketch the Idea here and refer the rigorous proof to stochastic calculus lecture. \\[1ex]
  If $G(*,\omega )$  is a.s. continuous then we can take  
 \begin{align*}
   G^{n}(t) \coloneqq  G(\frac{k}{n})  \quad \frac{k}{n} \le t < \frac{k+1}{n} \quad k=0,\ldots ,\floor{nT}
 .\end{align*}
 For general $G \in  \mathbb{L}^2([0,T])$ let 
 \begin{align*}
  G^{m}(t) \coloneqq  \int_0^{t} m e^{m(s-t)}G(s) ds   
 .\end{align*}
 Then $G^{m} \in  \mathbb{L}^2([0,T]) $ , $t \mapsto G^{m}(t,\omega ) $ is continuous for a.s. $\omega $ and 
 \begin{align*}
  \int_0^{T} \abs{G - G^{m} }^2 dt \to 0 \text{ a.s.}
 .\end{align*}

\begin{definition}[It\^o Integral]\label{ito_integral}
  If $G \in  \mathbb{L}^2([0,T])$. Let step processes $G^{n} $ be an approximation of $G$. Then we define
  the It\^o  integral by using the limit 
  \begin{align*}
    I(G) = \int_0^{T} GdW_t \coloneqq  \lim_{n\to \infty} \int_0^{T} G^{n} dW_t
  ,\end{align*}
  where the limit exists in $L^2(\Omega)$.
\end{definition}
In order to derive the validity of this definition, one has to check 
\begin{enumerate}
  \item Existence of the limit. This can be obtained by showing that it is a Cauchy sequence, namely by It\^o isometry we have
    \begin{align*}
      \E[\left( \int_0^{T} (G^{m} - G^{n}  ) dW_t  \right)^2 ] = \E[\int_0^{T} \abs{G^{m} - G^{n}  }^2 dt] \xrightarrow{n,m\to \infty} \to 0
    .\end{align*}
    This implies $\int_0^{T} G^{n} dW_t  $ has a limit in $L^2(\Omega )$ as $n\to \infty$
  \item The limit is independent of the choice of approximation sequences.
    Let $\tilde{G}^{n}  $ be another step process which converges to $G$. Then we have 
    \begin{align*}
      \E[\int_0^{T} \abs{\tilde{G}^{n} - G^{n}   }^2 dt ] \le  \E[\int_0^{T} \abs{G^{n} - G }^2 dt ] + \E[\int_0^{T} \abs{\tilde{G}^{n} - G  }^2 dt ]
    .\end{align*}
    it follows that 
    \begin{align*}
      \E[\left( \int_0^{T} \tilde{G}^{n} dW_t - \int_0^{T} G^{n} dW_t      \right)^2 ] = \E[\int_0^{T} \abs{\tilde{G}^{n} - G^{n}   }^2 dt ] \to 0
    .\end{align*}
\end{enumerate}
By using this approximation, all the properties for step  processes can be obtained for general $\mathbb{L}^2([0,T])$ processes
\begin{theorem}[Properties of the It\^o Integral]
  For $\forall  a,b \in  \mathbb{R}$  and $\forall  G,H \in \mathbb{L}^2([0,T])$ it holds 
  \begin{enumerate}
    \item $\int_0^{T} (aG+bH) dW_t = a\int_0^{T} GdW_t + b\int_0^{T} H dW_t   $ 
    \item $\E[\int_0^{T} GdW_t ] = 0$
    \item $\E[\int_0^{T} GdW_t * \int_0^{T} HdW_t  ] = \E[\int_0^{T} GH dt ]$
  \end{enumerate}
\end{theorem}
\begin{lemma}[It\^o Isometry]
  For general $G \in  \mathbb{L}^2([0,T])$  we have 
  \begin{align*}
    \E[\left( \int_0^{T} G dW_t  \right)^2 ] = \E[\int_0^{T} G^2  dt ]
  .\end{align*}
\end{lemma}
\begin{proof}
  Choose step processes $G_{n} \in \mathbb{L}^2([0,T]) $  such that $G_{n} \to G $ (in the sense previously defined) then by \autoref{ito_integral} we get 
  \begin{align*}
    \|I(G) - I(G_n)\|_{L^2} \xrightarrow{n\to \infty} 0
  .\end{align*}
  Then using the simple version of It\^o isometry one obtains 
  \begin{align*}
    \E[\left(\int_0^{T} G dW_t\right)^2] = \lim_{n\to \infty} \E[\left( \int_0^{T} G_{n} dW_t   \right)^2 ]  = \lim_{n\to \infty} \E[\int_0^{T} (G_n)^2dt ] = \E[\int_0^{T} (G)^2dt ]
  .\end{align*}
\end{proof}
\begin{remark}
  The It\^o integral is a map from $\mathbb{L}^2([0,T]) $  to $L^2(\Omega )$
\end{remark}
\begin{remark}
  For $G \in \mathbb{L}^2([0,T])$  the It\^o integral $\int_0^{\tau } G dW_t $ with $0 \le \tau  \le T$ is a martingale. We also refer the proof of this statement to stochastic calculus lecture.
\end{remark}
\subsection{It\^o's Formula}
\begin{definition}[It\^o Process]
 Let $X(*)$ be a real-valued process given by 
 \begin{align*}
  X(r) = X(s) + \int_s^{r} F dt + \int_s^{r} GdW_t  
 .\end{align*}
 for some $F \in  \mathbb{L}^1([0,T])$ and $G \in  \mathbb{L}^2([0,T])$ for $0\le s\le r\le T$, then $X(*)$ is called It\^o process.\\[1ex]
 Furthermore we say $X(*)$ has a stochastic differential.
 \begin{align*}
  dX =  Fdt + gdW_t \quad \forall  0\le t\le T
 .\end{align*}
\end{definition}
\begin{theorem}[It\^o's Formula]
  Let $X(*)$  be an It\^o process given by $dX = F dt + GdW_t$ for some $F \in  \mathbb{L}^{1}([0,T]) $ and $G \in  \mathbb{L}^2([0,T])$. Assume 
  $u : \mathbb{R} \times  [0,T] \to \mathbb{R}$ is continuous and $\frac{\partial u}{\partial t} ,\frac{\partial u}{\partial x} ,\frac{\partial ^2 u}{\partial x^2} $ exists and 
  are continuous. Then $Y(t) \coloneqq  u(X(t),t)$ satisfies 
  \begin{align*}
    dY &= \frac{\partial u}{\partial t} dt + \frac{\partial u}{\partial x} dX + \frac{1}{2 }\frac{\partial ^2 u}{\partial x^2} G^2 dt\\
       &=(\frac{\partial u}{\partial t} +\frac{\partial u}{\partial x} F + \frac{1}{2} \frac{\partial ^2 u}{\partial x^2} G^2 )dt + \frac{\partial u}{\partial x} G dW_t
  .\end{align*}
    \end{theorem}
Note that the differential form of the It\^o formula is understood as an abbreviation of the following integral form, for all $0\le s < r \le T$
  \begin{align*}
    &u(X(r),r) - u(X(s),s) \\
    &= \int_s^{r}(\frac{\partial u}{\partial t}(X(t),t)+\frac{\partial u}{\partial x}(X(t),t)F(t) + \frac{1}{2}\frac{\partial ^2 u}{\partial x^2}(X(t),t)G^2(t) )dt + \int_s^{r} \frac{\partial u}{\partial x}(X(t),t) G(t)dW_t
  .\end{align*}

\begin{proof}
  The proof is split into five steps \\[1ex]
  \textbf{Step 1.} First we prove two simple cases. If $X(t)=W_t$ then 
  \begin{enumerate}
    \item[(1)] $d(W_t)^2 = 2W_t dW_t + dt $
    \item[(1)] $d(tW_t) = W_t dt + t dW_t$
  \end{enumerate}
 The integral version of (1) is $W_t^2 - W_0^2 = \int_0^{t} 2W_s dW_s + t$ a.s.\\
  By definition of It\^o integral, for a.s. $\omega  \in \Omega $ we have 
  \begin{align*}
    \int_0^{t}2W_sdW_s &= 2 \lim_{n\to \infty} \sum_{k=0}^{n-1} W(t_k^{n} )\left(W(t_{k+1}^{n})-W(t_k^{n} )\right)\\
                       &= \lim_{n\to \infty} \Bigg[\sum_{k=0}^{n-1} W(t_k^{n})\left(W(t_{k+1}^{n}) - W(t_k^{n} )\right) - \sum_{k=0}^{n-1}\left(W(t_{k+1}^{n})-W(t_k^{n}) \right)^2  \\
                       & \hspace{1.1cm}+ \sum_{k=0}^{n-1}W(t_{k+1}^{n} )  \left(W(t_{k+1}^{n})-W(t_k^{n} ) \right) \Bigg]\\
                       &= - \lim_{n\to \infty} \Bigg[\sum_{k=0}^{n-1}\left( W(t_{k+1}^{n}) - W(t_k^{n} ) \right)^2 -  \sum_{k=0}^{n-1}\left(W(t_k^{n} )\right)^2 + \sum_{k=0}^{n-1}\left( W(t_{k+1}^n) \right)^2      \Bigg]\\
                       &= - \lim_{n\to \infty} \sum_{k=0}^{n-1}\left( W(t_{k+1}^{n}) - W(t_k^{n} ) \right)^2 +  \left(W(t)\right)^2 - \left( W(0) \right)^2 
  .\end{align*}
  where for any fixed $n$, the partition of $[0,T]$ is given by $0\le t_{0}^{n} < t_{1}^{n} < \ldots <t_n^{n} = T   $ and 
  $t_{k}^{n} - t_{k+1}^{n}   = \frac{1}{n}$ . It remains to prove that the limit 
  \begin{align*}
    \lim_{n\to \infty}\sum_{k=0}^{n-1} \left( W(t_{k+1}^{n} ) - W(t_k^{n} ) \right)^2 - t = 0 
  .\end{align*}
 Actually, by writing out the square we have 
  \begin{align*}
  &  \E \left[ \left( \sum_{k=0}^{n-1}\left( W(t_{k+1}^{n} ) - W(t_k^{n})\right)^2 - \left( t_{k+1}^{n} - t_k^{n}   \right)   \right)^2   \right] \\
  =&\E\Bigg[\sum_{k=0}^{n-1}\sum_{l=0}^{n-1}
    \left( \left( W(t_{k+1}^{n} ) - W(t_k^{n})\right)^2 - \left( t_{k+1}^{n} - t_k^{n}   \right)   \right)                                                                                                                                                                                                                                                                                         *\left( \left( W(t_{l+1}^{n} ) - W(t_l^{n})\right)^2 - \left( t_{l+1}^{n} - t_l^{n}   \right)   \right) \Bigg]
  .\end{align*}
  The terms with $k\neq l$ vanish because of the independence. Therefore
  \begin{align*}
    &\E \left[ \sum_{k=0}^{n-1}\left( \left( W(t_{k+1}^{n} ) - W(t_k^{n})\right)^2 - \left( t_{k+1}^{n} - t_k^{n}   \right)   \right)^2   \right] \\ 
    &= \sum_{k=0}^{n-1}(t_{k+1}^{n} - t_k^{n}  )^2 \E \left[ \left( \frac{\left( W(t_{k+1}^{n} ) - W(t_k^{n} ) \right)^2 }{t_{k+1}^{n} - t_{k}^{n}  } - 1 \right)^2  \right]  \\
    &= \sum_{k=0}^{n-1}(t_{k+1}^{n} - t_k^{n}  )^2 \E \left[ \left( \left(\frac{ W(t_{k+1}^{n} ) - W(t_k^{n} ) }{\sqrt{t_{k+1}^{n} - t_{k}^{n}}  } \right)^2 - 1 \right)^2  \right]  \\
    &\le C*\frac{t^2}{n}\to  0
  ,\end{align*}
  where we have used the fact that $Y = \frac{ W(t_{k+1}^{n} ) - W(t_k^{n} ) }{\sqrt{t_{k+1}^{n} - t_{k}^{n}}  } \sim \mathcal{N}(0,1)$. Hence $\E[(Y^2 - 1)^2]$ is 
  bounded by a constant $C$\\[1ex]
  We will prove the integral form of (2) : $tW_t - 0 W_0 =  \int_0^{t} W_s ds + \int_0^{t} s dW_s  $. Actually we have 
  that
  \begin{align*}
    \int_0^{t} s dW_s  + \int_0^{t} W_s ds &= \lim_{n\to \infty}   \sum_{k=0}^{n-1} t_k^{n}\left( W(t_{k+1}^{n}) -W(t_k^{n} ) \right) + \lim_{n\to \infty}\sum_{k=0}^{n-1} W(t_{k+1}^{n} ) (t_{k+1}^{n} - t_k^{n}  )\\
                                           &= W(t)*t - 0*W(0)
  .\end{align*}
  
  \vskip3mm
  \textbf{Step 2.} Now let us prove the It\^o product rule. Namely, if 
  \begin{align*}
    dX_{1} = F_{1}dt + G_{1}dW_t \quad \text{ and } \quad dX_{2} = F_{2}dt+G_{2}dW_t
  .\end{align*}
  for some $G_i \in  \mathbb{L}^2([0,T])$ and $F_i \in  \mathbb{L}^1([0,T])$ $i=1,2$ , then 
  \begin{align*}
    d(X_{1}X_{2}) &= X_{2}dX_{1} + X_{1}dX_{2} + G_{1}G_{2} dt
                  &=(X_{2}F_{1}+X_{1}F_{2}+G_{1}G_{2})dt + (X_{2}G_{1}+X_{1}G_{2})dW_t
  .\end{align*}
  where the above should be understood as the integral equation.\\
 \vskip2mm  (1) We prove the case $F_i,G_i$ are time independent. Assume for simplicity $X_{1}(0) = X_{2}(0)$ then it follows that 
  \begin{align*}
    X_i(t) = F_it +G_iW(t), \quad i=1,2.
  .\end{align*}
  Then it holds almost surely that 
  \begin{align*}
    &\int_0^{t} (X_{2}dX_{1} + X_{1}dX_{2} + G_{1}G_{2} ds) \\
    =& \int_0^{t} (X_{2}F_{1}+X_{1}F_{2}) ds + \int_0^{t} (X_{2}G_{1}+X_{1}G_{2}) dW_s + \int_0^{t} G_{1}G_{2}ds   \\
    =& \int_0^{t} \left( F_{1}(F_{2}s + G_{2}W(s))  + F_{2}(F_{1}s+G_{1}W(s))\right) ds + G_{1}G_{2}t\\
    &+ \int_0^{t} \left( G_{1}(F_{2}s + G_{2}W(s)) + G_{2}(F_{1}s+G_{1}W(s)) \right) dW_s \\
    =& G_{1}G_{2}t + F_{1}F_{2}t^2+(F_{1}G_{2}+F_{2}G_{1})\left( \int_0^{t}W(s) ds + \int_0^{t} s dW_s   \right) \\
    & + 2G_{1}G_{2} \int_0^{t} W(s) dW_s\\ 
=&  G_{1}G_{2}(W(t))^2  + F_{1}F_{2}t^2 + (F_{1}G_{2}+F_{2}G_{1})tW(t) = X_{1}(t)+X_{2}(t)
  ,\end{align*}
   where we have used the results from Step 1. Therefore It\^o formula is true when $F_i,G_i$ are time independent random variables.\\
  \vskip2mm
  (2) If $F_i,G_i$ are step processes, then we apply the above formula in each sub-interval.\\
  \vskip2mm
  (3) For $F_i \in  \mathbb{L}^1([0,T])$ and $G_i \in  \mathbb{L}^2([0,T])$, we take the step process approximation of them, namely
  \begin{align*}
    \E\Big[\int_0^{T} \abs{F_i^{n} - F_i } dt \Big] \to 0 \quad \E\Big[\int_0^{T} \abs{G_i^{n} - G_i }^2 dt \Big] \to 0 \qquad (n\to \infty), i=1,2
  .\end{align*}
  Notice that for each It\^o process given by step processes 
  \begin{align*}
    X_i^{n}(t) = X_i(0) + \int_0^{t} F_i^{n} ds + \int_0^{t} G_i^{n} dW_s     
  .\end{align*}
  the product rule holds, i.e. 
  \begin{align*}
    X_1^{n}(t)X_2^{n}(t) - X_1(0)X_2(0) = \int_0^{t} \left( X_1^{n}(s)dX_2^{n}(s) + X_2^{n}(s) dX_1^{n}(s) + G_{1}G_{2}ds     \right)    
  .\end{align*}
  Therefore after taking the limit, we obtain that the product rule holds for It\^o processes.
  \vskip2mm
  \textbf{Step 3.} If $u(X) = X^{m} $ for $m \in  \mathbb{N}$ then we claim 
  \begin{align*}
    d(X^{m} ) = mX^{m-1}dX + \frac{1}{2}m(m-1) X^{m-2} G^2dt
  .\end{align*}
  We prove this statement by induction.\\
  \textbf{IA} Note that $m=2$ is given by the product rule. \\
  \textbf{IV} Suppose the formula holds for $m-1 \in  \mathbb{N}$ \\
  \textbf{IS} $m-1 \to m$.\\
  By using the product rule, we have that 
  \begin{align*}
    d(X^{m} )= d(XX^{m-1} ) &= Xd(X^{m-1} ) + X^{m-1}dX + (m-1)X^{m-2} G^2dt \\
                            &\myS{IV}{=} X \left( (m-1)X^{m-2} dX + \frac{1}{2}(m-1)(m-2)X^{m-3}G^2 dt   \right) \\
                            & \quad + X^{m-1}dX + (m-1)X^{m-2}G^2 dt \\
                            &= mX^{m-1} dX + (m-1)(\frac{m}{2} -1 +1) X^{m-2} G^2 dt 
  .\end{align*}
  Thus the statement holds for all $m \in  \mathbb{N}$.\\
  \vskip3mm
  \textbf{Step 4.} If $u(X,t) = f(X)g(t)$ where $f$ and $g$ are polynomials $f(X) = X^{m} $ , $g(t) =t^n$.
  Then by the product rule we have 
  \begin{align*}
    d(u(X,t)) = d(f(X)g(t)) = f(X)dg + g df(X) + (G_{1}*0) dt
  .\end{align*}
  by step 3  this is equal to
  \begin{align*}
    f(X)g'(t)dt + g(t)f'(X)dX + \frac{1}{2}g(t)f''(X)G^2 dt =\frac{\partial u}{\partial t} dt + \frac{\partial u}{\partial X} dX + \frac{1}{2} \frac{\partial ^2 u}{\partial X^2} G^2 dt
  .\end{align*}
  Furthermore, by superposition, we know that the It\^o formula is also true if $u(X,t) = \sum_{i=1}^{m} g_m(t)f_m(X) $ where $f_m$ and $g_m$ are polynomials\\[1ex]
  \vskip3mm
  \textbf{Step 5.} For $u$ continuous such that $\frac{\partial u}{\partial t} , \frac{\partial u}{\partial x} ,\frac{\partial ^2 u}{\partial x^2}$ exists and are also continuous, then
  there exists polynomial sequences $u^{n} $ s.t.
  \begin{align*}
    u^{n} \to  u \quad \frac{\partial u^{n } }{\partial t} \to \frac{\partial u}{\partial t} , \quad \frac{\partial u^{n } }{\partial x} \to \frac{\partial u}{\partial x} , \quad \frac{\partial ^2 u}{\partial x^2} \to \frac{\partial ^2 u}{\partial x^2} 
  .\end{align*}
  uniformly on compact $K \subset  \mathbb{R}\times [0,T]$.

  By using the fact that 
  \begin{align*}
    u^{n}(X(t),t) -u^{n}(X(0),0)  = \int_0^{t} \left( \frac{\partial u^{n } }{\partial t} +\frac{\partial u^{n } }{\partial x} F + \frac{1}{2} \frac{\partial ^2 u^{n} }{\partial x^2} G^2  \right)  dr + \int_0^{t} \frac{\partial u^{n } }{\partial x}  GdW_r \quad \text{a.s.} 
 \end{align*}
 It\^o's formula is proven by taking the limit $n\to \infty$.
\end{proof}
\begin{remark}
 One can get the existence of the polynomial sequence in Step 5, by using Hermetian polynomials 
 \begin{align*}
  H_n(x) = (-1)^{n}  e^{\frac{x^2}{2}} \frac{d^{n}}{d x^{n} } e^{-\frac{x^2}{2}} 
 .\end{align*}
\end{remark}
\begin{exercise}
  If $u \in  \mathcal{C}^{\infty} $  , $\frac{\partial u}{\partial x} \in  \mathcal{C}_b$ then prove Step 4 $\implies$ Step 5 \\[1ex]
  \textit{Use Taylor expansion and use the uniform convergence of the Taylor series on compact support }
\end{exercise}
\newpage
\subsection{Multi-Dimensional It\^o processes and Formula}
We shortly extend the definition of It\^o processes and the It\^o Formula to the multi-dimensional case, we
include the dimensionality as a subscript for clearness.
\begin{definition}[Multi-Dimensional It\^o's Integral]
  We the define the $n-$dimensional It\^o integral for $G \in  \mathbb{L}^{2}_{n*m}([0,T]) $ , $G_{ij} \in  \mathbb{L}^{2}([0,T])$ $1\le i\le n \ , \ 1 \le j \le m$
  \begin{align*}
    \int_0^{T} G d W_t = \begin{pmatrix} \vdots \\ \int_0^{T} G_{ij} d W^{j}_t \\ \vdots    \end{pmatrix}_{n \times 1}
  .\end{align*}
\end{definition}
\begin{remark} It is a direct consequence from $1-$ Brownian motion that
  \begin{align*}
    \E\Big[\int_0^{T} G dW_t \Big] &= 0  \\
    \E\Big[\Big(\int_0^{T} G dW_t \Big)^2\Big] &= \E\Big[\int_0^{T} \abs{G}^2 dt \Big]
  ,\end{align*}
  where $\displaystyle\abs{G}^2 = \sum_{i,j}^{n,m} \abs{G_{ij}}^2 $ 
\end{remark}  
\begin{definition}[Multi-Dimensional It\^o process]
 We define the $n-$dimensional It\^o process as  
 \begin{align*}
   X(t) &= X(s) + \int_s^{t} F_{n \times  1}(r) dr   + \int_0^{t} G_{n \times  m}(r) dW_{m \times  1}(r)  \\
\mbox{or differential version }\quad   dX^{i} &= F^{i} dt + \sum_{j=1}^{m} G^{ij} dW_t^i      \qquad 1\le i \le n
 .\end{align*}
\end{definition}
\begin{theorem}[Multi Dimensional It\^o's formula]
  We define the $n-$dimensional It\^o's formula for $u \in  \mathcal{C}^{2,1}(\mathbb{R}^{n} \times [0,T],\mathbb{R} ) $ by 
  \begin{align*}
    du(X(t),t) &= \frac{\partial u}{\partial t}(X(t),t) dt + \nabla u(X(t),t) * dX(t) \\
               &+ \frac{1}{2} \sum_{i,j=1}^n \frac{\partial ^2 u}{\partial x_i \partial x_j}(X(t),t) \sum_{l=1}^{m}  G^{il} G^{il}dt 
  .\end{align*}
\end{theorem}
\begin{prop}
  For real valued processes $X_{1},X_{2}$
 \begin{align*}
  \begin{cases}
    dX_{1} &= F_{1} dt + G_1 dW_1 \\
    d X_2 &= F_{2} dt + G_{2} dW_2
  \end{cases} \implies d(X_{1}X_{2}) = XdX_{2} + X_{2}dX_{1} + \sum_{k=1}^{m} G_1^{k} G_2^{k} dt   
 .\end{align*} 
\end{prop}
\begin{remark}[Multiplication Rules]
 The following formal multiplication rules are frequently used in computation:
 \begin{align*}
   (dt)^2 = 0 \ , \ dt dW^{k} = 0 \ , \ dW^{k}dW^{l} = \delta_{kl} dt \\
 .\end{align*}
  Using the above we can write the It\^o's formula into a short version as follows 
\begin{align*}
  du(X,t) &= \frac{\partial u}{\partial t} dt + \nabla u*dX + \frac{1}{2}\sum_{i,j=1}^{n} \frac{\partial ^2  u}{\partial x_i \partial x_j}   dX^{i}dX^{j}   \\ 
          &= \frac{\partial u}{\partial t} dt + \sum_{i=1}^{n} \frac{\partial u}{\partial x^{i} } F^{i} dt + \sum_{i=1}^{n} \frac{\partial u }{\partial x_i}      \sum_{i=1}^{m} G^{ik} d W_k   \\
          &+  \frac{1}{2} \sum_{i,j=1}^{n}  \frac{\partial ^2  u}{\partial x_i \partial x_j} \left(F^{i} dt + \sum_{k=1}^{m} G^{ik} dW_k   \right)\left( F^{j} dt + \sum_{l=1}^{m} G^{i;} dW_l    \right)   \\
          &= (\frac{\partial u}{\partial t} + F*\nabla u + \frac{1}{2} H*D^2 u) dt + \sum_{i=1}^{n} \frac{\partial u}{\partial x_i} \sum_{k=1}^{m} G^{ik} dW_{k}
,\end{align*}
where 
\begin{align*}
  dX^{i} &= F^{i} dt + \sum_{k=1}^{m}  G^{ik} dW_k   \\
  H_{ij} &= \sum_{k=1}^{m} G^{ik}G^{jk}  \ , \ A *B = \sum_{i,j=1}^{m} A_{ij} B_{ij} 
.\end{align*}
\end{remark}
\begin{example}
 A typical example for $G$ is  
  \begin{align*}
      G^{T}G = \sigma  I_{n \times  n} 
  .\end{align*}
\end{example}
\begin{remark} 
 If $F$ and $G$ are deterministic 
 \begin{align*}
   dX =  F(t) dt + G(t)dW_t
 .\end{align*}
 Then for arbitrary test function $u \in  \mathcal{C}_0^{\infty}(\mathbb{R}^{n} ) $ we have  by It\^o's formula 
 \begin{align*}
   u(X(t)) - u(X(0)) &= \int_0^{t} \nabla u (X(s)) * F(s) ds + \int_0^{t}  \frac{1}{2}(G^{T}G ) : D^2u(X(s)) ds \\
                     &+ \int_0^{t} \nabla u(X(s)) * G(s) dW_{s} 
 .\end{align*}
 Let $\mu(s,*)$  be the law of $X(s)$ then by taking the expectation of the above integral 
 \begin{align*}
   \int_{\mathbb{R}^{n} } u(x) d\mu(s,x) - \int_{\mathbb{R}^{n} } u(x) d\mu_0(x) &=  \int_{0}^{t} \int_{\mathbb{R}^{n} }  \nabla u(x) * F(s) d\mu(s,x)\\
     &+ \int_0^{t} \int_{\mathbb{R}^{n} }  \frac{1}{2}(G^{T}(s)G(s)) : D^2 u(x) * d\mu (s,x) \\
                                                                                 &+ 0
 .\end{align*}
 Remember the weak derivative of measures, we obtain that the law $\mu(s,*)$ satisfies the following second order parabolic equation in the sense of distribution,
 \begin{align*}
   \partial_t \mu  - \frac{1}{2} \sum_{i,j=1}^{n}  \frac{\partial^2}{\partial x_i\partial x_j} \Big(\sum_{k=1}^{m}  G^{ik}G^{kj}  \mu\Big)  + \nabla * (F \mu )  = 0 
 .\end{align*} 
\end{remark}
\begin{example}
  If $F=0$  $m=n$ and $G=\sqrt{2}I_{n \times  n} $, then the law $\mu $ of $X(t)=\sqrt{2} dW_t $ fulfills the heat equation 
  \begin{align*}
    \partial_t{\mu}- \Delta \mu  = 0.
  .\end{align*}
\end{example}

\section{Relation To The Mean Field Limit}
To find out how all this translates to our Mean field Limit we consider the particle system given by 
\begin{align*}
  \begin{cases}  
  d X_i &=  \frac{1}{N} \sum K(x_i,x_j) dt + \sqrt{2} dW_t^1  \qquad 1\le i\le N \ N\to \infty\\
  X_i(0)    &= x_{0,i} \\
  \mu_N(t) &= \frac{1}{N} \sum_{i=1}^{N} \delta_{X_i(t)} 
  \end{cases}
.\end{align*}
And denote 
\begin{align*}
  \mathbb{X}_N = F(\mathbb{X}_N) dt + \sqrt{2}dW_{t} 
.\end{align*}
At time $t = 0$ the $X_i$ are independent random variables, at any time $t>0$ they are dependent and the particles have joint law 
\begin{align*}
  (X_{1}(t),\ldots ,X_N(t)) \sim  u(X_{1},\ldots ,X_n)
.\end{align*}
Where $u \in  \mathcal{M}(\mathbb{R}^{dN})$, then by It\^o's formula we get for arbitrary test function $\forall  \phi  \in  \mathcal{C}_0^{\infty}(\mathbb{R}^{dN} ) $ 
\begin{align*}
  \phi(\mathbb{X}_N(t)) &=  \phi(\mathbb{X}_N(0)) + \int_0^{t} \nabla\phi  *\begin{pmatrix} \vdots \\ \frac{1}{n} \sum_{j=1}^{N} K(X_i,X_j) \\ \vdots  \end{pmatrix} \\
                        &+ \int_0^{t}  \Delta{\mathbb{X}_N} \phi  dt + \int_0^{t} \sqrt{2} \nabla \phi  dW_t^{i} 
.\end{align*}
Taking the expectation on both sides, then the last term disappears by definition of It\^o processes 
\begin{align*}
  \partial_t - \sum_{i=1}^{N} \Delta_i u  + \sum_{i=1}^{N} \nabla_{X_i} \left( \frac{1}{N} \sum_{j=1}^{N} K(X_i,X_j) u \right)  = 0
.\end{align*}
Now consider the Mean-Field-Limit, if the joint particle law can be rewritten as the tensor product of a single $\overline{u}$ 
\begin{align*}
 u(X_{1},\ldots ,X_N)  = \overline{u}^{\otimes N}  
.\end{align*}
the equation simplifies
\begin{align*}
  \partial_t - \sum_{i=1}^{N} \Delta_i u  + \sum_{i=1}^{N} \nabla_{X_i} \left( \overline{u}^{\otimes N}k \star \overline{u}(X_i)   \right)  = 0
.\end{align*}
\newpage
\section{Solving Stochastic Differential Equations}
The setup of the following section will be the following 
\begin{definition}[Basic Setup]
 We consider the probability space $(\Omega ,\mathcal{F},\mathbb{P})$, With a $m$-$D$ dimensional Brownian motion $W(*)$.
 Let $X_0$ be an $n$-$D$ dimensional random variable independent of $W(0)$, then our Filtration is given by
 \begin{align*}
  \mathcal{F}_t = \sigma(X_{0}) \cup \sigma(W(s) , 0\le s\le t)
 .\end{align*}
\end{definition}
\begin{definition}[SDE]\label{sde}
 Given the above basic setup we are trying to solve equations of the type 
 \begin{align*}
  \begin{cases}
    d\underbrace{X_t}_{n \times 1} &= \underbrace{b}_{n \times 1}(X_t,t) dt + \underbrace{B}_{n \times m}(X_t,t) d\underbrace{W_t}_{m \times 1} \quad 0\le t\le T \\
    X_{t}\rvert_{t=0} &= X_{0} \quad X \ : \ (t,\omega ) \to  \mathbb{R}^{n} 
  \end{cases}
 .\end{align*}
 Where 
 \begin{align*}
   b &: \mathbb{R}^{n} \times [0,T] \to \mathbb{R}^{n}   \\
   B &: \mathbb{R}^{n} \times [0,T] \to  M^{n\times m}  
 .\end{align*}
\end{definition}
\begin{remark}
 The differential equation should always be understood as the Integral equation 
 \begin{align*}
  X_t - X_{0} = \int_0^{t}  b(X_s,s) ds + \int_0^{t} B(X_s,s) dW_s 
 .\end{align*}
\end{remark}
\begin{definition}[Solution]
 We say an $\mathbb{R}^{n}$-valued stochastic process $X(*)$ is a solution of the SDE if 
 \begin{enumerate}
  \item $X_t$ is progressively measurable w.r.t $\mathcal{F}_t$
  \item (drift) $F\coloneqq b(X_t,t) \in  \mathbb{L}_{n}^{1}([0,T]) \ \Leftrightarrow \  \int_0^{t} \E[F_s] ds < \infty $ 
  \item (diffusion) $G\coloneqq B(X_t,t) \in  \mathbb{L}_{n \times m}^{2}([0,T]) \Leftrightarrow \  \int_0^{t} \E[\abs{G_s}^2] ds < \infty $ 
 \end{enumerate}
\end{definition}
\begin{remark}
  (1) implies that for any given $t \in  [0,T]$ $X_t$ is random variable measurable with respect to $\mathcal{F}_t$.
\end{remark}
\hspace{0mm}\\
The goal from now on is to prove the existence and uniqueness of such solutions, for that we first define what it means 
for a solution to be unique
\begin{definition}
 For two solution $X,\tilde{X} $ we say they are unique if
 \begin{align*}
   \P(X(t) = \tilde{X}(t), \ \forall  t \in  [0,T] ) = 1 \Leftrightarrow \max_{0\le t \le T} \abs{x(t)-\tilde{x}(t) }  = 0 \text{ a.s.}
 .\end{align*}
 i.e they are indistinguishable.
\end{definition}
\begin{assumption}\label{assumption_sde_sol}\hspace{0mm}\\
  Let $b : \mathbb{R}^{n} \times  [0,T] \to  \mathbb{R}^{n}  $ and 
  $B : \mathbb{R}^{n} \times  [0,T] \to  M^{n \times m}  $,
  be continuous (in $(t,x)$) and Lipschitz continuous with respect to $x$ for some $L > 0$.
  Furthermore assume they fulfill the linear growth condition
  \begin{align*}
    \abs{b(x,t)} + \abs{B(x,t)} \le  L(1+\abs{x}) 
  .\end{align*}
\end{assumption}
\begin{remark}
  Note the Lipschitz continuity from \autoref{assumption_sde_sol} implies that there $\exists L >0$ such that
  \begin{align*}
    \abs{b(x,t) - b(\tilde{x},t )} +  \abs{B(x ,t) - B(\tilde{x},t )} \le  L \abs{x - \tilde{x} }
  .\end{align*}
\end{remark}
\begin{theorem}[Existence and Uniqueness of Solution]\label{sde_solution_theorem}
  Let \autoref{assumption_sde_sol} hold for an \nameref{(SDE)} and assume the initial data $X_0$ is  
  square integrable and independent of $W^{t}(0)$.
  Then there exists a unique solution $X \in  \mathbb{L}^2_n([0,T])$ of the \nameref{sde}.
\end{theorem}
\begin{proof}
  We begin with the uniqueness prove.\\[1ex] 
  Suppose we have two solutions $X$ and $\tilde{X} $ of the SDE then the goal is to show that they are indistinguishable,
  then by using the definition of a solution 
  \begin{align*}
    X_t - \tilde{X}_t = \int_0^{t} (b(X_s,s) - b(\tilde{X}_s,s )) ds + \int_0^{t} B(X_s,s) - B(\tilde{X}(s),s )  dW_s
  .\end{align*}
  If the diffusion term were 0 we could use a Grönwall type inequality and get the uniqueness.\\[1ex]
  Instead we consider the square of the above and apply It\^os isometry. Note that
  generally $\abs{a+b}^2 \nleqslant  (a^2+b^2)$  but  $\abs{a+b}^2 \le  2(a^2+b^2)$
  \begin{align*}
    \abs{X_t - \tilde{X}_t}^2 \le  2\abs{\int_0^{t} (b(X_s,s) - b(\tilde{X}_s,s )) ds}^2 + \abs{\int_0^{t} B(X_s,s) - B(\tilde{X}(s),s )  dW_s}^2
  .\end{align*}
  Now consider the following 
  \begin{align*}
    \E[\abs{X_t-\tilde{X}_t}^2] &\le 2\E[\abs{\int_0^{t} \abs{b(X_s,s) - b(\tilde{X}_s,x )} ds}^2 ]  \\
                                & \qquad + 2 \E[\abs{\int_0^{t} B(X_s,s) - B(\tilde{X}_s,s ) dW_s}^2]\\
                                &\myS{Hold.}{\le } 2t \E[\int_0^{t} \abs{b(X_s,s) - b(\tilde{X})s,s )}^2 ds ] + 2\E[\int_0^{t} \abs{B(X_s,s)-B(\tilde{X}_s,s )}^2 ds ] \\
                                &\myS{Lip.}{\le } 2(t+1)L^2 \E[\int_0^{t} \abs{X_s-\tilde{X}_s }^2 ds ]\\
                                &= 2(t+1)L^2 \int_0^{t} \E[\abs{X_s-\tilde{X}_s }^2  ]ds\\
  .\end{align*}
 Where the following Hoelders inequality was used 
 \begin{align*}
   \left( \int_0^{t} 1 \abs{f} ds  \right)^2 &\le  \left( \int_0^{t} 1^2 ds  \right)^{\frac{1}{2}*2}*\left( \int_0^{t} \abs{f}^2 ds  \right)^{\frac{1}{2}*2}  \\
                                             &\le t \int_0^{t} \abs{f}^2 ds 
 .\end{align*}
 Now by Gronwalls inequality we have 
 \begin{align*}
   \E[\abs{X_t-\tilde{X}_t }^2] = 0
 .\end{align*}
 i.e $X_t$ and $\tilde{X}_t $ are modifications of each other and it remains to show that they are actually
 indistinguishable.\\[1ex]
 Define 
 \begin{align*}
  A_t = \{ \omega  \in  \Omega  \ | \ \abs{X_t - \tilde{X}_t  } > 0\}   \qquad \P(A_t) = 0
 .\end{align*}
 \begin{align*}
   \P(\max_{t \in  \mathbb{Q} \cap [0,T]} \abs{X_t-\tilde{X}_t } > 0 ) = \P(\bigcup_{k=1}^{\infty} A_{t_k} ) = 0
 .\end{align*}
 Now since $X_t(\omega )$ is continuous in $t$ we can extend the maximum over the entire interval $[0,T]$ 
 \begin{align*}
   \max_{t \in  \mathbb{Q} \cap [0,T]} \abs{X_t - \tilde{X}_t} = \max_{t \in  [0,T]} \abs{X_t - \tilde{X}_t}
 .\end{align*}
 Then the probability over the entire interval must also be 0 
 \begin{align*}
   \P(\max_{t \in  [0,T]} \abs{X_t - \tilde{X}_t} >0)  = 0 \quad \text{ i.e. } X_t = \tilde{X}_t \ \forall  t \text{ a.s.} 
 .\end{align*}
 This concludes the uniqueness proof, for existence similar to the deterministic case we use Picard iteration.\\[1ex]
 First define the Picard iteration by  
 \begin{align*}
   X_t^{0} &= X_0  \\
           &\vdots\\
   X_t^{n+1} &= X_0 + \int_0^{t} b(X_s^{n},s ) ds + \int_0^{t} B(X_s^{n},s ) dW_s   
 .\end{align*}
 Let $d(t)^{n} = \E[\abs{X_t^{n+1}-X_t^{n}}^2] $, then we claim by induction that $d^{n}(t) \le  \frac{(Mt)^{n+1} }{(n+1)!} $ for some $M>0$.\\
  \textbf{IA:} For $n=0$ we have
  \begin{align*}
    d(t)^{0} = \E[\abs{X_t^1-X_t^0}^2] &\le  \E[2 (\int_0^{t} b(X_0,s) ds )^2 + 2 (\int_0^{t} B(X_0,s )dW_s )^2]  \\
                                       &\le  2t \E[\int_0^{t} L^2(1+X_{0}^2) ds ] + 2\E[\int_0^{t} L^2(1+X_{0}) ds ] \\
                                       &\le  tM \qquad \text{ where } M\ge 2L^2(1+\E[X_{0}^2]) +2L^2(1+T)
  .\end{align*}
  \textbf{IV:} suppose the assumption holds for $n-1 \in  \mathbb{N}$\\
  \textbf{IS:} Take $n-1 \to n$ then 
  \begin{align*}
    d^{n}(t) &= \E[\abs{X_t^{n+1} - X^{n}_t }^2] \le  2 L^2 T \E[\int_0^{t} \abs{X_s^{n} - X_s^{n-1}  }^2 ds ]  + 2L^2\E[\int_0^{t} \abs{X_s^{n} - X_s^{n-1}  }^2  ds] \\
             &\myS{IV}{\le } 2L^2(1+T) \int_0^{t} \frac{(Ms)^n}{n!} ds \\
             &= 2L^2(1+t) \frac{M^n}{(n+1)!} t^{n+1} \le \frac{M^{n+1}t^{n+1}}{(n+1)!} 
  .\end{align*}
  Because of $\Omega $ we cannot use completeness to argue the convergence and instead are forced to use a similar argument as in the uniqueness proof. 
  \begin{align*}
    &\E[\max_{0\le t \le T} \abs{X^{n+1}_t - X^{n}_t  }^2] \\
    &\le \E[\max_{0\le t\le T} 2\abs*{\int_0^{t} b(X_s^{n},s )-b(X_s^{n-1},s ) ds}^2 + 2\abs*{\int_0^{t}B(X_s^{n},s )-B(X_s^{n-1},s ) dW_s}^2] \\
    &\le 2TL^2 \E[\int_0^{T} \abs{X_s^{n} - X_s^{n-1}  }^2 ds ] + 2\E[\max_{0\le t\le T} \abs*{\int_0^{t} B(X_s^{n},s )- B(X_s^{n-1},s ) ds W_s}]\\
    &\le  2TL^2 \E[\int_0^{T} \abs{X_s^{n} - X_s^{n-1}  }^2 ds ] + 8\E[\int_0^{T} \abs{B(X_s^{n},s  )-B(X_s^{n-1},s )}^2 ds ]\\
    &\le C*\E[\int_0^{T} \abs{X_s^{n}-X_s^{n-1}  }^2 ds ]
  .\end{align*}
  Where we used the following Doobs martingales $L^{p}$ inequality 
  \begin{align*}
    \E[\max_{0\le s\le t} \abs{X(s)}^{p} ] \le  (\frac{p}{p-1})^{p} \E[\abs{X(t)}^{p}  ] 
  .\end{align*}
  By Picard iteration we know the distance $d^{n}(t) = \E[\abs{X_s^{n}-X_s^{n-1}  }^2] $ is bounded by 
  \begin{align*}
    C*\E[\int_0^{T} \abs{X_s^{n}-X_s^{n-1}  }^2 ds ] &= C* \int_0^{T} \E[\abs{X_s^{n} - X_s^{n-1}  }^2] ds \\
                                                     &\le \int_0^{T} \frac{(Mt)^{n} }{(n)!} \\
                                                     &= C \frac{M^{n} T^{n+1} }{(n+1)!}
  .\end{align*}
  Further more we get with a Markovs inequality
  \begin{align*}
    \P(\underbrace{\max_{0\le t\le T} \abs{X_t^{n+1}-X_t^n}^2 > \frac{1}{2^{n} }}_{A_n}) &\le 2^{2n} \E[\max_{0\le t\le T} \abs{X_t^{n+1}-X_t^n}^2]\\
                                                                      &\le 2^{2n} \frac{CM^{n}T^{n+1}  }{(n+1)!} 
  .\end{align*}
  Then by Borel-Cantelli
  \begin{align*}
    \sum_{n=0}^{\infty} \P(A_n) \le  C \sum_{n=0}^{\infty}2^{2n} \frac{(MT)^{n} }{(n+1)!}    <\infty \implies \P(\bigcap_{n=0}^{\infty} \bigcup_{m=n}^{\infty} A_m ) = 0
   .\end{align*}
   i.e $\exists  B \subset  \Omega  $ with $\P(B) = 1$ s.t $\forall \ \omega  \in  B$ , $\exists \ N(\omega ) > 0 $ s.t
   \begin{align*}
     \max_{0\le t\le T} \abs{X_t^{n+1}(\omega ) - X_t^{n}(\omega )} \le  2^{-n} 
   .\end{align*}
   In fact we can give $B$ directly by 
   \begin{align*}
     \left( \bigcap_{n=0}^{\infty} \bigcup_{m=n}^{\infty} A_m   \right)^{C} = \bigcup_{n=0}^{\infty} \bigcap_{m=n}^{\infty} A_m^{C} = B     
   .\end{align*}
   then for each $\omega  \in  B$  we can make a Cauchy sequence argument by
   \begin{align*}
     \max_{0\le t\le T} \abs{X_t^{n+k} - X^{n}_t } &\le  \sum_{j=1}^{k} \max \abs{X_t^{n+j} - X^{n+(j-1)}_t } \\
                                                   &\le \sum_{j=1}^{k} \frac{1}{2^{n+j-1} }\\
                                                   &< \frac{1}{2^{n-1} } 
   .\end{align*}
   By the above we get 
   \begin{align*}
     X_t^{n}(\omega ) \to X_t(\omega ) \qquad \text{ uniform in } t\in [0,T]
   .\end{align*}
   Therefore for a.s. $\omega $ , take the limit in the iteration  and obtain 
   \begin{align*}
    X_t = X_{0} + \int_0^{t} b(X_s,s) ds + \int_0^{t} B(X_s,s) dW_s  
   .\end{align*}
   It remains to show that $X_t \in  \mathbb{L}^2([0,T])$ note that $X_{0} \in  \mathbb{L}^2([0,T])$ already and 
   \begin{align*}
     \E[\abs{X_t^{n+1}}^2] &\le  C(1+\E[\abs{X_0}^2]) + C \int_0^{t} \E[\abs{X_s^{n} }^2] ds \\
                           &\le  C \sum_{j=0}^{n} C^{j+1} \frac{t^{j+1}}{(j+1)!} (1+\E[\abs{X_0}^2]) \\
                           &\le  C* e^{Ct} 
   .\end{align*}
   Where we used $\E[X_{0}] = 0$ ,the  linear growth condition for the first integral and It\^o isometry for the second and then again the linear growth condition\\[1ex]
    Using the above we conclude by Fatous's lemma
    \begin{align*}
      \E[\abs{X_t}^2] = \E[\lim_{n\to \infty} \abs{X^{n+1}_t }]  \le  \liminf_{n\to \infty} \E[\abs{X^{n+1}_t }^2] \le C*e^{Ct} 
    .\end{align*}
    Therefore 
    \begin{align*}
      \int_0^{T} \E[\abs{X(t)}^2]  \le  CT*e^{CT} 
    .\end{align*}
\end{proof}
\begin{remark}
  One should remember that if the diffusion term $B(X_t,t)$ is 0 then we get a unique solution iff $b(X_t,t)$ is Lipschitz
\end{remark}
\begin{theorem}[Higher Moments Estimate]
  Assumptions for $b$ , $B$ and $X_{0}$ are the same as before, if in addition 
  \begin{align*}
    \E[\abs{X_0}^{2p} ]< \infty
  .\end{align*}
  for some $p \ge 1$ then $\forall  t \in  [0,T]$ 
  \begin{align*}
    \E[\abs{X_t}^{2p}]\le C(1+\E[\abs{X}_0^{2p} ])e^{Ct} 
  .\end{align*}
  and $\E[\abs{X_t - X_0}^{2p} ] \le  C(1+\E[\abs{X_0}^{2p} ])e^{Ct}t^p $
\end{theorem}
\begin{proof}
 Left as an exercise \\
\end{proof}

\section{Stochastic Mean Field Limit}
First recall the metric we use to talk about distance between two measures i.e the Wasserstein Distance
\begin{definition}[Wasserstein Distance]
  For all $\mu , \nu  \in  \mathcal{P}_p(\mathbb{R}^{d} )$  , $(p\ge 1) $ the Wasserstein Distance of $\mu $ and $\nu $ is given by 
  \begin{align*}
    W^{p}(\mu ,\nu ) = \dist_{MK,p}(\mu ,\nu ) = \inf_{\pi \in  \Pi(\mu ,\nu )} \left( \int \int_{\mathbb{R}^{2d} } \abs{x-y}^{p} \pi(dxdy) \right)^{\frac{1}{p}}  
  .\end{align*}
  Where  
  \begin{align*}
    \Pi(\mu ,\nu ) = \Bigg\{\pi \in \mathcal{P}(\mathbb{R}^{d} \times  \mathbb{R}^{d}  ) : &\int_{\mathbb{R}^{d}\times E } \pi(dx,dy) = \nu(E) \\
                                                                                      &\int_{E \times  \mathbb{R}^{d} } \pi(dx,dy) = \mu(E)\Bigg\}  
  .\end{align*}
\end{definition}
\begin{remark}
 Note that 
 \begin{align*}
   W_1(\mu ,\tilde{\mu } ) \le  W_{2}(\mu ,\tilde{\mu } )
 .\end{align*}
 follows naturally by Hölders inequality, in fact this holds for all $p>q$
  \begin{align*}
   W_q(\mu ,\tilde{\mu } ) \le  W_{p}(\mu ,\tilde{\mu } )
 .\end{align*}
\end{remark}
\begin{remark}
  Let $(\mu_n)_{n \in  \mathbb{N}} \subset  \mathcal{P}_p(\mathbb{R}^{d} )$ be a sequence of measures,then following are equivalent
  \begin{enumerate}
   \item $W_p(\mu_n,\mu ) \to 0$
    \item  For $\forall f \in \mathcal{C}(\mathbb{R}^{d} )$ such that $\abs{f(x)} \le  C(1+\abs{x}^{p} )$
     \begin{align*}
      \int  f d\mu_n \to \int  f d\mu 
     .\end{align*} 
    \item $\mu_n \rightharpoonup \mu $
  \end{enumerate} 
\end{remark}
\newpage
\subsection{Stochastic Particle System}
Let us begin by shortly defining the stochastic particle systems we study. 
\begin{definition}[Empirical Measure (Stochastic version)]\label{empirical_stochastic}
  For random variables $(X_i)_{i\le N}$  we define the (stochastic) empirical measure by
  \begin{align*}
    \mu_N(\omega ) = \frac{1}{N}\sum_{i=1}^{N} \delta_{X_i(\omega)} 
  .\end{align*}
\end{definition}
Then our stochastic particle system is given by,
\begin{definition}[Stochastic Particle System ]\label{sden}
  For $N$ interacting particles $(X^{1} ,\ldots ,X^{N} )$ with i.i.d initial data $(X_i^{N}(0))_{i \in  \{1,\ldots ,N\}  } \subset  L^2(\Omega) $ and law $\mu_0$
\begin{align*}
  \text{(SDEN)}\begin{cases}
    d X_i^{N}(t) &=   b(X_i^{N}(t) , \mu_N(t) )dt + \sigma(X_i(t)^{N},\mu_N(t))dW^{i}_t \\
    X_i^{N}(0) &= X_{i,0}^{N}   
  \end{cases}
.\end{align*}
Where $\mu_N$ is the stochastic empirical measure and note $\mathcal{L}(X_{0}) = \mu_0$
\end{definition}
\begin{remark}
  The dimensions for our Stochastic-Particle-System are the same as in \autoref{sde}
\end{remark}
\begin{remark}
  For our initial measure we already have
\begin{align*}
  \E[W_2^{2}(\mu_N(0),\mu_0) ] \to  0
.\end{align*}
\end{remark}
\subsection{I.I.D Case}
Let us shortly consider the convergence of the empirical measure in the case where our random variables are i.i.d, note that in our  mean field limit 
this is only the case for our initial data, since for $t>0$ they are no longer i.i.d.
\begin{corollary}
  If  $(X_i)_{i \in  \{1,\ldots ,N\}  }$ are i.i.d random variables with law $\mu_{X}$ then  $\forall  \ f \in  \mathcal{C}_b(\mathbb{R}^{d} ) $ it holds that
  \begin{align*}
    \P(\lim_{N \to \infty} \int  f d \mu_N = \int f d\mu ) = 1
  .\end{align*}
\end{corollary}
We can actually prove the stronger statement that the choice of $f \in  \mathcal{C}_b$ does not matter for the convergence i.e.
we can pull the function selection into the probability similarly to the difference between modification and indistinguishable. 
\begin{corollary}
  If  $(X_i)_{i \in  \{1,\ldots ,N\}  }$ are i.i.d random variables with law $\mu_{X}$ then  it holds that
\begin{align*}
  \P(\mu_N \rightharpoonup \mu ) = 1
.\end{align*}
i.e 
\begin{align*}
 \P( \forall  f \in  \mathcal{C}_b(\mathbb{R}^{d} ) \ : \int f d\mu_N \to  \int  f d\mu )  = 1
.\end{align*}
\end{corollary}
\begin{proof}
  \textcolor{Red}{Needs revision, this should only work for $\mathcal{C}_b(K)$ $K$ compact in my opinion}\\
 The proof relies mainly on showing that $\mathcal{C}_b(\mathbb{R}^{d} )$  is separable for compact support 
 we can use the density of the polynomials. Then we can go from arbitrary $f$ to the union over a countable sequence of $f$ 
 and then argue through separability that this is equal to the entire space.
\end{proof}
\newpage
\begin{lemma}[General Dominated Convergence]\label{general_dct}
  Let $(X_n)_{n \in  \mathbb{N}} \subset   L^{p} $ be a sequence of random variables then the following are equivalent 
  \begin{enumerate}
    \item $(X_n)_{n \in  \mathbb{N}}$ are uniformly integrable  and $X_n \to X$ $\P$-a.s.
    \item $\|X_n - X\| \to  0$ for some $X \in  L^p$
  \end{enumerate}
\end{lemma}
\begin{proof}

\end{proof}
\begin{remark}
  In general a sequence $(X_i)_{i \in  \mathbb{N}}$ is called uniform integrability if
 \begin{align*}
   \lim_{r \to  \infty} \sup_{i \in  \mathbb{N}} \E[\abs{X_i}*\cha_{\abs{X_i}\ge r}] = 0
 .\end{align*}
\end{remark}
\begin{lemma}[De la Vall\`ee Poussin Criterion]\label{de_la_valle}
 A sequence of random variables $(X_i)$  is uniformly integrable iff there 
 $\exists \phi $ convex with
 \begin{align*}
   \lim_{x \to \infty} \frac{\phi(x)}{x} = \infty
 .\end{align*}
s.t.
\begin{align*}
 \sup_i \E[\phi(\abs{X_i})] < \infty
.\end{align*}
\end{lemma}
\begin{proof}
  As the construction of $\phi $ is heavily technical we refer to xyz
\end{proof}
\begin{corollary}\label{wasserstein_convergence_arb}
  If $(X_i)_{i \in \{1,\ldots ,N\}  }$  are i.i.d random variables with law $\mu_X$ and $\int \abs{x}^{p} \mu  < \infty $ i.e $\mu  \in  \mathcal{P}^p(\mathbb{R}^{d} ) $
 \begin{align*} 
 W_p(\mu_N,\mu ) \to  0 \qquad \text{ a.s.}
 .\end{align*}
 and 
 \begin{align*}
   \E[W_p^{p}(\mu_N,\mu ) ] \to 0
 .\end{align*}
 Where 
 \begin{align*}
   \mu_N = \frac{1}{N} \sum_{i=1}^{N} \delta_{X_i} 
 .\end{align*}
\end{corollary}
\begin{proof}
 Remember that the following convergences are equivalent 
 \begin{enumerate}
   \item $W_p(\mu_N,\mu ) \to 0$
   \item $\mu_N \rightharpoonup \mu $ and $\int \abs{x}^{p} d\mu_N \to \int \abs{x}^{p} d\mu   $
   \item $\mu_n \rightharpoonup \mu $ and $\lim_{n\to \infty} \sup_r \int_{\abs{x} \ge r} \abs{x}^p d\mu_N = 0$
 \end{enumerate}
 Note that if we fix a.s. $\omega $ then we can treat this as the deterministic case. \\[1ex]
 We already know that 
 \begin{align*}
  \mu_N \rightharpoonup \mu \text{ a.s.}
 .\end{align*}
since $(X_i)$ are i.i.d then $\abs{X_i}^{p} $ is also i.i.d  and we use the Law of large numbers
 \begin{align*}
   \int \abs{x}^{p} d\mu_N   &= \frac{1}{N}\sum_{i=1}^{N} \abs{X_i}^{p}  \xrightarrow{L.L.N.} \E[\abs{X_i}^{p} ] < \infty
 .\end{align*} 
 And we get a.s. that $W_p(\mu_N,\mu ) \to  0$\\[1ex]
 For the stronger statement 
 \begin{align*}
   \E[W^{p}(\mu_n,\mu ) ] \to  0
 .\end{align*}
  we first note that  
 \begin{align*}
   W_p^{p}(\mu_N,\mu )  &\le  2^{p-1} (W^{p}_p(\mu_N,\delta_0)  + W_p^{p}(\delta_0,\mu )) \\
                        &= 2^{p-1} (\frac{1}{N} \sum_{i=1}^{N}   \abs{X_i}^{p}  + W_p^{p}(\delta_0,\mu ))
 .\end{align*}
 then it is sufficient to show the uniform integrability of the first part 
 \begin{align*}
  \frac{1}{N}\sum_{i=1}^{N} \abs{X_i}^{p}  
 .\end{align*}
 Since $\abs{X_i}^{p} $ is integrable then there exists a convex function  $\phi $ with $\lim_{x \to \infty} \frac{\phi(x)}{x} = \infty$ and 
 \begin{align*}
   \E[\phi(\abs{X_i}^{p} )] < \infty
 .\end{align*}
 Since $\phi $ is convex we apply Jensen's inequality to get 
 \begin{align*}
   \sup_N \E[\phi \left(\frac{1}{N}\sum_{i=1}^{N} \abs{X_i}^{p}\right)  ]  \myS{Jen.}{\le } \sup_N \sum_{i=1}^{N}\E[\phi (\abs{X_i}^{p} )] =  \E[\phi(\abs{X_i}^{p} )] < \infty
 .\end{align*}
 Finally \autoref{de_la_valle}  implies the uniform integrability and we conclude by \autoref{general_dct}
 \begin{align*}
   \E[W_p^{p}(\mu_N,\mu ) ]\to 0
 .\end{align*}
\end{proof}
\hspace{0mm}\\[1ex]
All the above statement only apply to arbitrary i.i.d sequences of random variables, but in our Mean-Field-Limit we only get 
the i.i.d property at $t=0$ such that we seek to prove that even as $N \to \infty$ we nonetheless get a convergence.
\begin{remark}
 Formally our goal is to prove the convergence 
 \begin{align*}
   \E[\sup_{t} W_2^{2}(\mu_N(t),\mu(t))  ] \to 0
 .\end{align*}
\end{remark}
\newpage
\subsection{Toy Example}
Let us first consider a simple stochastic particle system given by 
\begin{assumption}\label{sde_solution_assumption_strong}
Assume drift  $b : \mathbb{R}^{d} \times  \mathcal{P}^2(\mathbb{R}^{d} ) \to  \mathbb{R}^{d} $ and diffusion $\sigma : \mathbb{R}^{d} \times  \mathcal{P}^2(\mathbb{R}^{d} ) \to \mathbb{R}^{d \times  m}  $   are Lipschitz continuous i.e. $\exists  L >0$ s.t.
 \begin{align*}
  \abs{b(X,\mu ) - b(\tilde{X},\tilde{\mu }  )} + \abs{\sigma(X,\mu ) - \sigma(\tilde{X},\tilde{\mu }  )} \le  L \left( \abs{X - \tilde{X} } + W_2(\mu ,\tilde{\mu } ) \right) 
 .\end{align*}
\end{assumption}
\begin{example}[Stochastic Toy Model]
  Let our particle system be given as in \autoref{sden} with drift and diffusion for $\nabla V \in  \text{Lip}$
 \begin{align*}
   b(X,\mu )&= \nabla V \star  \mu(X)\\
   \sigma(X,\mu ) &= \sigma_0 >0
 .\end{align*}
\end{example}
\begin{exercise}
 Think about what happens if the initial data is i.i.d but the diffusion term is 0, can you prove a convergence ?
\end{exercise}
\begin{theorem}[Convergence Of Toy Model For Fixed $N$]
  Let our \hyperref[sden]{(SDEN)} be given with drift and diffusion as above and assume they fulfill \autoref{sde_solution_assumption_strong}, then  
  for fixed $N$ we get a unique strong solution in $\mathbb{L}^{2}_{dN}([0,T]) $
\end{theorem}
\begin{proof} 
  First we note that by \autoref{sde_solution_assumption_strong} we get
  \begin{align*}
    \abs{b(X,\mu ) - b(\tilde{X},\tilde{\mu }  )} &= \abs*{\int \nabla V(X-y) d\mu(y) - \int  \nabla V(\tilde{X}-y ) d \tilde{\mu }(y) }\\
                                                  &\ge \int \abs{\nabla V(X-y) - \nabla V(\tilde{X}-y )}d\mu (y) + \abs*{\int \nabla V(\tilde{X - y}(d\mu(y) - d \tilde{\mu }(y) ) )}\\
                                                  &\myS{Lip.}{\le }L*\abs{X - \tilde{X} } + LW_1(\mu ,\tilde{\mu } ) \\
                                                  &\le L*(\abs{X-\tilde{X} } + W_2(\mu,\tilde{\mu } ))
  .\end{align*}
  Let use the notation $\mathbb{X} = (X_1^{N},\ldots ,X_N^{N}  ) \in  \mathbb{R}^{dn} $ and $\mathbb{W} = (W^{1},\ldots ,W^{N}  )$ then
  \begin{align*}
    &B(\mathbb{X}) = \begin{pmatrix} \vdots \\ b(X_i^{N},\frac{1}{N}\sum_{k=1}^{N}  \delta_{X_k} ) \end{pmatrix}_{dN} \\ 
    &\Sigma(\mathbb{X})_{dN \times mN} \ : \ \text{diag}(\Sigma(\mathbb{X})) = \begin{pmatrix} \delta(X_1,\frac{1}{N}\sum_{k=1}^{N} \delta_{X_k} ), \ldots \delta(X_N,\frac{1}{N}\sum_{k=1}^{N} \delta_{X_k} ) \end{pmatrix}
  .\end{align*}
  Then our SDE is given by 
  \begin{align*}
    d \mathbb{X}(t) = B(\mathbb{X}(t)) dt + \Sigma(\mathbb{X}(t)) d \mathbb{W}_t
  .\end{align*}
  Now if $B$ and $\Sigma $ satisfy \autoref{assumption_sde_sol} we get a solution by \autoref{sde_solution_theorem}
  \begin{align*}
    \abs{B(\mathbb{X})-B(\mathbb{Y})}_{\mathbb{R}^{dn} }^2  &= \sum_{j=1}^{N} \abs{X_j,\frac{1}{N}\sum_{k=1}^{N} \delta_{X_k}  - b(Y_j,\frac{1}{N}\sum_{k=1}^{N} \delta_{Y_k} )}  \\
                                                            &\le \sum_{j=1}^{N} 2L^2 \left( \abs{X_j-Y_j}^2  + W_2^2(\mu_N(X),\mu_N(Y))\right)   \\
                                                            &\le  4L^2 \|\mathbb{X} - \mathbb{Y}\|^2 
  .\end{align*}
  For $\Sigma $ the argument is analog where for the Wasserstein distance we used 
  Then by \autoref{sde_solution_theorem} we get a solution $X \in  L^2([0,T])$ for fixed $N$\\[1ex]
\end{proof}
\begin{remark}
 To get a bound on the Wasserstein Distance we used the following  
  \begin{align*}
    \pi  = \frac{1}{N} \sum_{k=1}^{N} \delta_{(X_k,Y_k)}  \in  \Pi
  .\end{align*}
  then  the Wasserstein distance is given by 
  \begin{align*}
   \frac{1}{N}\sum_{k=1}^{N} \abs{X_k-Y_k}^2  
  .\end{align*}
  and one can further simplify to get the bound used.
\end{remark}
\begin{remark}
  As $N \to \infty$ we expect to get the following 
  \begin{align*}
    \begin{cases}
      dY^{i}(t) &= b(Y^{i}(t) ,\mu(t) ) dt + \sigma(Y^{i}(t),\mu(t) )dW_t^i \\
      Y^{i}(0)  &=  X_{i,0}^{N} \in  L^2(\Omega ) \text{ i.i.d} 
    \end{cases}
  .\end{align*}
  In fact since the above system beyond the initial data is independent of $N$, we may consider the simplified equation
  \begin{align*}
    \begin{cases}
      dY(t) &= b(Y(t) ,\mu(t) ) dt + \sigma(Y(t),\mu(t) )dW_t^i \\
      Y(0)  &=  \xi \in  L^2(\Omega ) \text{ i.i.d} 
    \end{cases}
  .\end{align*}
this equation is called Makean-Vlasov equation which is a non-linear non-local SDE
\end{remark}
\newpage
\subsection{Makean-Vlasov}
\begin{definition}[Makean-Vlasov Equation]\label{MVE}
  The following non-linear and non-local SDE is called Makean-Vlasov Equation 
\begin{align*}
    \text{(MVE)} \begin{cases}
      dY(t) &= b(Y(t) ,\mu(t) ) dt + \sigma(Y(t),\mu(t) )dW_t^i \\
      Y(0)  &=  \xi \in  L^2(\Omega ) \text{ i.i.d} 
    \end{cases}
  .\end{align*}  
\textcolor{Red}{Add Space of $Y$ and dimensions} 
\end{definition}

\begin{definition}[Space Of Continuous Sample Paths]
 The Space  $\mathcal{C}^{d} = \mathcal{C}([0,T];\mathbb{R}^{d} ) $ is called the continuous sample path space with norm 
 \begin{align*}
   \|X\|_t = \sup_{0\le t \le T} \abs{X(t)}
 .\end{align*}
 this norm $\|*\|_T$ induces a $\sigma$-algebra on $\mathcal{C}^{d} $ 
\end{definition}
\begin{definition}[Random Variable]
  A random Variable on $\mathcal{C}^{d} $ is a map 
  \begin{align*}
    X : \Omega_{\text{a.s.}} \to \mathcal{C}^{d} 
  .\end{align*}
\end{definition}
\begin{definition}[Measure]
 Since the norm $\|*\|_T$  induces a $\sigma$-algebra on $\mathcal{C}^{d} $ we can define measures $\mu \in \mathcal{P}^2(\mathcal{C}^{d} )$ by 
 \begin{align*}
   \mu  \coloneqq  (\mu(t))_{t \in  [0,T]} \qquad \mu(t)
 .\end{align*}
 and by using the function 
 \begin{align*}
  l_t : \mathcal{C}^{d} \to  \mathbb{R}^{d} \ X \mapsto X(t)  
 .\end{align*}
 then we get a measure on $\mathbb{R}^{d} $ by using the pushforward
 \begin{align*}
  \mu_t \coloneqq \mathcal{B} \to \mathbb{R}^{d}   \ A \mapsto \mu(l^{-1}_t(A) )
 .\end{align*}
\end{definition}

\begin{definition}[Wasserstein Distance]\label{c_d_wasserstein}
And we can define for arbitrary measures $\mu ,\tilde{\mu } \in  \mathcal{P}^2(\mathcal{C}^{d} ) $ the Wasserstein distance by
\begin{align*}
  \sup_{t \in  [0,T]} W_{\mathbb{R}^{d},2 }(\mu(t),\tilde{\mu }(t) ) \le  W_{\mathcal{C}^{d},2 } (\mu ,\tilde{\mu } )
.\end{align*}
Where 
\begin{align*}
  W_{\mathcal{C}^{d},2 }(\mu ,\tilde{\mu } ) = \inf_{\pi  \in  \Pi(\mu ,\tilde{\mu } )} \int_{\mathcal{C}^{d} \times  \mathcal{C}^{d}  } \|x-y \|^2 d\pi(x,y)
.\end{align*}
\end{definition}
\begin{corollary}
  Let us prove the inequality
\end{corollary}
We choose concrete $\pi_t =  l_t \# \pi $ for $\pi  \in  \Pi_{\mathcal{C}^{d}  }(\mu ,\tilde{\mu } )$
\begin{proof}
  \begin{align*}
    \sup_{t \in  [0,T]} W(\mu_t,\tilde{\mu }_t ) &\le \sup_{t \in  [0,T]} \int_{\mathbb{R}^{d} \times  \mathbb{R}^{d} } \abs{x-y}^2 d \pi_t(x,y)\\
                                                 &= \sup_{t \in  [0,T]} \int_{\mathbb{R}^{d} \times  \mathbb{R}^{d} } \abs{x-y}^2 d l_t^2 \# \pi_t(x,y)\\
                                                 &= \sup_{t \in  [0,T]} \int_{\mathcal{C}^{d}  \times  \mathcal{C}^{d} } \abs{* - *}^2 \circ l^2_t(x,y) d\pi(x,y) \\
                                                 &= \sup_{t \in  [0,T]} \int_{\mathcal{C}^{d} \times \mathcal{C}^{d}   } \abs{x(t)-y(t)}^2 d\pi(x,y)\\
                                                 &\le  \int_{\mathcal{C}^{d} \times  \mathcal{C}^{d}  } \sup_{t \in  [0,T]} \abs{x(t)-y(t)}^2 d\pi(x,y) \\
                                                 &= \int_{\mathcal{C}^{d} \times  \mathcal{C}^{d}  } \|x-y\|_\infty ^2 d\pi(x,y)
  .\end{align*}
  It remains to check that $p_t \in  \Pi(\mu_t,\tilde{\mu}_t )$, let $A \in  \mathcal{B}(\mathbb{R}^{d} )$
  \begin{align*}
    \pi_t(A \times  \mathbb{R}^{d} ) &= \pi (l_t^{-1}(A \times  \mathbb{R}^{d} )) \\
                                     &= \pi(\{(x,y) \in  \mathcal{C}^{d} \times  \mathcal{C}^{d} \ : \ l_t(x) \in  A, l_t(y) \in  \mathbb{R}^{d}   \}  )\\
                                     &=\pi (\{(x,y) \in  \mathcal{C}^{d} \times  \mathcal{C}^{d} \ : \ l_t(x) \in  A  , y \in  \mathcal{C}^{d}  \}  ) \\
                                     &= \pi(l_t^{-1}(A) \times \mathcal{C}^{d} )\\
                                     &= \mu(l_t^{-1}(A)) \\
                                     &= l_t \# \mu(A)\\
                                     &= \mu_t(A)
  .\end{align*}
\end{proof}
\begin{remark}
 Note that
 \begin{align*}
   \int_{\mathcal{C}^{d} } f(x) d\mu(x) = \int_{\mathbb{R}^{d} }  f(x(t)) d\mu_t
 .\end{align*}
\end{remark}
\begin{theorem}[Unique and Existence of Solution for Makean-Vlasov]\label{solution_vlasov}
  If $b$ and $\sigma $ satisfy \autoref{sde_solution_assumption_strong} then MVE has a unique and strong solution
  $Y \in  \mathbb{L}^2([0,T])$ and $\mu  \in  \mathcal{L}(Y)$
\end{theorem}
\begin{proof}
 We use the notation 
 \begin{align*}
   d_t^2 =  \inf_{\pi  \in  \Pi(\mu,\tilde{\mu } )} \int_{\mathcal{C}^{d} \times  \mathcal{C}^{d}  }\|x-y\|^2_t d\pi(x,y)
 .\end{align*}
 For any given $\mu  \in \mathcal{P}^2(\mathcal{C}^{d} )$ we consider the following SDE 
 \begin{align*}
   \begin{cases} 
   dY^{\mu } (t) &= b(Y^{\mu} (t),\mu(t))dt + \sigma(Y^{\mu } (t),\mu(t))dW_t\\
    Y(0)   & \xi \in  L^2(\Omega)
   \end{cases}
 .\end{align*}
 Let $\phi(\mu ) = \mathcal{L}(Y^{\mu } ) $ be the law of $Y^{\mu} $. \\[1ex]
 For the existence  and the uniqueness of $Y^{\mu } $ we need to check 
 \begin{align*}
   \abs{b(x,\mu(t)) - b(\tilde{x},\mu(t) )} + \abs{\sigma(x,\mu(t)) - \sigma(\tilde{x},\mu(t))} \le  L \abs{x-\tilde{x} }
 .\end{align*}
 Since it is the same measure the Wasserstein distance is 0 and the above is true by \autoref{sde_solution_assumption_strong}.\\
 If $\phi $ has a fixpoint $\overline{\mu } $, then $\overline{\mu } $ is the solution of MVE.
 We prove this by first bounding the difference between two measures, let $\mu ,\tilde{\mu }$ be arbitrary given measure i n
 $\mathcal{P}^2(\mathcal{C}^{d} )$, first note
 \begin{align*}
  Y^{\mu }(t) - \xi = \int_0^{t} b(Y^{\mu }(s),\mu(s) ) ds + \int_0^{t} \sigma(Y^{\mu }(s),\mu(s) ) dW_s \qquad \mu= \mu,\tilde{\mu} 
 .\end{align*}
 then by taking the difference 
 \begin{align*}
   &\sup_{0\le t \le \tau }\abs{Y^{\mu }(t) - Y^{\tilde{\mu } }(t)}^2 \\
   &= \sup_{0\le t \le s}\abs*{\int_0^{t} b(Y^{\mu }(s),\mu(s) )- b(Y^{\tilde{\mu}}(s),\tilde{\mu}(s) ) ds + \int_0^{t} \sigma(Y^{\mu }(s),\mu(s) ) - \sigma (Y^{\tilde{\mu}}(s),\tilde{\mu}(s) ) dW_s}^2\\
   &\le \sup_{0\le t \le \tau } 2t \int_0^{t} \abs{b(Y^{\mu }(s),\mu(s) )- b(Y^{\tilde{\mu}}(s),\tilde{\mu}(s) )}^2 ds \\
   &+ \sup_{0\le t \le \tau }2 \abs*{\int_0^{t} \sigma(Y^{\mu }(s),\mu(s) ) - \sigma (Y^{\tilde{\mu}}(s),\tilde{\mu}(s) ) dW_s}^2 \\
 .\end{align*}
 Now taking the expectation 
 \begin{align*}
   &\E[\sup_{0\le t \le \tau }\abs{Y^{\mu }(t) - Y^{\tilde{\mu } }(t)}^2] \\
   &\le 4\tau  L^2 \E\left[\int_0^{\tau } \abs{Y^{\mu }(s) - Y^{\tilde{\mu } }(s)  }^2 + W_2^2(\mu(s) ,\tilde{\mu}(s) ) ds \right]\\
   &+ 16 L^2 \E[\int_0^{\tau } \abs{Y^{\mu}(s) - Y^{\tilde{\mu } }(s)  }^2 + W_2^2(\mu(s),\tilde{\mu }(s) )  ds]
 .\end{align*}
 Where we used Doobs-$L^{p} $ inequality for the second term.
 \begin{align*}
   &\E[\sup_{0\le t\le \tau }\abs*{\int_0^{t} \sigma(Y^{\mu }(s),\mu(s) ) - \sigma (Y^{\tilde{\mu}}(s),\tilde{\mu}(s) ) dW_s}^2] \\
   &\le 8 \E[\int_0^{\tau } \abs{\sigma(Y^{\mu }(s),\mu(s) ) - \sigma (Y^{\tilde{\mu}}(s),\tilde{\mu}(s) )}^2 ds]\\
   &\le 8 \E[\int_0^{\tau } \abs{Y^{\mu}(s) - Y^{\tilde{\mu } }(s)  }^2 + W_2^2(\mu(s),\tilde{\mu }(s) )  ds]
 .\end{align*}
 All together 
 \begin{align*}
   \E[\|Y^{\mu } - Y^{\tilde{\mu } }\|_{\tau}^2] &\le C \int_0^{\tau } \E[\|Y^{\mu } - Y^{\tilde{\mu } }\|_s^2] ds + C \int_0^{\tau } \E[W_2^2(\mu(s),\tilde{\mu }(s) )]  ds  \\
 .\end{align*}
 So by Grönwall inequality we get 
 \begin{align*}
   \E[\|Y^{\mu } - Y^{\tilde{\mu } }\|_{\tau}^2] &\le C(\tau )* \int_0^{\tau }  W_2^2(\mu(s) ,\tilde{\mu }(s) ) ds \\                                               
                                                 &\le C(\tau )* \int_0^{\tau }  \sup_{0\le t \le s} W_2^2(\mu(t) ,\tilde{\mu }(t) ) ds \\                                              
                                                 &\le C(\tau ) \int_0^{\tau } d_s(\mu ,\tilde{\mu } )ds
 .\end{align*}
 using  the inequality \autoref{c_d_wasserstein}\\[1ex]
 remember that $\phi(\mu) = \mathcal{L}(Y^{\mu } )$ and $\phi(\tilde{\mu } ) = \mathcal{L}(Y^{\tilde{\mu } } )$, then
 \begin{align*}
   d_{\tau }^2(\phi(\mu ),\phi(\tilde{\mu } )) =  \inf_{\pi  \in  \Pi(\phi(\mu) ,\phi(\tilde{\mu } ))} \int_{\mathcal{C}^{d} \times  \mathcal{C}^{d}   } \|x-y\|_\tau^2 d\pi(x,y)
 .\end{align*}
 now if we take joint distribution of $Y^{\mu } $ and $Y^{\tilde{\mu } }$ . $\pi_1$ we can write 
 \begin{align*}
   \E[\|Y^{\mu }-Y^{\tilde{\mu } } \|_\tau^2 ] &= \int_{\mathcal{C}^{d},\mathcal{C}^{d}  } \|x-y\|_{\tau }^{2} d\pi_1(x,y)  \\
                                               &\le  C(\tau ) \int_0^{\tau } d_s(\mu ,\tilde{\mu } )ds
 .\end{align*}
 Lets summarize, for $\forall  \mu , \tilde{\mu } \mathcal{P}^2(\mathcal{C}^{d} ) $ we obtained 
 \begin{align*}
   d_t(\phi(\mu ),\phi(\tilde{\mu } )) \le C(t) \int_0^{t} d_s(\mu ,\tilde{\mu } ) ds \tag{*}
 .\end{align*}
To prove the uniqueness of solutions. If we have two solutions $\mu ,\tilde{\mu } $ i.e. 
\begin{align*}
  \phi(\mu ) &= \mu \\
  \phi(\tilde{\mu } ) &= \tilde{\mu } 
.\end{align*}
then the above estimate (*) says 
\begin{align*}
  d(\mu,\tilde{\mu } ) \le  C(t) \int_0^{t} ds(\mu ,\tilde{\mu } )  ds \implies d_t(\mu ,\tilde{\mu } )  = 0
.\end{align*}
To prove the existence. Take arbitrary $\mu_0 \in  \mathcal{P}^2(\mathcal{C}^{d} )$, (for example $\mu_0 = \mathcal{L}(\xi)$)
\begin{align*}
  \phi(\mu_0) &= \mu_1 \\
  \phi(\mu_1) &= \mu_2 \\
              &\vdots \\
  \phi(\mu_k) &= \mu_{k+1}
.\end{align*}
the estimate means that $(\mu_k)$ is Cauchy in $\mathcal{P}^2(\mathcal{C}^{d} )$
\begin{align*}
  d_t(\mu_{k+m},\mu_m) \le  \sum \ldots 
.\end{align*}
Then there exists a $\mu \in \mathcal{P}^2(\mathcal{C}^{d} )$ such that 
\begin{align*}
  W_2^{2}(\mu_k,\mu ) \to  0 
.\end{align*}
\end{proof}
\begin{remark}
 That in our case the empirical measure $\mu_N$ is not exactly the law of $X^{N} $ and is 
 stochastic, such that the above proof does not exactly holds for our (SDEN)\\
 For our initial data we already know that 
 \begin{align*}
   \E[W_2^2(\mu_N(0),\mu_0  )] \xrightarrow{N\to \infty} 0
 .\end{align*}
 and we expect for any $t>0$
 \begin{align*}
   \E[W_{\mathcal{C}^{d},2 }^2(\mu_N(t),\mu )] \to 0
 .\end{align*}
\end{remark}
\begin{theorem}[Mean-Field-Limit]
  Let $b$ and $\sigma$ fulfill \autoref{sde_solution_assumption_strong} and use 
  $\mu_N$ the empirical measure, then there exists a measure $\mu \in \mathcal{P}^2(\mathcal{C}^{d} )$ s.t. 
  \begin{align*}
    \lim_{N\to \infty}\E]W_{\mathcal{C}^{d},2 }^2(\mu_N,\mu ) = 0
  .\end{align*}
  and for any fixed $k \in  \mathbb{N}$ it holds
  \begin{align*}
    \left( X^{N}_1,\ldots ,X_k^{N}   \right)  \xRightarrow{(D)} \left( Y_{1},\ldots ,Y_{k}   \right) 
  .\end{align*}
\end{theorem}
\begin{proof}
  The proof is similar to what we have done in the \autoref{solution_vlasov}, the critical part is to 
  work with our stochastic empirical measure, we do so by introducing an intermediate empirical measure. We compute 
  \begin{align*}
    \abs{X_i^{N}(t) - Y_i(t) }^2 &\le 2 t \int_0^{t} \abs{b(X^{N}_i(s),\mu_N(s) ) -  b(Y_i(s),\mu(s) )} \\
    &+ 2 \abs*{\int_0^{t} \sigma(X_i^{N}(s),\mu_N(s) ) - \sigma(Y_i(s),\mu(s))  dW_s^{i} }^2
  .\end{align*}
   We get %(Now using Grönwall's inequality implies)
  \begin{align*}
    \frac{1}{N}\sum_{i=1}^{N} \E[\sup_{0\le r\le t} \abs{X_i^{N}(r) - Y_i(r) }^2] &\le C \E{\int_0^{t} W_2^{2}(\mu_N(s),\mu(s))  } ds\\
                                                        &\le  C*\E[\int_0^{t} d_r^2(\mu_N,\mu ) dr]
  .\end{align*}
  Let $\overline{\mu }_N $ be the empirical measure of $Y_i$ 
  \begin{align*}
    \overline{\mu }_N = \frac{1}{N} \sum_{i=1}^{N} \delta_{Y_i}  
  .\end{align*}
  And let $\mu \sim \mathcal{L}(Y_i)$ for $\forall  t > 0$ then 
  \begin{align*}
    \E[W_2^2(\overline{\mu }_N,\mu  )] \to  0
  .\end{align*}
  Now we consider for $\forall $ a.s. $\omega  \in  \Omega $
  \begin{align*}
    d_t^2(\overline{\mu }_N,\mu_N ) = \inf_{\pi  \in  \Pi(\mu_N,\overline{\mu }_N )} \int_{\mathcal{C}^{d} \times  \mathcal{C}^{d}  } \|x-y\|^2_t d\pi(x,y) 
  .\end{align*}
  By taking $\pi  = \mu_N \otimes \overline{\mu }_N $ we can write the above integral explicitly 
  \begin{align*}
    \le \frac{1}{N} \sum_{i=1}^{N} \|X^{N}_i - Y_i \|_t^2 
  .\end{align*}
  We continue by taking the expectation
  \begin{align*}
    \E[d_t^2(\mu_N,\overline{\mu }_N )] &\le  \frac{1}{N} \sum_{i=1}^{N} \E[\sup_{0\le s \le t} \|X_i(s)^{N} - Y_i(s) \|_t^2] \\
                                        &\le 2C\int_0^{t} \E[d_r^2(\mu_N,\mu )] dr 
  .\end{align*}
  Goal is to get a Grönwall inequality for 
  \begin{align*}
    \E[d_t^2(\mu_N,\mu )] &\le  2 \E[d_t^2(\mu_N,\overline{\mu }_N )] + 2 \E[d_t^2(\mu_N,\mu )]\\
                          &\le  C \int_0^{t} \E[d_r^2(\mu_N,\mu )] dr + C\E[d_t^2(\overline{\mu}_N,\mu )] \\
  .\end{align*}
  Then by Grönwall 
  \begin{align*}
    \E[d_t^2(\mu_N,\mu )] &\le e^{CT} \E[\mu_{N,0}] + e^{CT} \E[d_t^2(\overline{\mu }_N,\mu  )] \xrightarrow{N\to \infty} 0 
  .\end{align*}
  and then for $\forall  1\le k < \infty$. 
  \begin{align*}
    \E[\max_{1\le i \le  k}\sup_{0\le r\le t} \|X_i^{N}(r) - Y_i(r) \|^2] &\le \max_{1 \le i\le k} \frac{1}{N}\sum_{i=1}^{k} \E[\|X_i^{N}-Y_i \|_t^2] \\
                                                                          &\le  C*k\E[d_t^2(\mu_N,\mu   )]\\
                                                                          &\xrightarrow{N\to \infty} 0
  .\end{align*}
  This concludes the proof. \textcolor{Red}{Add small summary}
\end{proof}
