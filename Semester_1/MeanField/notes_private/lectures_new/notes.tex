\chapter{Notes to Theorems and Such}
\section{General Notion}
We consider many particle systems governed by a mean interaction kernel i.e.
\begin{align*}
  \begin{cases}
    \frac{d}{dt} X_i(t) &= \braket{K(X_i,*),\mu_N(t,*)}\\
    X_i(0) &= x_{i,0}
  \end{cases}
.\end{align*}
if $x_{i,0}$ are i.i.d RV then at time $t=0$ we  get that $\mu_N \to \mathcal{L}(X_{0})$ by L.L.N,
for $t>0$ the particles cease to be i.i.d and we need a new method to investigate the empirical measure.\\[1ex]
Our approach is to consider that $\mu_N$ solves a weak-PDE i.e.  (in the deterministic case)
\begin{align*}
  \frac{d}{dt} \braket{\mu_N,\phi } = \braket{\nabla * (\mu K \mu ),\phi }
.\end{align*}
This leads to the Mean-Field-Equation
\begin{align*}
  \text{MFE}\begin{cases}
    &\partial_t \mu_N  + \nabla * (\mu_N K \mu_N) = 0 \\
    &\mu_N(0) = \mu_0
  \end{cases}
.\end{align*}
As $N\to \infty$ and $\mu_N \to \mu$ in some sense we expect $\mu $ to solve the (limiting) MFE. 
The question arises then, if we can solve the limiting PDEs can we trace them back to the $N$ PDEs.
\section{Deterministic}
The section consists of a first showcase of tools to solve deterministic ODEs, they are 
\begin{enumerate}
  \item[\bf Existence] Picard-Iteration relying on the Lipschitz condition to give a Cauchy sequence in an appropriate space
  \item[\bf Uniqueness] This uses Grönwall mostly or some other exponential trick, also relying on the Lipschitz condition 
\end{enumerate}
We have the following pairs of O/PDEs to solve 
\begin{align*}
  &\begin{cases}
    \frac{d}{dt} X_i(t) &= \braket{K(X_i,*),\mu_N(t,*)}\\
    X_i(0) &= x_{i,0}
  \end{cases} \qquad  
\begin{cases}
    \frac{d}{dt} X(t) &= \braket{K(X,*),\mu(t)}\\
    X(0) &= x_{0}
  \end{cases}\\
  &\begin{cases}
    &\partial_t \mu_N  + \nabla * (\mu_N K \mu_N) = 0 \\
    &\mu_N(0) = \mu_0
  \end{cases} 
  \qquad  \begin{cases}
    &\partial_t \mu  + \nabla * (\mu K \mu) = 0 \\
    &\mu(0) = \mu_0
  \end{cases} \\
.\end{align*}
We first consider the well posedness of the limiting equations by considering , the characteristic flow from which we derive a solution to the limiting PDE 
for that we consider as a function of the initial conditions 
\begin{align*}
  \begin{cases}
    x(t,x_{0},\mu_0) &= \braket{K(x(t,x_{0},\mu_0),*),\mu(t,*)}\\
    x(0,x_{0},\mu_0) &= x_{0} \\
    \mu(t) &= x(t,*,\mu_0) \# \mu_0
  \end{cases}
.\end{align*}
under assumption $K$ i.e. that the (total) gradient is bounded and $x$ growing slower than linear we attain a solution by considering the iteration 
\begin{align*}
  x_i(t,y) = y + \int \int K(x_{i-1}(s,y),x_{i-1}(s,z)) d\mu_0(z) ds 
.\end{align*}
this is fairly natural to consider, just integrate the ode with respect to $t$ and initial condition $y$. We obtain the bound 
\begin{align*}
  \abs{x_i - x_{i-1}} &\le \int \int \abs{K(x_{i-1},x_{i-1})-K(x_{i-2},x_{i-2})} d\mu_0(z) dt\\
                      &\le \int \int L \norm{x_{i-1}-x_{i-2}}_{X}*(1+\abs{y}) + \norm{x_{i-1}-x_{i-2}}*(1+\abs{z})  d\mu_0(z) dt\\
                      &\le \int 2L\norm{x_{i-1}-x_{i-2}}_{X} \left(  \int 1+\abs{y} d\mu_0(z) +   \int 1+\abs{z} d\mu_{0}(z) \right) \\
.\end{align*}
Notice that $\mu_0 \in  \mathcal{P}_1$, such that the first integral is 1 and the second is finite equal to $C$
\begin{align*}
  &\le \int 2L\norm{x_{i-1}-x_{i-2}}_{X} \left(  \int 1+\abs{y} d\mu_0(z) +   \int 1+\abs{z} d\mu_{0}(z) \right) \\
  &\le (1+\abs{y})\int  L(2+C)\norm{x_{i-1}-x_{i-2}}_{X}* dt
.\end{align*}
This implies 
\begin{align*}
  \norm{x_i-x_{i-1}}_X &\le L(2+C) \int_0^{t}  \norm{x_{i-1}-x_{i-2}}_{X}\\
                       &\le \frac{(L(2+C)*t)^{i} }{(i-1)!}
.\end{align*}
We can conclude the standard Cauchy argument by using triangle inequality such that 
\begin{align*}
  \|x_n - x_m\|_X \le C(n)\sum_{i=1}^{\infty} \frac{(L(2+C)*t)^{i} }{(i-1)!} = C(n)\exp(L(2+C)\abs{t})
.\end{align*}
Where as $C(n) \to 0$.\\
Uniqueness follows by using Grönwall similar technique really.\\[1ex]
Now for a solution $x(t,x_{0},\mu_0)$ we obtain that the push forward measure solves the PDE since 
\begin{align*}
  \frac{d}{dt} \braket{\mu(t),\phi } &= - \int \partial_t \phi  d\mu(t)\\
                                     &= - \int \partial_t (\phi(x(t,y,\mu_0))) d\mu_0(y)\\
                                     &= - \int \nabla \phi * \frac{d}{dt} x(\ldots ) d\mu_0(y) \\
                                     &= - \int \nabla \phi \int_{\mathbb{R}^{d} } K(x(y),x(z)) d\mu_0(z) d\mu_0(y)\\
                                     &= - \int \nabla \phi \int_{\mathbb{R}^{d} } K(y,z) d\mu(z) d\mu(y)
.\end{align*}
This implies tho that
\begin{align*}
  \partial_t \mu_t + \nabla * (\mu K \mu ) = 0
.\end{align*}
We can further show that since $\mu_0 \in  \mathcal{P}^{1}$ then $\mu_t \in  \mathcal{P}^{1} $ and that the solutions to the MFE are unique,
Take $\mu_t$ (by a solution to the characteristic flow) then we need to prove that 
\begin{align*}
  \mu_t \sim \lambda 
.\end{align*}
We observe that if 
\begin{align*}
  \lambda(B) = 0
.\end{align*}
Then since
\begin{align*}
  \mu_t(B) = x(t,*,\mu_0) \# \mu_0(B)
.\end{align*}
and $x(t,*,\mu_0)$ is bijective $\mu_t(B) = 0 $ and 
\begin{align*}
  \|f(t,*)\|_{L^{1} } = \int \abs{f(t,y)} dy =  \int \abs{y} d \mu_t  = \int \abs{x(t,y,\mu_0)} d\mu_0  = 1
.\end{align*}
\begin{align*}
  \int_{\mathbb{R}^{d} } f(x,t) dx  = \int_{\mathbb{R}^{d} } f_0(x)  dx
.\end{align*}
Ideally we would like to use test function $\phi =1$, this is not possible such that we use 
\begin{align*}
  \phi_R(x) = j_{\frac{R}{2}} \star \cha_{[-R,R]}
.\end{align*}
And for $t>\tilde{t} \ge 0 $ we compare 
\begin{align*}
  \abs*{\int_{\mathbb{R}^{d} } f(t,x)\phi_R(x) dx - \int_{\mathbb{R}^{d} } f(\tilde{t},x ) dx } 
.\end{align*}
the PDE tells us that for shifted time $\tilde{t}$ 
\begin{align*}
  \int_{\mathbb{R}^{d} } \phi(x) f(t,x) = \int_{\mathbb{R}^{d} } \phi(x) f(\tilde{t},x ) + \int_{\tilde{t} }^{t} \int_{\mathbb{R}^{2d} }  fkf \nabla \phi dxdyds
.\end{align*}
Thus
\begin{align*}
  &\abs*{\int_{\mathbb{R}^{d} } f(t,x)\phi_R(x) dx - \int_{\mathbb{R}^{d} } f(\tilde{t},x ) dx } \\
  &=  \abs*{\int_{\tilde{t} }^{t} \int_{\mathbb{R}^{2d} } fkf \nabla \phi_{R} dx dy ds} \\
  &\le  \frac{C}{R} \abs{t-\tilde{t} }
.\end{align*}
Anyway, we have all the tools necessary to prove the mean field limit, that means if 
\begin{align*}
 W^{1}(\mu_0^{N},\mu_0) \to 0
.\end{align*}
Then 
\begin{align*}
  W^{1}(\mu ^{N}(t) , \mu_t ) \to  0 
.\end{align*}
This follows by using Dobrushin stability which tells us that for two solutions to the characteristic flow 
\begin{align*}
  x(t,*,\mu_0) , \overline{x}(t,*,\overline{\mu }_0 ) 
.\end{align*}
Their Wasserstein distance (of the push forward) is bounded as follows
\begin{align*}
  W^{1}(\mu_t,\overline{\mu}_t )  \le  e^{2 \abs{t}L}  W^{1}(\mu_0,\overline{\mu }_0 ) 
.\end{align*}
To prove the above we use the push forward structure and abuse the fact that for all $\overline{\pi }_t \in  \Pi(\mu_t,\overline{\mu }_t ) $
\begin{align*}
  \inf_{\pi_t \in \Pi(\mu_t,\overline{\mu }_t )} \int \abs{x-y} d\pi_t \le  \int \abs{x-y} d\overline{\pi}_t
.\end{align*}
Then using Grönwall and the Lipschitz constant $L$ of $K$ we obtain the bound.
For convergence in measure one can check that $Lip(\mathbb{R}^{d} )$ is dense in $C_0^{\mathbb{R}^{d} } $ using mollifiers, and simply bounding the difference by choosing 
a joint measure. 
\section{Stochastic}
When considering a Stochastic Many particle system or rather its limiting equation the Makean Vlasov equation, 
 \begin{align}\label{MVE*}
   \text{(MVE*)}\begin{cases}
   d Y(t) &= F\star\mu(t)(Y(t)) dt + \sqrt{2} dW_t\\
    Y(0) &= \xi \in  L^{2}(\Omega ), \quad \mu(t)\sim law (Y(t))\\
    \mu_0 &\sim law(\xi)
   \end{cases}
 .\end{align}
 Then by Itos formula we know that for $\forall  \phi \in  \mathcal{C}_0^{\infty} $ :
 \begin{align*}
   \phi(Y(t),t) - \phi(Y(0),0) =  \int_{0}^{t} \partial_t \phi(Y(t),t) + \nabla \phi(Y(t),t)*\partial_t Y(t) + \Delta  \Phi(Y(t),t) dt + \frac{1}{2} \int 2 \nabla \phi(Y(s),s) dW_t
 .\end{align*}
 Such that $\mu$ solves
\begin{align}\label{DiffusionPDE}
  \begin{cases}
    &\partial_t \mu  - \Delta \mu  + \nabla * (F\star\mu  \mu )  =0\\
    &\mu(0) = \mu_0
  \end{cases}
.\end{align}
This indicates that solutions should be possible even for singular interaction potential, and indeed we see later that is the case under appropriate scaling.\\[1ex]
Our goal is still to solve \autoref{MVE*} we notice that if we get a solution of \autoref{DiffusionPDE} and plug it into \autoref{MVE*}
\begin{align*}
  \begin{cases}
    dY(t) &= F \star  u(t)(Y(t)) dt + \sqrt{2}dW_t \\
     Y(0) &= \xi \in  L^{2} 
  \end{cases}
.\end{align*}
Suppose $F \star  u$ is bounded and Lipschitz then we get a solution $Y \sim \mathcal{L}(\overline{u} )$ and it holds by taking the expectation
\begin{align*}
  \int  \phi(x,t) \overline{u} (x,t) - \int \phi(x,0)u_{0}(x) dx = \ldots 
.\end{align*}
Such that our density $\overline{u} $ solves 
\begin{align*}
  \begin{cases}
    &\partial_t \overline{u}  - \Delta  \overline{u}  + \nabla * (F \star u \overline{u} ) =0\\
    &\overline{u} \rvert_{t=0} = u_{0}
  \end{cases}
.\end{align*}
i.e. if $u=\overline{u}$ then we get a solution to \autoref{MVE*}.
I think the way to interpret this, is to consider that if we can solve the PDE equation in standalone we can get back to the MVE
by comparing it to solutions of the above, and when $u = \overline{u}$ we attain a solution to the MVE, in the deterministic case.
\section{Heat Equation}
Pretty much just use fourier transform then solve the ODE , invert the transform and you get the Heat Kernel. 
basically just shows two things, one is superposition i.e. we can solve the inhomgenous pde with initial value in two steps,
Working around the singularity.
\section{Well-posedness of non local pde}
Introduces sobolev space as the space we prove the existence of our weak solution.
The main goal of this section is to sovle the non-local pde, we do so by considering a linear (local) pde i.e.
replace the non-local part (convolution) with a "constant" function and considert the PDE as an operator of this function.
When we get a fix point of this operator we get a solution to the original PDE.\\[1ex]
We do the entire thing illustrative by studying a LDE with given drift term, we solve that pde by using a fixpoint iteration, and then prove the necessary bounds 
to apply Aubin Lion. \\[1ex]
The proof of PDE($v$) goes as follows
\begin{enumerate}
  \item[\bf Existence] Solve the mollified problem (gives us smooth and bounded), this is done by similar argument as in LDE 

\end{enumerate}
