\chapter{Introduction}
Mean Field Particle Systems is about the study of particles which are represented by (stochastic) differential equations.
This course in particular is concerned with the behaviour of the system as the size grows to infinity:
\begin{definition}[Toy Mean Field Particle System]
 Let $N \in  \mathbb{N}$ then a Mean Field Particle System of first order is given by : 
 \begin{align*}
   x_1(t),\ldots ,x_n(t) \in  \mathcal{C}^{1}([0,T];\mathbb{R}^{d } )  \qquad x_i(0) = c_i
 .\end{align*}
 Where each particle satisfies 
 \begin{align*}
   d x_i = \frac{1}{N} \sum_{j=1}^{N} K(x_i,x_j) dt + \sigma dB_i(t)
 .\end{align*}
 Where $B_i$ is a Brownian motion; For $\sigma  = 0$ the system is called deterministic.
\end{definition}
\begin{comment}
  The term mean comes from the symmetric way the particles interact with each other.
\end{comment}
\begin{example}
 Example choices for K are : 
 \begin{align*}
  K(x_{i},x_{j}) = \triangledown(\abs{x_i - x_j}^2)
 .\end{align*}
 or : 
 \begin{align*}
   o \gamma  = \frac{x_i-x_j}{\abs{x_i - x_j}^{d } }
 .\end{align*}
\end{example}
\hspace{0mm}\\
Goal is to study what happens at $N \to \infty$, to do so we consider how the measure of a system converges  
\begin{definition}[(Empirical) Measure of a System]
  Consider the point measure for every $x_i : \delta_{x_{i}(t)}$ , then the measure of the System of order N is : 
 \begin{align*}
   \mu_N(t) = \frac{1}{N} \sum_{i=1}^{N} \delta_{x_{i}(t)}
 .\end{align*}
\end{definition}
\begin{assumption}
  For initial data the empirical measure of a system converges $\mu_N(0) \to \mu(0)$ where $\mu $ is absolutely continuous with respect to the Lebesgue Measure
\end{assumption}
\begin{corollary}
  By Radon Nikodym 
 \begin{align*}
  d\mu  = \rho_0 dx \quad \rho_0 \in  L^{1}(\mathbb{R}^{d} ) 
 .\end{align*}
\end{corollary}
\hspace{0mm}\\
It can be shown that $\mu $ solves a PDE, to do so we compute the derivative of $\mu $ using test functions
\begin{align*}
  \forall \phi \in  C_0^{\infty}(\mathbb{R}^{d} ) 
.\end{align*}
\begin{align*}
  \frac{d}{dt} \braket{\mu_{N}(t),\phi} = \frac{d}{dt} \int_{\mathbb{R}^{d} } \phi(x) d\mu_{N}(t)(x) &= \frac{d}{dt} \int \frac{1}{N} \sum_{i=1}^{N} \phi(x) d \delta_{x_i(t)} \\
                                                                                              &= \frac{1}{N} \sum_{i=1}^{N} \frac{d}{dt} \phi(x_i(t)) \\
                                                                                              &\myS{Chain.}{=} \frac{1}{N} \sum_{i=1}^{N}  \triangledown \phi(x_{i}(t)) \frac{d}{dt} x_i(t)\\
                                                                                              &= \frac{1}{N} \sum_{i=1}^{N} \triangledown_x \phi(x_i(t)) * \underbrace{\frac{1}{N} \sum_{j=1}^{N} K(x_i(t),x_j(t))}_{\text{Def.}}   \\
                                                                                              &=  \frac{1}{N} \sum_{i=1}^{N} \triangledown_x \phi(x_i(t))* \frac{1}{N} \sum_{j=1}^{N} \int_{\mathbb{R}^{d } }K(x_i(t),y) d\delta_{x_j(t)}(y) \\
                                                                                              &=  \frac{1}{N} \sum_{i=1}^{N} \triangledown_x \phi(x_i(t))*  \int_{\mathbb{R}^{d } }K(x_i(t),y) d\mu_{N}(t) \\
                                                                                              &= \int_{\mathbb{R}^{d} } \triangledown \phi(x) \int_{\mathbb{R}^{d } } K(x,y) d\mu_{N}(t,y) d\mu_N(t,x)  \\
.\end{align*}
Where the last line can be rewritten by using Integration by Parts (Divergence Theorem) :
\begin{align*}
  \int_{\mathbb{R}^{d} } \triangledown \phi(x) \int_{\mathbb{R}^{d } } K(x,y) d\mu_{N}(t,y) d\mu_N(t,x)  &\myS{Part.}{=} - \braket{\triangledown*(\mu_N \int_{\mathbb{R}^{d }} K(\cdot,y)d\mu_{N}(y)),\phi} \\
.\end{align*}
This means $\mu $ satisfies : 
\begin{align*}
  &\partial_t \mu_N + \braket{\triangledown*(\mu_N \int_{\mathbb{R}^{d }} K(\cdot,y)d\mu_{N}(y)),\phi} = 0
  &\xrightarrow{N\to \infty} \partial_t \mu + \braket{\triangledown*(\mu \int_{\mathbb{R}^{d }} K(\cdot,y)d\mu(y)),\phi} = 0
.\end{align*}
In practical applications (Theoretical Physics , Biology) systems that are considered are often of second order
\begin{definition}[Toy Second Order System]
  Given $N \in  \mathbb{N} $ a Second Order System is given by 
  \begin{align*}
    (x_i(t),v_i(t)),\ldots ,(x_{N}(t),v_{N}(t)) \in  \mathbb{R}^{2d } 
  .\end{align*}
  Such that : 
  \begin{align*}
    \frac{d}{dt} x_i(t) &= v_i(t)  \\
    \frac{d}{dt} v_i(t) &= \frac{1}{N} \sum_{j=1}^{N} F(\underbrace{x_i(t),v_i(t)}_{\text{\tiny Position and Velocity of itself}} ; x_{j}(t),v_j(t))  + \sigma \frac{d B_t}{dt}\\ 
  .\end{align*}
\end{definition}
\begin{example}[Gravitational Force]
  An example of F could be : 
  \begin{align*}
    F(x,v,y,u) = \frac{x-y}{\abs{x-y}^{d} }
  .\end{align*}
\end{example}
\begin{definition}[Second Order Measure]
 The Measure of a second order System is given by : 
 \begin{align*}
   \mu_{N}(x,v) = \frac{1}{N} \sum_{i=1}^{N} \delta_{(x_{i}(t),v_i(t))} 
 .\end{align*}
\end{definition}
\begin{exercise}
  Show what PDE $\mu $ solves for $\sigma = 0$, \textit{Hint :} Calculate  $\frac{d}{dt} <\mu_N,\phi> $ for some test function $\phi  \in C_{0}^{\infty}(\mathbb{R}^{2d} ) $
\end{exercise}
