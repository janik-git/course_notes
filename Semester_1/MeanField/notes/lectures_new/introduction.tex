\chapter{Model description and Introduction}
The following chapter will outline how the relevant particle models 
are defined, we differentiate between first and second order systems
focusing here on first order systems while leaving the second order setting
as exercises 
\section{1st Order Particle Systems}
\begin{definition}[1st Order Particle System]
  We consider a system of $N$ particles and denote by $(x_{1}(t),x_{2}(t),\ldots,x_N(t)) \in  \mathcal{C}^{1}([0,T];\mathbb{R}^{d} ),\ i=1,\ldots ,N $ 
  the trajectories of the particles.\\[1ex]
  Our first order system is then governed by the system of ordinary differential equations 
  \begin{align*}
    \begin{cases}
      d x_i(t) &= \frac{1}{N}\sum_{j=1}^{N} K(x_{i},x_{j}) dt, \quad 1\le i \le N  \\
        x_i(t)\rvert_{t=0} &= x_i(0)
    \end{cases}
  .\end{align*}
  where $K : \mathbb{R}^{2d} \to \mathbb{R}^{d}  $ is a given function.
\end{definition}
We consider the following examples for $K$ 
\begin{example}
A common example for a well-behaved $K$ is 
\begin{align*}
  K(x,y) =  \nabla (\abs{x-y}^2)
.\end{align*}
which is a locally Lipschitz continuous function. \\[1ex]
Another typical interaction  force is the potential field given by Coulomb potential, namely 
\begin{align*}
  K(x,y) = \nabla \frac{1}{\abs{x-y}^{d-2}} = \frac{x-y}{\abs{x-y}^{d} }
.\end{align*}
which is not continuous
\end{example}
\begin{definition}[Empirical Measure]\label{empirical_measure}
 For a set of particles\\
 $(x_{1}(t),x_{2}(t),\ldots,x_N(t)) \in  \mathcal{C}^{1}([0,T];\mathbb{R}^{d} ),\ i=1,\ldots ,N $ we define the empirical measure by 
 \begin{align*}
   \mu^{N}(t) \triangleq \frac{1}{N} \sum_{j=1}^{N} \delta_{x_i(t)} 
 .\end{align*}
\end{definition}
Our goal is the study of the limit of this system as $N \to  \infty$. An appropriate quantity is to consider the empirical measure \ref{empirical_measure}.
If the initial empirical measure converges in some sense to a measure $\mu(0)$ i.e. 
\begin{align*}
  \mu^{N}(0) \to \mu(0)
.\end{align*}
would $\mu^N(t)$ also converge to some measure $\mu(t)$ ?
\begin{align*}
  \mu^N(t) \xrightarrow{?} \mu(t)
.\end{align*}
Furthermore, can we find an equation which $\mu(t)$ satisfies and in which sense does it satisfy this equation?\\[1ex]
\begin{example}
  Consider the following case when the limit measure $\mu(t)$ is absolutely continuous with respect the Lebesgue measure, this means that 
\begin{align*}
  d\mu(0,x) = \rho_0(x) dx \quad \rho_0 \in  L^{1}(\mathbb{R}^{d} ) 
.\end{align*}
would the limit function have the same property ?
\end{example}
\section{Motivation For Partial Differential Equation}
Let us derive the partial differential equation that $\mu(t)$ should satisfy by 
considering the following calculation \\[1ex]
For $\forall \phi  \in  \mathcal{C}_0^{\infty}(\mathbb{R}^{d} ) $ we consider  
\begin{align*}
  \frac{d}{dt} \braket{\mu^N(t),\phi } &\triangleq \frac{d}{dt} \int_{\mathbb{R}^{d} } \phi(x) d\mu^N(t,x)  \\
                                       &\myS{Def.}{=} \frac{d}{dt} \int_{\mathbb{R}^{d} } \frac{1}{N} \sum_{j=1}^{N} \phi(x)d\delta_{x_{i}(t)}\\
                                       &\myS{Lin.}{=} \frac{1}{N} \frac{d}{dt} \phi (x_i(t))\\
                                       &= \frac{1}{N} \sum_{j=1}^{N} \nabla \phi(x_i(t)) *\frac{d}{dt} x_i(t)
.\end{align*}

