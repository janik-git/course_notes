\documentclass[10pt]{beamer}
% 
% basics
\usepackage[utf8]{inputenc}
\usepackage[T1]{fontenc}
\usepackage{listings}
% \usepackage{fontspec}
% \setmainfont[
% BoldFont = {Sans Serif}
% ]{\normalfont}
\usepackage{textcomp}
% \usepackage[dutch]{babel}
\usepackage{url}
\usepackage{hyperref}
\hypersetup{
  linkcolor  = mylinkcolor!85!black,
  citecolor  = mycitecolor!85!black,
  urlcolor   = myurlcolor!85!black,
  colorlinks = true,
}
\usepackage{graphicx}
\usepackage{float}
\usepackage{booktabs}
\usepackage{enumitem}
% \usepackage{parskip}
\usepackage{emptypage}
\usepackage{subcaption}
\usepackage{multicol}
\usepackage[usenames,dvipsnames]{xcolor}

\usepackage{pgfplots}

\usepackage{amsmath, amsfonts, mathtools, amsthm, amssymb}
\usepackage{mathrsfs}
\usepackage{cancel}
\usepackage{bm}
% font to use computer modern bright
\usepackage{cmbright}
% \usepackage{bbm}

\usepackage{systeme}
\usepackage{stmaryrd} % for \lightning
% % figure support
% \usepackage{import}
% \usepackage{xifthen}
% \pdfminorversion=7
% \usepackage{pdfpages}
% \usepackage{transparent}
% \newcommand{\incfig}[1]{%
%     \def\svgwidth{\columnwidth}
%     \import{./figures/}{#1.pdf_tex}
% }

\usepackage{thmtools}
\usepackage[framemethod=TikZ]{mdframed}
% %http://tex.stackexchange.com/questions/76273/multiple-pdfs-with-page-group-included-in-a-single-page-warning

\usepackage{fancyhdr}

\usepackage{tcolorbox}
\pdfsuppresswarningpagegroup=1


\definecolor{correct}{HTML}{009900}
\definecolor{mylinkcolor}{HTML}{F79359}
\definecolor{mycitecolor}{HTML}{E46264}
\definecolor{myurlcolor}{HTML}{DB9C98}


% \renewcommand\bfseries\ssfamily{b}
% \bfseries\sffamily

\usepackage{marginnote}
\usepackage{tablefootnote}
% fix notes 
\makeatletter
\newcommand{\spewnotes}{%
\tfn@tablefootnoteprintout%
\global\let\tfn@tablefootnoteprintout\relax%
\gdef\tfn@fnt{0}%
}
\makeatother

% 
% basics
\usepackage[utf8]{inputenc}
\usepackage[T1]{fontenc}
\usepackage{listings}
% \usepackage{fontspec}
% \setmainfont[
% BoldFont = {Sans Serif}
% ]{\normalfont}
\usepackage{textcomp}
% \usepackage[dutch]{babel}
\usepackage{url}
\usepackage{hyperref}
\hypersetup{
  linkcolor  = mylinkcolor!85!black,
  citecolor  = mycitecolor!85!black,
  urlcolor   = myurlcolor!85!black,
  colorlinks = true,
}
\usepackage{graphicx}
\usepackage{float}
\usepackage{booktabs}
\usepackage{enumitem}
% \usepackage{parskip}
\usepackage{emptypage}
\usepackage{subcaption}
\usepackage{multicol}
\usepackage[usenames,dvipsnames]{xcolor}

\usepackage{pgfplots}

\usepackage{amsmath, amsfonts, mathtools, amsthm, amssymb}
\usepackage{mathrsfs}
\usepackage{cancel}
\usepackage{bm}
% font to use computer modern bright
\usepackage{cmbright}
% \usepackage{bbm}

\usepackage{systeme}
\usepackage{stmaryrd} % for \lightning
% % figure support
% \usepackage{import}
% \usepackage{xifthen}
% \pdfminorversion=7
% \usepackage{pdfpages}
% \usepackage{transparent}
% \newcommand{\incfig}[1]{%
%     \def\svgwidth{\columnwidth}
%     \import{./figures/}{#1.pdf_tex}
% }

\usepackage{thmtools}
\usepackage[framemethod=TikZ]{mdframed}
% %http://tex.stackexchange.com/questions/76273/multiple-pdfs-with-page-group-included-in-a-single-page-warning

\usepackage{fancyhdr}

\usepackage{tcolorbox}
\pdfsuppresswarningpagegroup=1


\definecolor{correct}{HTML}{009900}
\definecolor{mylinkcolor}{HTML}{F79359}
\definecolor{mycitecolor}{HTML}{E46264}
\definecolor{myurlcolor}{HTML}{DB9C98}


% \renewcommand\bfseries\ssfamily{b}
% \bfseries\sffamily

\usepackage{marginnote}
\usepackage{tablefootnote}
% fix notes 
\makeatletter
\newcommand{\spewnotes}{%
\tfn@tablefootnoteprintout%
\global\let\tfn@tablefootnoteprintout\relax%
\gdef\tfn@fnt{0}%
}
\makeatother


\newcommand{\sqalg}{\textbf{SqAlg} }
\newcommand{\myS}[2]{\mathrel{\stackrel{\makebox[0pt]{\mbox{\normalfont\tiny #1}}}{#2}}}
\newcommand\N{\ensuremath{\mathbb{N}}}
\newcommand\R{\ensuremath{\mathbb{R}}}
\newcommand\Z{\ensuremath{\mathbb{Z}}}
\renewcommand\O{\ensuremath{\emptyset}}
\newcommand\Q{\ensuremath{\mathbb{Q}}}
\newcommand\C{\ensuremath{\mathbb{C}}}

% \usepackage{systeme}
\let\svlim\lim\def\lim{\svlim\limits}
\let\implies\Rightarrow
\let\impliedby\Leftarrow
\let\iff\Leftrightarrow
\let\epsilon\varepsilon

% \usepackage{stmaryrd} % for \lightning
\newcommand\contra{\scalebox{1.1}{$\lightning$}}
% \let\phi\varphi

%%%% commands 
\makeatother
\DeclareMathOperator*{\argmax}{arg\,max}
\DeclareMathOperator*{\argmin}{arg\,min}
\DeclareMathOperator*{\supp}{supp}
\DeclareMathOperator{\sgn}{sgn}

\DeclareMathOperator*{\regsq}{Reg^{Sq}}

\DeclarePairedDelimiter\abs{\lvert}{\rvert}
\DeclarePairedDelimiter\norm{\lVert}{\rVert}
\DeclarePairedDelimiter{\braket}\langle\rangle
\makeatletter
%%%

\usetheme{metropolis}
\usepackage{appendixnumberbeamer}

\usepackage{booktabs}
\usepackage[scale=2]{ccicons}

\usepackage{pgfplots}
\usepgfplotslibrary{dateplot}

\usepackage{xspace}
\newcommand{\themename}{\textbf{\textsc{metropolis}}\xspace}

\usepackage{bm}
\usepackage{amsmath, amsfonts, mathtools, amsthm, amssymb}
\DeclarePairedDelimiter\abs{\lvert}{\rvert}
\begin{document}
\begin{frame}{What is the MVE}
The Mckean-Vlasov Equation is in a sense the limiting equation of a Stochastic Many Particle System 
\begin{align*}
  \text{(SDEN)} \begin{cases}
    &d X_i^N(t) = b(X_i^{N}(t),\mu_N(t) ) dt + \sigma(X_i^{N}(t),\mu_N(t) )dW_t^i\\
    &X_i^N(0)  = X_{i,0}^{N} 
  \end{cases}
.\end{align*}
Now as $N\to \infty$ we get 
\begin{align*}
  \text{(MVE)} \begin{cases}
    &d Y(t) = b(Y(t),\mu(t) ) dt + \sigma(Y(t),\mu(t) )dW_t\\
    &Y(0)  = \xi \in  L^{2}  \\
    &\mu \sim \mathcal{L}(Y)
  \end{cases}
.\end{align*}
\end{frame}
\begin{frame}{What is the MVE}
  Using an SDE approach we get a Solution to the MVE  as long as $b$ and $\sigma $ are Lipschitz.
  The Mean-Field-Limit can then be formulated by considering an intermediate Empirical Measure (of $Y_i$), then 
  \begin{align*}
    \mathbb{E}[d_t^2(\mu_N,\mu )] \le  2 \mathbb{E}[d_t^2(\mu_N,\mu_N^{Y}  )] + 2 \mathbb{E}[d_t^2(\mu,\mu_N^{Y}  )]
  .\end{align*}
\end{frame}
\begin{frame}{PDE Approach}
The PDE setting makes the following observations, that if 
\begin{align*}
  b(Y(t),u) = \int F(Y(t)-y) u(y) dy = \int  F(y)u(Y(t)-y) dy
.\end{align*}
And 
\begin{align*}
  \sigma  = \sqrt{2} 
.\end{align*}
Then 
\begin{align*}
\text{(MVE*)} \begin{cases}
  &d Y(t) =  \bigg[F \star  \mu (Y(t)) \bigg]dt + \sqrt{2} dW_t\\
  &Y(0)  = \xi \in  L^{2}  \\
  &\mu(t) \sim \mathcal{L}(Y(t))
\end{cases}
.\end{align*}
Has a solution if $F \star  \mu $ is bounded Lipschitz. This allows the possibility of singularities.
\end{frame}
\begin{frame}{PDE Approach}
  We check for $\phi  \in  \mathcal{C}_0^{\infty} $ , by It\^os formula we see
\begin{align*}
  \phi(Y(t),t) - \phi(Y(0),0)  &= \int_0^{t} \partial_t \phi + \nabla \phi*(F\star \mu(s))(Y(s)) ds\\
                               &+ \int_0^{t} \Delta \phi  ds + \int_0^{t} \nabla \phi \sqrt{2}dW_s  
.\end{align*}
Then by taking the expectation we see that, $\mu $ satisfies the parabolic pde 
\begin{align*}
  \begin{cases}
    &\partial_t \mu  - \Delta \mu  + \nabla * [(F\star \mu)*\mu ] = 0 \\
    &\mu(0) = \mu_0
  \end{cases}
.\end{align*}
If $\mu$ has density $u$ then the PDE gives us additional regularity such that we can indeed consider 
"worse" $F$.
\end{frame}
\begin{frame}{Goal}
 We notice that if we get a solution $d\mu  = u$ to the above PDE, then the SDE
 \begin{align*}
\text{(MVE*)} \begin{cases}
  &d Y(t) =  (F \star  u)(Y(t)) dt + \sqrt{2} dW_t\\
  &Y(0)  = \xi \in  L^{2}  \\
\end{cases}
 .\end{align*}
 has a solution $Y$ if $F \star u$ is bounded and Lipschitz, in turn the Law of $\mathcal{L}(Y) = \overline{\mu } $ solves 
 \begin{align*}
  \begin{cases}
    &\partial_t \overline{\mu}   - \Delta \overline{\mu}  + \nabla * [(F\star \mu)*\overline{\mu} ] = 0 \\
    &\overline{\mu}(0)  = \mu_0
  \end{cases}
 .\end{align*}
 or with densities 
\begin{align*}
  \begin{cases}
    &\partial_t \overline{u}   - \Delta \overline{u}  + \nabla * [(F\star u)*\overline{u} ] = 0 \\
    &\overline{u}(0)  = u_0
  \end{cases}
 .\end{align*}
 This implies that if $u=\overline{u} = \mathcal{L}(Y) $, then we solve the MVE* (knowing the law is enough ?)
\end{frame}
\begin{frame}{Technique}
  We seek to solve the non-local parabolic PDE
\begin{align*}
  \text{FIN}\begin{cases}
    &\partial_t \mu  - \Delta \mu  + \nabla * [(F\star \mu)*\mu ] = 0 \\
    &\mu(0) = \mu_0
  \end{cases}
.\end{align*}
\begin{enumerate}
 \item Solve simple Heat-Equation by Heat-kernel/Fundamental-Solution Representation 
 \item We break up the above PDE into a couple intermediate ones : 
   \begin{align*}
    \text{LDE} \to \text{PDE}(v) \to \text{FIN}
   .\end{align*}
   Roughly that means, a fixpoint of PDE($v$) is a solution to FIN, we get that PDE(v) is
   well defined by the previous LDE 
\end{enumerate}
\end{frame}
\begin{frame}{Steps} 
   \begin{align*}
      \text{(LDE)}\begin{cases}
        &\partial_t u  - \Delta  u + \underbrace{\nabla \cdot (b(x,t)*u)}_{\approx f} = 0\\
        &u(0) = u_0
      \end{cases} 
   .\end{align*}
   Has a solution by Fundamental-Solution Representation 
   \begin{align*}
     u(x,t) &= \int_{\mathbb{R}^{d} }K(x-y,t)u_{0}(y) dy  &\\
            &+ \int_{0}^{t} \int_{\mathbb{R}^{d} } \nabla K(x-y,t-\tau ) * (b(y,\tau )u(y,\tau ) ) dy d\tau \\
            &= I + II
   .\end{align*}
  We need $b \in  L^{q}((0,T);L^{\infty} )$ , $u_{0} \in  L^{1}$. Tools are, 
  \begin{enumerate}
    \item Fix point iteration (acting on $u$), by contraction it is unique 
    \item $I \le \|u_{0}\|_{L^{1} }$ , (integral of $K$ is = 1)
    \item $II \le \|b\|*C $ 
  \end{enumerate}
\end{frame}
\begin{frame}{Steps} 
  \begin{align*}
  .\end{align*}
   \begin{align*}
   \text{(PDE)}_{\epsilon}\begin{cases}
    &u_t^{\epsilon} - \Delta u^{\epsilon}  + \nabla*(\tilde j_{\epsilon}\star (F \star  v (1_{\abs{x}\le \frac{1}{\epsilon}} \ u^{\epsilon})) = 0\\
    &u^{\epsilon}  \rvert_{t=0} = {j}_{\epsilon} \star (1_{\abs{x}\le \frac{1}{\epsilon}}  u_{0} )
  \end{cases}
   .\end{align*}
   Has a solution by the LDE solution
\end{frame}
\end{document}
