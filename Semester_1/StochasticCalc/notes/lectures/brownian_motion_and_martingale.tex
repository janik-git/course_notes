\chapter{Brownian Motion and Martingales}
These are merely my thoughts and notes to the stochastic calc script and will only 
include the entire statement if i had any special thoughts to it. \\[1ex]

\begin{definition}[usual conditions]
  The filtration $(\mathcal{F}_t)_{t \in [0,T]}$ is said to satisfy the usual conditions if : 
  \begin{enumerate}
    \item $\mathcal{F}_0$ contains all $\Pr$-null sets $\mathcal{N}$ ("completeness") 
    \item $\mathcal{F}_t = \mathcal{F}_{t+} \coloneqq \bigcap_{s>t} F_s$ for $t \in [0,T)$ ("right-continuity")
  \end{enumerate}
\end{definition}
\begin{remark}
 Completeness assures us that any modifications to an adapted stochastic process is again adapted (same null sets). \\
 Right-continuity can be thought of as giving us the ability to slightly peak into the future, consider the following hitting time :
 \begin{align*}
   \tau_A = \inf \{t \in [0,T] | X_t \in A\}
 .\end{align*}
 For some open set $A \subset  \mathbb{R}$ then for any $s \in  [0,T]$  the event : 
 \begin{align*}
   \{\tau_A = s\}   = \{X_s \in  \overline{A} \}  
 .\end{align*}
 Meaning that at time $s$ we do not know if $X_s$ is already in A or just right on the boundary of entering, such that we need the ability to peak slightly into
 the future.
\end{remark}
\begin{prop}
  Let $(B_t)_{t \in [0,T]}$  be a Brownian motion. The completed natural filtration $(\mathcal{F}_t)_{t \in  [0,T]}$ of a Brownian motion $(B_t)_{t \in [0,T]}$ is defined by  
  \begin{align*}
    \mathcal{F}_t =  \sigma(\mathcal{F}_t^{B},\mathcal{N} )
  .\end{align*}
  is right-continuous 
\end{prop}
\begin{proof}
  Idea is to show $\mathcal{F}_{t+} \subseteq \mathcal{F}_t$ by taking any continuous and bounded $f : \mathbb{R}^{d} \to  \mathbb{R} $ and showing that for $d \in  \mathbb{N}$ , $0 \le t_1<t_2<\ldots <t_d$
  \begin{align*}
    \E[f(B_{t_1},\ldots ,B_{t_d}) \ | \ \mathcal{F}_{t+}] \text{ is } \mathcal{F}_t \text{-measurable}
  .\end{align*}
  We know that by the properties of Brownian motions any increment $B_t - B_s $ is independent of $\mathcal{F}_s^{B} $,
  take $k \in  \{1,\ldots ,d-1\}  $ such that $t_k \le t \le t_{k+1}$. For $n \in \mathbb{N}$ large : 
  \begin{align*}
    t + \frac{1}{n} < t_{k+1}
  .\end{align*}
  and : 
  \begin{align*}
    \lim_{n \to \infty}  t+\frac{1}{n} = t
  .\end{align*}
  Idea is to first show that  $\E[f(B_{t_{1}},\ldots ,B_{t_d})\ | \ \mathcal{F}_{t+\frac{1}{n}}]$ converges against a $\mathcal{F}_t$ measurable limit 
  and converges against $\E[. \ | \ \mathcal{F}_{t+}]$ which concludes the proof. \\[1ex]
  \begin{align*}
    \E[f&(B_{t_{1}},\ldots ,B_{t_d})\ | \ \mathcal{F}_{t+\frac{1}{n}}] \\
    &=  \E[f(B_{t_{1}},\ldots,B_{t_k},\underbrace{B_{t+\frac{1}{n}}+ (B_{t_{k+1}} - B_{t+\frac{1}{n}})}_{=0},\ldots  ,\underbrace{B_{t+\frac{1}{n}}+ (B_{t_{d}} - B_{t+\frac{1}{n}})}_{=0})\ | \ \mathcal{F}_{t+\frac{1}{n}}]\\
    &= \int_{\mathbb{R}^{d-k} } f(B_{t_{1}},\ldots,B_{t_k},B_{t+\frac{1}{n}}+ x_1,\ldots  ,B_{t+\frac{1}{n}}+ x_{d-k}) \rho_n(x) dx
  .\end{align*}
  Question? why do we ignore the first $t_k$ elements \\ 
  Convergence of integral is shown by DCT against : 
  \begin{align*}
    \int_{\mathbb{R}^{d-k} } f(B_{t_{1}},\ldots,B_{t_k},B_{t}+ x_1,\ldots  ,B_{t}+ x_{d-k}) \rho(x) dx
  .\end{align*}
  Which is clearly $\mathcal{F}_t$ mb. \\
  Convergence of left hand side is shown by backward martingale theorem. \\[1ex]
  Plugging in $\char_{A}$
\end{proof}


