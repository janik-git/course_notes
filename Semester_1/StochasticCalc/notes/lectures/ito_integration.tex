\chapter{Ito-Integration}
\begin{lemma}[Itos isometry]
  For $f \in  \mathcal{H}_0^2$ (f is of form $f(\omega ,s) = \sum_{i=0}^{n-1} a_i(\omega )\cha_{(t_i]} $, $a_i$ is any random variable, its just 
  a discrete case)   
\end{lemma}
\begin{proof}
  Idea is similar to law of large numbers , i.e split the quadratic and non quadratic terms are 0 as $\E[(B_{t_{i+1}} - B_{t_i})] = 0 $ is 0 and for a brownian motion 
  the increments are Normal with mean $t_{i+1} - t_i$
\end{proof}
\begin{remark}
 The idea is to define a space of simple functions that lies dense in the space of functions (Stochastic processes )  this allows 
 us to prove properties of simple functions and transfer that to more complex one , think Stochastic 1 
\end{remark}
The following Proposition does just that and tells us that this sequence exists
\begin{prop}
  For every f $f \in  \mathcal{H}^2$ there exists a sequence  $(f_n)_{n \in  \mathbb{N}} \subset  \mathcal{H}_0^{2} $ such that 
  \begin{align*}
    \|f_n - f\|_{\mathcal{H}^{2}} \to  0 \quad \text{ as } n\to \infty
  .\end{align*}
\end{prop}
\begin{proof}
 In essence the proof first shows that we can define us $f_n $ as 
 \begin{align*}
  f_n =  \max(-n, \min(f,n))
 .\end{align*}
 This is just this 
\begin{figure}[H]
   \begin{center}
 \begin{tikzpicture}
\begin{axis}[
    ytick=false,
    xtick=false,
    ymin=-3.0, ymax=3,
    ] 
    % Define the step function
    \addplot[color=blue, thick, forget plot] {2} node[pos=0.7,below right] {$n$}; 
    \addplot[color=blue, thick, forget plot] {-2} node[pos=0.7,below right] {$-n$}; 
    \addplot[color=red, thick] {x^3 + x^2 + x};
  \end{axis}
 \end{tikzpicture}    
   \end{center}
 \end{figure}
 Then as $f_n$ is bounded by f clearly we can use $DCT$ to get the convergence, (swap integral and limit) \\[1ex]
 Second step 
\end{proof}
\begin{definition}[3.11]
 For a fixed $T \in  (0,\infty)$  we introduce 
 \begin{align*}
   \mathcal{H}_{\text{loc}}^2 \coloneqq  \{f : \Omega  \times  [0,T] \to  \mathbb{R} : f \text{ is measurable, adapted and } \int_0^{T} f^2(*,s) ds <\infty   \mathbb{P}\text{-a.s.} \}  
 .\end{align*}
 clearly $\E[I(f^2)] < \infty$  implies $f \in  \mathcal{H}_{\text{loc}}^2$ find example for f that is in $\mathcal{H}_{\text{loc}}^2$ but is not in $\mathcal{H}^2$, 
\end{definition}
\begin{remark}
  Localizing sequence is an increasing sequence $\nu_n$ of $[0,T]$  stopping times such that for $f \in  \mathcal{H}_{\text{loc}}^2$ if $f \cha_{[0,\nu_n]} \in  \mathcal{H}^2$ for all $n \in  \mathbb{N}$ and 
  \begin{align*}
    \mathbb{P}(\bigcup_{n=1}^{\infty} \{\nu_n = T\}   ) = 1
  .\end{align*}
  Question , what does the above allow us to do 
\end{remark}
\begin{example}
  Consider any $f \in  \mathcal{H}_{\text{loc}}^2$  and $\nu_n$ then 
  \begin{align*}
    \E[\int_0^{T} f^2(*,s)] ds  &= \E[\int_{0}^{T} f^2(*,s) ds *\cha_{\bigcup_{n=1}^{\infty} \{v_n = T\}  }]\\
                                &\myS{?}{=}  \int_{0}^{T} \int_{\Omega }  f^2(*,s) *\cha_{\bigcup_{n=1}^{\infty} \{v_n = T\}} d\mathbb{P} ds \\
                                &= \int_0^{T} \int_{\Omega }  \sum_{n=1}^{\infty} f^2(*,s) *\cha_{\{v_n = T\}} d\mathbb{P} ds \\ 
                                &\myS{?}{=}  \int_0^{T}  \sum_{n=1}^{\infty} \int_{\Omega }  f^2(*,s) *\cha_{\{v_n = T\}} d\mathbb{P} ds \\ 
                                &\myS{?}{=}  \sum_{n=1}^{\infty} \int_{\Omega } \int_0^{T}    f^2(*,s) *\cha_{\{v_n = T\}} d\mathbb{P} ds 
  .\end{align*}
\end{example}
