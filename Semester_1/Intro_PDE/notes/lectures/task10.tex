\begin{align*}
  &\text{Leon Fiethen 1728330}\\
 &\text{Janik Sperling 1728567}
.\end{align*}
\section*{Sheet 10}
\subsection*{30}
\begin{exercise}[a]
 Solutions of PDEs that are constant in the time variable are called "steady-state"  solutions.
 Describe steady-state solutions of the inhomogenous heat equation 
\end{exercise}
\begin{proof}
 Suppose we have a steady-state solution $u$, then  by assumption
 \begin{align*}
  \dot{u} = 0
 .\end{align*}
 such that
 \begin{align*}
  \dot{u}  - \Delta u  = f \leftrightarrow  -\Delta u = f
 .\end{align*}
 Which are all the solutions to the Poisson equation. 
\end{proof}
\begin{exercise}[b]
 Consider the heat equation $\dot{u} - \Delta  u  =0 $  on $\mathbb{R}^{n} \times  \mathbb{R}^{+}  $ with smooth initial condition 
 $u(x,0) = h(x)$. Suppose that the Laplacian an of $h$ is a constant. Show that there is a solution whose time derivative is constant
\end{exercise}
\begin{proof}
  If the time derivative is a constant then $u$ is linear in time i.e 
  \begin{align*}
    u(x,t) = u_{1} + t*c
  .\end{align*}
  where by initial condition 
  \begin{align*}
    u(x,0) = h(x) = u_{1}
  .\end{align*}
  Then we have 
  \begin{align*}
    \dot{u}  - \Delta u = 0 \leftrightarrow c -c_{2} = 0  
  .\end{align*}
  So $c = c_{2} = \Delta h$ is a solution.
\end{proof}
\begin{exercise}[c]
 Consider "translational solutions"  to the heat equation on $\mathbb{R} \times  \mathbb{R}^{+} $(i.e. n=1).
 These are solutions of the form $u(x,t)= F(x-bt)$. Find all such solutions.
\end{exercise}
\begin{proof}
 We have $u(x,t) = F(z(x,t)) = F(x-bt)$
 \begin{align*}
   \dot{u}  &=  -b F' \\
   \frac{du}{dx} &=  F'  \\
   \frac{d^2u}{dx^2} &= F''  \\
 .\end{align*}
 where $F' = \frac{d}{dz} F$, then 
 \begin{align*}
  \dot{u}  - \Delta  F = 0 \leftrightarrow -b F' - F^{''}  = 0
 .\end{align*}
 By integration we must have 
 \begin{align*}
  -b F - F' = c
 .\end{align*}
 For some constant $c$, then 
 \begin{align*}
  F' = -(c + bF)
 .\end{align*}
 Which is a first order linear ode with solution  ...
\end{proof}
\begin{exercise}[d]
  If $u$ is a solution to the heat equation, show for every $\lambda  \in  \mathbb{R}$ that 
  \begin{align*}
    u_{\lambda }(x,t) = u(\lambda x, \lambda ^2t)
  .\end{align*}
 is also a solution to the heat equation 
\end{exercise}
\begin{proof}
 We check 
 \begin{align*}
   \dot{u}_\lambda  &= \lambda^2 \dot{u}   \\
   \Delta u_\lambda  &= \lambda^2 \Delta u 
 .\end{align*}
 Then 
 \begin{align*}
  \dot{u}_\lambda - \Delta u_\lambda =  \lambda ^2(\dot{u}- \Delta u  ) =  0
  .\end{align*}
\end{proof}
\newpage
\subsection*{31. The Fourier transform}
In this question we expand on some details from Section 4.1. Recall that the Fourier transform
of a function $h(x) : \mathbb{R}^{n} \to  \mathbb{R} $ is defined to be function $\hat{h}(k) : \mathbb{R}^{n}  \to  \mathbb{R}$  given by 
\begin{align*}
  \hat{h}(k) = \int_{\mathbb{R}^{n} } e^{-2\pi ik*x} h(x)dx
.\end{align*}
Lemma 4.3 shows that it is well-defined for Schwartz functions.
\begin{exercise}[a]
  Argue that $f : \mathbb{R} \to \mathbb{R} $  given by $f(x) = \exp(-x^2)$ is a Schwartz function
\end{exercise}
\begin{proof}
 We want to check that for any $l \in  \mathbb{N}$ and $k \in  \mathbb{N}$ 
 \begin{align*}
   \sup_{x \in  \mathbb{R}} \abs{x}^{2l} \abs{f^{(k)}(x) } 
 .\end{align*}
 is bounded, first we check a couple derivatives of $f$
 \begin{align*}
   f' &=  -2xe^{-x^2}   = -2x*f\\
   f^{''} &= -2f - 2xf' =  -2f +4x^2 f \\
   f^{(3)} &= -2f' + 8xf + 4x^2f' = 4xf + 8xf + 4x^2(-2f + 4x^2f)  = 12xf -8x^2f + 16x^{4}f 
 .\end{align*}
Or in other words
\begin{align*}
  f^{(k)} = p(x)*f = p(x)e^{-x^2} 
.\end{align*}
for some $n=k+1$ order polynomial we get 
\begin{align*}
  p(x)e^{-x^2} = \frac{p(x)}{e^{x^2} } 
.\end{align*}
Where for every $d \in  \mathbb{N}$ we have by cutting off the series
\begin{align*}
  e^{x^2} = \sum_{k=0}^{\infty} \frac{\abs{x^2}^{k} }{k!} \ge  1 + \frac{x^{d}}{d!}
.\end{align*}
Now suppose $p(x)$ is a polynomial of order $n \in  \mathbb{N}$ then we have 
\begin{align*}
  \abs{p(x)} \le C \abs{x}^{n} 
.\end{align*}
For some $C > 0$  and $\abs{x} \ge 1$, Proof
\begin{align*}
  \abs{p(x)} = \abs{\sum_{i=0}^{n}  x^{i}*a_i} \le  \sum_{i=0}^{n} \abs{x^{i} a_i} \le  a_0 + \sum_{i=1}^{n} \abs{x^{i} a_i} 
.\end{align*}
then 
\begin{align*}
  \frac{\abs{p(x)}}{\abs{x}^{n} } \le  a_0 + a_n + \sum_{i=1}^{n-1} \frac{1}{\abs{x}^{n-i}a_i}   \le  C
.\end{align*}
For $\abs{x} \ge  1$ and some constant $C>0$ \\[1ex]
Then 
\begin{align*}
  \sup_{\abs{x}\ge 1} \abs{x}^{2l}\abs{p(x)e^{-x^2} } &= \sup_{\abs{x}\ge 1} \frac{\abs{x}^{2l}\abs{p(x)} }{e^{-x^2} } \\
                                                          &\le  \sup_{\abs{x}\ge 1} \frac{\abs{x}^{2l}C\abs{x^{n} } }{1+\frac{x^{d}}{d!} }\\
                                                          &=  C\sup_{\abs{x}\ge 1} \frac{\abs{x}^{d} }{1+\frac{x^{d}}{d!} }\\
                                                          &\le C\sup_{\abs{x}\ge 1} \frac{\abs{x}^{d} }{\frac{x^{d}}{d!} }\\
                                                          &=  C d!
.\end{align*}
where $d  = 2l+n$ \\
For $x \in  [-1,1]$ we get that $p(x)$ is bounded by some constant $\tilde{C} $, as it is continuous and $[-1,1]$ is compact.
Such that 
\begin{align*}
  \sup_{x \in  \mathbb{R}} \abs{x}^{2l}\abs{p(x)e^{-x^2} } \le \max\{\tilde{C},Cd! \}
.\end{align*}
for any polynomial $p(x)$ , and thus $e^{-x^2} $ is a Schwartz function.
\end{proof}
\begin{exercise}[b]
  Consider 
  \begin{align*}
    I^2= \left(\int_\mathbb{R} e^{-x^2} dx   \right)^2 =   \int_{\mathbb{R}^{2} } e^{-x^2-y^2} dx dy
  .\end{align*}
By changing to polar coordinates, compute this integral
\end{exercise}
\begin{proof}
 Using $\Phi(r,\phi ) = (r \cos \phi ,r \sin \phi ) \coloneqq (x,y)$  then 
 \begin{align*}
   J\Phi  = \begin{pmatrix} \cos(\phi ) & \sin(\phi ) \\ -r \sin(\phi ) &  r\cos(\phi )  \end{pmatrix} 
 .\end{align*}
 and 
 \begin{align*}
   \det{J\Phi } = r \cos^2 + r \sin^2 = r 
 .\end{align*}
 Where $r \in  [0,\infty)$ and $\theta \in [0,2\pi ]$
 \begin{align*}
   \int_{\mathbb{R}^{2}} e^{-x^2-y^2} dx dy &=  \int_0^{\infty} \int_0^{2\pi } r e^{-r^2(\cos^2(\phi) + \sin^2(\phi ))} d\phi dr \\
                                            &=\int_0^{\infty} \int_0^{2\pi}   r e^{-r^2} d\phi  dr \\
                                            &= 2\pi \int_0^{\infty}  r e^{-r^2} dr\\
                                            &= \pi 
 .\end{align*}
\end{proof}
So $I = \sqrt{\pi} $ like we saw in the lecture ($n=1$).
\begin{exercise}[c]
Prove the rescaling law for Fourier transforms:  if $h(x) = g(ax)$   then 
\begin{align*}
  \hat{h}(k)  = \abs{a}^{-n} \hat{g}(a^{-1}k ) 
.\end{align*}
\end{exercise}
\begin{proof}
 We compute 
 \begin{align*}
   \hat{h}(k)  &= \int_{\mathbb{R}^{n} } e^{-2\pi ik*x}h(x) dx  \\
               &=   \int_{\mathbb{R}^{n} } e^{-2\pi ik*x}g(ax) dx  \\
 .\end{align*}
 By using the transformation
 \begin{align*}
  z = ax \leftrightarrow \begin{pmatrix} z_{1} \\ \vdots \\ z_n \end{pmatrix}  = a*\begin{pmatrix} x_{1} \\ \vdots \\ x_n \end{pmatrix} 
 .\end{align*}
 Then we get the determinant 
 \begin{align*}
  \frac{1}{a^{n} } 
 .\end{align*}
  Then
 \begin{align*}
   \int_{\mathbb{R}^{n} } e^{-2\pi ik*x}g(ax) dx  &=   \int_{\mathbb{R}^{n} }\abs{a^{-n}  }e^{-2\pi ik*(z*\frac{1}{a})}g(z) dz \\
                                                  &=   \int_{\mathbb{R}^{n} }\abs{a^{-n}  }e^{-2\pi \frac{ik}{a}*z}g(z) dz \\
                                                  &= \abs{a}^{-n}\hat{g}(a^{-1}k ) 
 .\end{align*}
\end{proof}
\begin{exercise}[d]
 Prove the shift law for Fourier transforms: if $h(x) = g(x-a)$, then
 \begin{align*}
  \hat{h}(k) =e^{-2\pi ia*k} \hat{g}(k) 
 .\end{align*}
\end{exercise}
\begin{proof}
 We compute  
 \begin{align*}
  \hat{h}(k)  &= \int_{\mathbb{R}^{n} } e^{-2\pi ik*x}h(x) dx  \\
               &=   \int_{\mathbb{R}^{n} } e^{-2\pi ik*x}g(x-a) dx  \\
 .\end{align*}
 Using the transform 
 \begin{align*}
  z = x-a \leftrightarrow \begin{pmatrix} z_{1} \\ \vdots \\ z_n \end{pmatrix}  = \begin{pmatrix} x_{1}-a_{1} \\ \vdots \\ x_n-a_n \end{pmatrix} 
 .\end{align*}
 Which has determinant $1$ then 
 \begin{align*}
   \int_{\mathbb{R}^{n} } e^{-2\pi ik*x}g(x-a) dx &= \int_{\mathbb{R}^{n} } e^{-2\pi ik*(z+a)}g(z) dx\\
                                                  &= \int_{\mathbb{R}^{n} } e^{-2\pi ik*z+k*a}g(z) dx\\
                                                  &= \int_{\mathbb{R}^{n} } e^{-2\pi ik*z}*e^{-2\pi ik*a)}g(z) dx\\
                                                  &= e^{-2\pi ik*a}\int_{\mathbb{R}^{n} } e^{-2\pi ik*z}g(z) dx\\
                                                  &= e^{-2\pi ik*a} \hat{g}(k) 
 .\end{align*}
\end{proof}
\begin{exercise}[e]
 Show that $\delta $  is a tempered distribution
\end{exercise}
\begin{proof}
 We recall \\
 \begin{Definition}[Tempered]
   Suppose that $\phi_m$ is a sequence of test functions that converges to zero in $\mathcal{S}$ , i.e. $\lim_{m\to \infty} \rho_{l,\alpha }(\phi_m) = 0$  for all $l \in  \mathbb{N}, \alpha \in \mathbb{N}_0^{n}  $. We say
that $F$ is a tempered distribution $F \in  \mathcal{S}^{'} $ if $\lim_{m\to \infty} F(\phi_m) = 0$.\\
Where 
\begin{align*}
  \rho_{l,\alpha }(\phi_m) = \sup \abs{x}^{2l}\abs{\partial^\alpha \phi_m}  
.\end{align*}
 \end{Definition}
 Since $\alpha \in \mathbb{N}_0$ in our case we have 
 \begin{align*}
   \rho_{l,0}(\phi_m) = \sup \abs{x}^{2l}\abs{\phi_m(x)}  \to 0
 .\end{align*}
 and for all $m \in  \mathbb{N}$ we have by properties of $\sup$
 \begin{align*}
  \sup \abs{x}^{2l}\abs{\phi_m(x)} \ge \abs{\phi_m(0)} \ge 0
 .\end{align*}
 So we have for $\forall  m \in  \mathbb{N}$ 
 \begin{align*}
   0\le \abs{\phi_m(0)} = \abs{\delta(\phi_m)} \le \abs{x}^{2l}\abs{\phi_m(x)} \xrightarrow{m \to \infty} 0
 .\end{align*}
This shows $\delta $ is a tempered distribution.
\end{proof}
\begin{exercise}[f]
  Compute the Fourier transform of $\delta $
  \end{exercise}
  \begin{proof}
    For $\phi  \in  \mathcal{S} \subset  \mathcal{C}_0^{\infty} $, and since $\delta $ is a tempered distribution by the above $\hat{\delta }(\phi ) = \delta(\hat{\phi } )$
 \begin{align*}
   \hat{\delta}(\phi ) &= \delta(\hat{\phi}(k) )\\
                       &= \delta \left(\int_{\mathbb{R}^{n} } e^{-2\pi i (\star) *x} \phi (x) dx \right)\\
                       &= \int_{\mathbb{R}^{n} } e^{-2\pi i0 *x} \phi (x) dx \\
                       &=  \int_{\mathbb{R}^{n} }  \phi (x) dx \\
 .\end{align*}
  \end{proof}
  \begin{exercise}[g]
   Try to compute the Fourier transform of $1$ using Definition 4.8. What is the difficulty? 
  \end{exercise}
  \begin{proof}
   If we want to use Definition we identify $1$ with the distribution 
   \begin{align*}
     F_1(\phi ) = \int_{\mathbb{R}^{n}}  1*\phi 
   .\end{align*}
   We recognize this as 
   \begin{align*}
    F_1(\phi ) = \hat{\delta}(\phi )=  \delta(\hat{\phi } )
   .\end{align*}
   Since $\delta $ is tempered so is $F_1$ and we use 4.8.
   \begin{align*}
     \hat{F_1}(\phi )  = F_1(\hat{\phi } ) &= \int_{\mathbb{R}^{n} } 1*\hat{\phi}  dk\\
                                           &= \int_{\mathbb{R}^{n} } e^{2\pi ik*0} \hat{\phi } dk\\
                                           &= \mathcal{F}^{-1}(\hat{\phi})(0)\\
                                           &= \phi(0)\\
                                           &= \delta(\phi )
   .\end{align*}
  \end{proof}
\subsection*{32 One step at a time}
\begin{exercise}
 Prove the following identity for the fundamental solution in one dimension ($n=1$) 
 \begin{align*}
  \Phi(x,s+t) = \int_\mathbb{R} \Phi(x-y,t)\Phi(y,s) dy
 .\end{align*}
 Interpret this equation in the context of Theorem 4.7.
\end{exercise}
\begin{proof}
  We calculate  for $t,s>0$
  \begin{align*}
    \int_\mathbb{R} \Phi(x-y,t)\Phi(y,s) dy &= \int_\mathbb{R} \frac{1}{\sqrt{16\pi^2 ts} } e^{-\frac{\abs{x-y}^2}{4t} - \frac{\abs{y}^2}{4s}} dy
  .\end{align*}
  and for simplicity consider first  we want to get ($-A +By - Cy^2$)
  \begin{align*}
    -\frac{\abs{x-y}^2}{4t} - \frac{\abs{y}^2}{4s} &= -\frac{x^2-2xy+y^2}{4t} - \frac{y^2}{4s} \\
                                                   &=  -\frac{x^2}{4t} +\frac{x}{2t}*y - \frac{y^2}{4t} - \frac{1}{4s}y^2\\
                                                   &= -\underbrace{\frac{x^2}{4t}}_{A} +\underbrace{\frac{x}{2t}}_{B}*y - \underbrace{(\frac{1}{4t} + \frac{1}{4s})}_{C}y^2
  .\end{align*}
  Then by the hint we get for the integral $\sqrt{\frac{\pi}{C}} \exp(\frac{B^2}{4C}-A)$ 
  \begin{align*}
    \int_\mathbb{R} \Phi(x-y,t)\Phi(y,s) dy &= \int_\mathbb{R} \frac{1}{\sqrt{16\pi^2 ts} } e^{-\frac{\abs{x-y}^2}{4t} - \frac{\abs{y}^2}{4s}} dy\\
                                            &= \frac{1}{\sqrt{16\pi^2 ts} }  \sqrt{\frac{\pi }{(\frac{1}{4t}+\frac{1}{4s})}} \exp(\frac{x^{2} }{16t^2*(\frac{1}{4t}+\frac{1}{4s})}-\frac{x^2}{t})\\
                                            % &= \frac{1}{\sqrt{16\pi^2 ts} }  \sqrt{\frac{\pi }{\frac{4(t+s)}{16ts}}} \exp(\frac{x^{2} }{16t^2*\frac{4(t+s)}{16ts}}-\frac{x^2}{t})\\
                                            % &=  \frac{1}{\sqrt{16\pi^2 ts} }  \sqrt{\frac{4ts\pi }{s+t}} \exp(\frac{x^{2}s }{4t(s+t)}-\frac{(4(s+t))x^2}{4t(s+t)})\\
                                            % &= \frac{1}{\sqrt{4\pi(s+t) } }  \exp(\frac{x^{2}s - (4(s+t))x^2 }{4t(s+t)})\\
                                            % &= \frac{1}{\sqrt{4\pi(s+t) } }  \exp(\frac{x^{2}(s - 4(s+t)) }{4t(s+t)})\\
                                            &\vdots\\
                                            &= \Phi(x,s+t)
  .\end{align*}
  Where we pinky promise we did the intermediate transformations. \\[1ex]
  Theorem 4.7 says that, for $h \in  \mathcal{C}_b(\mathbb{R}^{n},\mathbb{R} )$
  \begin{align*}
    u(x,t) = \int_{\mathbb{R}^{n} } \Phi(x-y,t)h(y) d^{n} y
  .\end{align*}
  has the properties 
  \begin{enumerate}
    \item $u \in  \mathcal{C}^{\infty}(\mathbb{R}^{n} \times  \mathbb{R}^{+}  ) $
    \item $\dot{u} - \Delta  u  = 0 $
    \item $u$ extend continuously to $\mathbb{R}^{n} \times  [0,\infty) $ with $\lim_{t \to 0} u(x,t) = h(x)$
  \end{enumerate}
  Now lets say we have $u$ as given by the representation of 4.7., and take $n=1$ since we've only shown the identity for that, then 
  \begin{align*}
    u(x,t+s) &= \int_{\mathbb{R} }\Phi(x-y,t+s)h(y) d y  \\
             &= \int_{\mathbb{R} }\left(\int_{\mathbb{R}} \Phi(x-y-z,t)\Phi(z,s) dz \right)h(y) d y\\
             &= \int_{\mathbb{R} }\tilde{u}(x-z,t)\Phi(z,s) d z
  .\end{align*}
Which by Theorem 4.7 is a solution to the Cauchy problem with initial condition
\begin{align*}
  h(y) \coloneqq \tilde{u}(y,t)  
.\end{align*}
So we can always construct a new heat equation by taking the previous state as our new initial state i.e. the initial condition only matters till the immediately following 
state.
\end{proof}
