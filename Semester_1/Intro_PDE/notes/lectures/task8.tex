
\begin{align*}
  &\text{Leon Fiethen 1728330}\\
 &\text{Janik Sperling 1728567}
.\end{align*}
\section*{Sheet 7}
\subsection*{23}
\begin{question}
Suppose that $u \in  \mathcal{C}^2(\mathbb{R}^{2} )$ is harmonic with 
critical point at $x_{0}$. Assume the Hessian of $u$ has non-zero determinant. Show 
that $x_{0}$ is a saddle point. Explain the connection
to the maximum principle  
\end{question}
\begin{solution}
  Since $u \in  \mathcal{C}^2(\mathbb{R}^{2} )$  is harmonic and assume 
  assume $x_{0}$ is a maximum, then by the maximum principle $u$ must be constant and $u' = 0$, 
  therefore $H(u)$ must have 0 determinant too, for minima we argue that $x_{0}$ minima of $u$ implies
  a maxima of $-u$ and the same argument applies. \\[1ex]
  It follows if $u$ has a Hessian with non zero determinant it must have a saddle point at $x_{0}$
\end{solution}
\subsection*{24}
Let $\Omega \subset  \mathbb{R}^{n} $ be an open and connected region. A continuous function $v : \overline{\Omega } \to \mathbb{R} $
is called subharmonic if for all $x \in  \Omega $ and $r>0$ with $B(x,r) \subset  \Omega $ it lies below its spherical mean 
\begin{align*}
  v(x) \le  \mathcal{S}[v](x,r)
.\end{align*}
\begin{question}[a]
 Prove that every subharmonic function obeys the maximum-principle i.e. 
 if the maximum of $v$ can be found inside $\Omega $ then $v$ is constant
\end{question}
\begin{solution}
 Suppose $x_{0} \in  \Omega $ is the maximum of $v$ then on a ball of radius $r>0$ around $x_{0}$ we have
 \begin{align*}
   0 \ge  v(x_{0}) - S[v](x_{0},r) = \frac{1}{C} \int_{\partial B(x_{0},r)} v(x_{0}) - v(y) d\sigma(y) \ge  0
 .\end{align*}
 As from $x_{0}$ maxima it follows that  for $\forall  y \in  B(x_{0},r)$ 
 \begin{align*}
  v(x_{0}) - v(y) \ge 0
 .\end{align*}
 We conclude that for all $y \in  \partial B(x_{0},r)$ 
 \begin{align*}
  v(x_{0}) = v(y)
 .\end{align*}
 Now suppose there exists $y_{1} \in B(x_{0},r)  $ such that $\abs{v(y_{1})-v(x_{0})} > 0$ then by continuity we have 
 $\forall  x \in  B(y_{1},r)$ also (where $r>0$ is sufficiently small )$\abs{v(x)-v(x_{0})} > 0$ such that 
 \begin{align*}
   0 &< \frac{1}{C} \int_{B(y_{1},r)} v(x_{0}) - v(y) \\
     &\le \frac{1}{C} \int_{B(y_{1},r)} S[v](x_{0},r) - v(y) d\mu(y)\\ &
     = S[v](x_{0},r) - \frac{1}{C} \int_{B(y_{1},r)} v(y) d\mu(y)  
 .\end{align*}
 for $r \to 0$ we must have 
 \begin{align*}
   0 < S[v](x_{0},r) - \frac{1}{C} \int_{B(y_{1},r)} v(y) d\mu(y) \le x_{0}-y_{1} \ge  0
 .\end{align*}
 a contradiction.  (idk about this fully tbh). \\[1ex]
 Now we know that if $v$ attains a maxima $x_{0}$ it must be constant on a small ball centered at $x_{0}$ with $r>0$. 
 By compactness we can cover $\overline{\Omega }$  by finite many balls of radius $\frac{r}{2}>0$ 
 \begin{align*}
  B(\gamma_{1},\frac{r}{2}),\ldots ,B(\gamma_n,\frac{r}{2})
 .\end{align*}
 Pick $\gamma_{1} = x_{0}$ then v is constant on the first ball and the next center $\gamma_{2}$ must necessarily be contained in the ball 
 $B(x_{0},r)$ such that $v$ is also constant on this ball, by repeating this argument we get that $v$ must be constant on all balls.
\end{solution}
\begin{question}[b]
 Suppose that $v$ is twice continuous differentiable. Show that $v$ is subharmonic if and only 
 if $- \Delta v \le 0$ in $\Omega$
\end{question}
\begin{solution}
 Assume first that $-\Delta v \le 0$ in $\Omega$ and define 
 \begin{align*}
   \tilde{v}(r) = S[v(x)-v](x,r) 
 .\end{align*}
 for $x \in  \mathbb{R}$ then by the divergence theorem we get that 
 \begin{align*}
   \frac{d}{dr}\tilde{v}(r)  =  -\frac{1}{C} \int_{B(x,r)} \Delta v d\mu
 .\end{align*}
 By assumption 
 \begin{align*}
   \frac{d}{dr} \tilde{v}(r)  \le  0
 .\end{align*}
 Such that we must have for all $\tilde{r} \le r $
 \begin{align*}
   \tilde{v}(r) - \tilde{v}(\tilde{r}) \ge 0
 .\end{align*}
 But 
 \begin{align*}
   \tilde{v}(r) - \tilde{v}(\tilde{r})  &= v(x) - S[v](x,r) - (v(x) - S[v](x,\tilde{r})) \\
                                          &= S[v](x,\tilde{r}) - S[v](x,r) 
                                          &\le 0
 .\end{align*}
 i.e. 
 \begin{align*}
   S[v](x,\tilde{r}) \le S[v](x,r)
 .\end{align*}
 but by continuity of $v$ for $\tilde{r}\to 0$  we have
 \begin{align*}
   v(x) \le S[v](x,r)
 .\end{align*}
 We may extend this argument for $\forall  x \in  \Omega $ and $\tilde{r}>0$ such that $B(x,\tilde{r}) \subset  \Omega $ since we can cover
 $\Omega $ by finite many $r$ Balls.\\[1ex]
 Now assume $v$ is sub harmonic then we have for $x \in  \Omega $ and $\forall r>0$ such that
 the ball $B(x,r) \subset  \Omega $ that
 \begin{align*}
   \tilde{v} \le  0 
 .\end{align*}
 and we get again by divergence theorem that 
 \begin{align*}
   \frac{d}{dr} \tilde{v} = \frac{1}{C} \int_{B(x,r)} - \Delta v(y) d\mu(y)
 .\end{align*}
 If $ - \Delta v > 0 $ on $B(x,r)$  then $\frac{d}{dr} \tilde{v}  > 0  $ and 
 \begin{align*}
   S[v](x,r) \le  S[v](x,\tilde{r} )
 .\end{align*}
 which is a contradiction if we take $\tilde{r} \to 0 $ idk cuh.
\end{solution}
\begin{question}[c]
 Let $u : \overline{\Omega } \to  \mathbb{R} $  be a harmonic function. Show that $\|\nabla u\|^2$ is subharmonic
\end{question}
\begin{solution}
 To show that 
 \begin{align*}
  v \coloneqq  \|\nabla u \|^2 \ : \ \overline{\Omega } \to  \mathbb{R} \ x \mapsto \|\nabla u(x)\|^2 
 .\end{align*}
 is subharmonic we show by (b)  $-\Delta v \le 0$ 
 \begin{align*}
   \Delta v =  \Delta \|\nabla v\|^2 &= \nabla * \nabla \|\nabla v\|^2 \\
                                     &=  \sum_{j=1}^{n} \frac{\partial}{\partial x_j} * \sum_{i=1}^{n} \frac{\partial ^2 u}{\partial x_j \partial x_i}  \\
 .\end{align*}
 First recognize that 
\end{solution}


