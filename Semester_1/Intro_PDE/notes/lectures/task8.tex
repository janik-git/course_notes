
\begin{align*}
  &\text{Leon Fiethen 1728330}\\
 &\text{Janik Sperling 1728567}
.\end{align*}
\section*{Sheet 7}
\subsection*{23 Back in the saddle}
\begin{question}
Suppose that $u \in  \mathcal{C}^2(\mathbb{R}^{2} )$ is harmonic with 
critical point at $x_{0}$, Assume the Hessian of $u$ has non-zero determinant. Show 
that $x_{0}$ is a saddle point. Explain the connection
to the maximum principle  
\end{question}
By Ana I we know that $u$ has a saddle point if the eigenvalues of the hessian at $x_{0}$ are of opposing sign i.e.
the determinant of $\det(H(u))\le 0$
\begin{solution}
  \begin{align*}
    \det(H(u))  =  \frac{\partial ^2 u}{\partial x^2}*\frac{\partial ^2 u}{\partial y^2} - \frac{\partial u}{\partial x \partial y}^2
  .\end{align*}
  Since u is harmonic
  \begin{align*}
   \frac{\partial ^2 u }{\partial x^2} + \frac{\partial ^2 u }{\partial y ^2}  = 0 \implies \frac{\partial ^2 u }{\partial x^2} = -\frac{\partial ^2 u }{\partial y ^2}
  .\end{align*}
  then we get 
  \begin{align*}
    \det(H(u)) =   \frac{\partial ^2 u}{\partial x^2}*\frac{\partial ^2 u}{\partial y^2} - \frac{\partial u}{\partial x \partial y}^2 = - (\frac{\partial ^2 u}{\partial x^2})^2 -  (\frac{\partial u}{\partial x \partial y})^2  \le 0
  .\end{align*}
\end{solution}
\newpage
\subsection*{24 Subharmonic Functions}
Let $\Omega \subset  \mathbb{R}^{n} $ be an open and connected region. A continuous function $v : \overline{\Omega } \to \mathbb{R} $
is called subharmonic if for all $x \in  \Omega $ and $r>0$ with $B(x,r) \subset  \Omega $ it lies below its spherical mean 
\begin{align*}
  v(x) \le  \mathcal{S}[v](x,r)
.\end{align*}
\begin{question}[a]
 Prove that every subharmonic function obeys the maximum-principle i.e. 
 if the maximum of $v$ can be found inside $\Omega $ then $v$ is constant
\end{question}
\begin{solution}
 Suppose $x_{0} \in  \Omega $ is the maximum of $v$ then on a ball of radius $r>0$ around $x_{0}$ we have
 \begin{align*}
   0 \ge  v(x_{0}) - S[v](x_{0},r) = \frac{1}{C} \int_{\partial B(x_{0},r)} \underbrace{v(x_{0}) - v(y)}_{v(x_{0})\text{ is max}} d\sigma(y) \ge  0
 .\end{align*}
 We conclude that for all $y \in  \partial B(x_{0},r)$ 
 \begin{align*}
  v(x_{0}) = v(y)
 .\end{align*}
 Now we want to extend this to the entire Ball $\forall y \in  B(x_{0},r)$ and then argue that we can cover 
 $\Omega $ by Balls where this holds.\\[1ex]
 We know that
 \begin{align*}
   \int_{B(x_{0},\tilde{r} )} v(x_{0})-v(y) d\mu(y) = \int_0^{\tilde{r}} \int_{\partial B(x_{0},\tilde{r} )} v(x_{0})-v(y) d\sigma(y) dr  = 0
 .\end{align*}
 We conclude $v$ is constant on the entire Ball $\forall y \in  B(x_{0},r)$ (and note our original argument didn't depend on the value of $r$ besides the ball being contained)\\[1ex]
 Now we know that if $v$ attains a maxima $x_{0}$ it must be constant on a small ball centered at $x_{0}$ with $r>0$. 
 By compactness we can cover $\overline{\Omega }$  by finite many balls of radius $\frac{r}{2}>0$ 
 \begin{align*}
  B(\gamma_{1},\frac{r}{2}),\ldots ,B(\gamma_n,\frac{r}{2})
 .\end{align*}
 Pick $\gamma_{1} = x_{0}$ then v is constant on the first ball and the next center $\gamma_{2}$ is contained in the ball 
 $B(x_{0},r)$ (otherwise relabel) such that $v$ is also constant on this ball, by repeating this argument we get that $v$ must be constant on all balls.
\end{solution}
\newpage
\begin{question}[b]
 Suppose that $v$ is twice continuous differentiable. Show that $v$ is subharmonic if and only 
 if $- \Delta v \le 0$ in $\Omega$
\end{question}
\begin{solution}
 Assume first that $-\Delta v \le 0$ in $\Omega$ and define 
 \begin{align*}
   \tilde{v}(r) = S[v(x)-v](x,r) 
 .\end{align*}
 for $x \in  \mathbb{R}$ then by the divergence theorem we get that 
 \begin{align*}
   \frac{d}{dr}\tilde{v}(r)  =  -\frac{1}{C} \int_{B(x,r)} \Delta v d\mu
 .\end{align*}
 By assumption 
 \begin{align*}
   \frac{d}{dr} \tilde{v}(r)  \le  0
 .\end{align*}
 Such that we must have for all $\tilde{r} \le r $
 \begin{align*}
   \tilde{v}(r) - \tilde{v}(\tilde{r}) \le  0
 .\end{align*}
 But 
 \begin{align*}
   \tilde{v}(r) - \tilde{v}(\tilde{r})  &= v(x) - S[v](x,r) - (v(x) - S[v](x,\tilde{r})) \\
                                          &= S[v](x,\tilde{r}) - S[v](x,r)\\
                                          &\le 0
 .\end{align*}
 i.e. 
 \begin{align*}
   S[v](x,\tilde{r} ) \le  S[v](x,r)
 .\end{align*}
 by continuity of $v$ for $\tilde{r}\to 0$  we have
 \begin{align*}
   v(x) \le S[v](x,r)
 .\end{align*}
 which is the sub-harmonic property \\[1ex]
 Now for the case $v$ sub-harmonic implies $-\Delta v \le 0$ we do this by proving that 
 $-\Delta v > 0 $ it follows $v$ not sub harmonic, but this just means we reverse all the inequalities above and make them strict
 such that we get 
 \begin{align*}
   v(x) > S[v](x,r)
 .\end{align*}
 i.e. $v$ is not sub-harmonic
\end{solution}
\begin{question}[c]
 Let $u : \overline{\Omega } \to  \mathbb{R} $  be a harmonic function. Show that $\|\nabla u\|^2$ is subharmonic
\end{question}
\begin{solution}
  By the previous sheet we know that any partial derivative of a harmonic function is again harmonic such that 
  \begin{align*}
    \frac{\partial u}{\partial x_i}
  .\end{align*}
  is harmonic,by (d) we have that for any convex function $f \circ (\frac{\partial u}{\partial x_i})$   is subharmonic,
  since $f(x) = x^2$ is convex, and the sum of sub-harmonic is trivially also sub-harmonic.
\end{solution}
\begin{question}[d]
 Show that $f \circ u$  is sub-harmonic for any smooth convex function $f : \mathbb{R} \to \mathbb{R} $
\end{question}
\begin{solution}
 We calculate 
 \begin{align*}
   \triangle f(u(x)) &= \sum_{i=1}^{n} \frac{\partial ^2 f}{\partial x_i^2}  \\
                     &\myS{Chn.}{=} \sum_{i=1}^{n} \frac{\partial}{\partial x_i}(f'(u)*\frac{\partial u}{\partial x_i})\\
                     &= \sum_{i=1}^{n} f^{''}(u) (\frac{\partial u}{\partial x_i})^2 + \sum_{i=1}^{n} f'(u)  \frac{\partial ^2 u}{\partial x_i^2}\\
                     &= \sum_{i=1}^{n} C^{''}  (\frac{\partial u}{\partial x_i})^2 + \sum_{i=1}^{n} C^{'}  \frac{\partial ^2 u}{\partial x_i^2}  \\
                     &= C^{''} \|\nabla u\|^2 + C^{'} \Delta u \ge 0  
 .\end{align*}
it follows $- \Delta f \le 0$  since $C'' \ge 0$
\end{solution}
\begin{question}[e]
 Let $v_{1},v_{2}$  be two sub harmonic functions. Show that $v = \max(v_{1},v_{2})$ is sub harmonic 
\end{question}
\begin{solution}
 We show this for $v_{1}$  for $v_{2}$  the process is analog 
 \begin{align*}
   v_{1}(x) \myS{Ass.}{\le } S[v_{1}](x,r) = \frac{1}{C} \int_{\partial B(x,r)} v_{1}(y) d\sigma(y) \le  \frac{1}{C} \int_{\partial B(x,r)}\max(v_{1},v_{2}) d\sigma(y) 
 .\end{align*}
 this implies 
 \begin{align*}
   \max(v_{1},v_{2})(x) \le  S[v](x,r)
 .\end{align*}
\end{solution}
\subsection*{25. Never judge a book by its cover}
Let $\Omega  \subset \mathbb{R}^{n} $ be an open ,connected and bounded subset and let 
$f : \Omega  \to \mathbb{R}$ and $g_{1},g_{2} : \partial \Omega  \to \mathbb{R}$ be continuous functions.
Consider then the two Dirichlet problems
\begin{align*}
  - \Delta u  = f \quad u \rvert_{\partial \Omega } = g_k
.\end{align*}
for $k=1,2$. Let $u_{1},u_{2}$ be respective solutions such that they are twice continuously differentiable
on $\Omega $ and continuous on $\overline{\Omega } $. Show that if $g_{1}\le g_{2}$ on $\partial \Omega $ then $u_{1}\le u_{2}$ on Omega
\begin{solution}
 As always we take the difference of two solutions of the inhomogeneous equation and get a solution to the homogeneous problem i.e.
 \begin{align*}
  \tilde{u} = u_{1} - u_{2} 
 .\end{align*}
 is harmonic, such that we consider 
 \begin{align*}
   \int_{\partial \Omega } u_{1} - u_{2} d\sigma  = \int_{\partial \Omega } g_{1}-g_{2} d\sigma  \le  0 
 .\end{align*}
 By the Weak Maximum Principle (3.11) a harmonic function takes its maximum on the boundary this means that any $y \in  \Omega $ such that 
 \begin{align*}
  u_{1}(y) - u_{2}(y) \ge 0
 .\end{align*}
 presents a contradiction as $y$ would be the maxima. Thus we have on the entirety of $\Omega $ $u_{1}\le u_{2}$
\end{solution}
\subsection*{To be or not to be}
\begin{question}
  Consider the Dirichlet problem for the Laplace equation $\Delta u$ on $\Omega $ with $u = g$ on $\partial \Omega $, where
$\Omega \subset  \mathbb{R}^{n} $ is an open and bounded subset and $g$ is a continuous function. We know from the weak
maximum principle that there is at most one solution. In this question we see that for some domains, existence is not guaranteed.
Consider $\Omega  = B(0,1) \setminus \{0\}  $, so that the boundary $\partial \Omega  = \partial B(0,1) \cup \{0\}  $ consists of two components.
We write $g(x) = g_1(x)$ for $x \in  \partial B(0,1)$ and $g(0) = g_2$. Show that there does not exist a solution
for $g_1(x) = 0$ and $g_2 = 1$.
\end{question}
\begin{solution}
 Let $u$ be a solution then $u$ is harmonic and is bounded on 
\end{solution}
