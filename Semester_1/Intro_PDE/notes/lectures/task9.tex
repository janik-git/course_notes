\begin{align*}
  &\text{Leon Fiethen 1728330}\\
 &\text{Janik Sperling 1728567}
.\end{align*}
\section*{Sheet 9}
\subsection*{27. It's not easy being green}
\begin{exercise}
Suppose that $\Omega $ is a bounded domain. Prove that there is at most one Green's function on $\Omega $  
\end{exercise}
\begin{proof}
Since $\Omega $ is bounded this clearly hints at using the weak maximums principle, using property (i) of greens function we get : \\
Assume two Green's functions exists and label them $G,\tilde{G} $ then 
\begin{align*}
  u(y) &= G(x,y) - \tilde{G}(x,y) = G(x,y)-\tilde{G}(x,y) + (\underbrace{\Phi(x-y) - \Phi(x-y)}_{=0}) \\
       &= G(x,y)-\Phi(x-y)  - (\tilde{G}(x,y) - \Phi(x-y) )
.\end{align*}
Then $u(y)$ is harmonic by 3.18 (i) and by the weak maximum principle $u(y)=  0$
\end{proof}
\begin{exercise}
  On the other hand, suppose that $\Omega $  has Green's function $G_{\Omega }$ and that there exists a non-trivial solution to the Dirichlet problem
  \begin{align*}
    \Delta u  = 0 \quad u \rvert_{\partial \Omega } = 0
  .\end{align*}
  We search for a function that satisfies (i) and (ii) from 3.18, since $u$ is non-trivial 
  \begin{align*}
    \tilde{G}(x,y) = G(x,y) + u(y)  \neq G(x,y)
  .\end{align*}
  and we check (i) 
  \begin{align*}
    y \mapsto \tilde{G}(x,y)  - \Phi(x-y) =  (G(x,y)-\Phi(x-y)) + u(y)
  .\end{align*}
  Both parts extend to a harmonic function  for $x \in  \Omega $ , first half  by virtue of being a Green's function , second part is harmonic by properties of being a result to the Dirichlet Problem.\\[1ex]
  For (ii) we check again for $y \in  \partial \Omega $
  \begin{align*}
    y \mapsto \tilde{G}(x,y)  - \Phi(x-y) =  G(x,y) + u(y)
  .\end{align*}
  is $0$ since $G$ is a  Greens function and we have $u \rvert_{\partial \Omega } = 0$. 
  It follows that $\tilde{G} $ is a second Greens Function.
\end{exercise}
\subsection*{28. Do nothing by halves}
Let $H^{+}_1  = \{x = (x_{1},\ldots ,x_n)\in  \mathbb{R}^{n} | x_{1} >0 \}   $ be the upper half space and $H^{0}_1  = \{x = (x_{1},\ldots ,x_n)\in  \mathbb{R}^{n} | x_{1} =0 \}   $
be the dividing hyperplane. We call $R_1(x)  = (-x_{1},x_2,\ldots ,x_n)$ reflection in the plane $H^{0} $
\begin{exercise}[a]
  Let $u \in  \mathcal{C}^{2}(\overline{H_1^{+} } ) $ be a harmonic function that vanishes on $H_1^{0} $. Show that the function 
  \begin{align*}
    v : \mathbb{R}^{n} \to  \mathbb{R} \ x \mapsto \begin{cases}
      u(x) &\text{ for } x_{1}\ge 0\\
      -u(R_1(x)) &\text{ for } x_{1}<0\\
    \end{cases} 
  .\end{align*}
  is harmonic
\end{exercise}
\begin{proof}
 We want to check that 
 \begin{align*}
  \Delta v = 0
 .\end{align*}
 We know (by past exercise sheet) that if  $u$ is harmonic then $u(R_{1}(x))$ is harmonic also. 
 We have a possible singularity at $x_{1}=0$ we have for $x_{1}=0$ that 
 \begin{align*}
  x \in  H_1^{0} 
 .\end{align*}
 But for $x\in H_1^{0} $ we have 
 \begin{align*}
   R_{1}(x) =  \cha_{H_1^{0} }(x) = x
 .\end{align*}
  And $u(x) = 0$, since $u$ is continuous we also have 
    for $x \in B(0,\epsilon)$ that $v(x) \equiv 0$. Such that $x_{1} = 0$ is not a singularity and 
    \begin{align*}
      v \in \mathcal{C}^{2} 
    .\end{align*}
  We check the partial derivatives of $v$ for $x \in  B(0,\epsilon)$
  \begin{align*}
    \partial_{x_i} v(x) = \lim_{h\to 0} \frac{v((0,\ldots ,x_i+h,\ldots))-v(x)}{h} = 
  .\end{align*}
\end{proof}
\begin{exercise}[b]
 Show that Green's function for $H_1^{+} $  is 
 \begin{align*}
  G(x,y) = \Phi(x-y) - \Phi(R_1(x)-y)
 .\end{align*}
\end{exercise}
\begin{proof}
 We check (i)  and (ii), For (i) we first note
 \begin{align*}
 G(x,y) - \Phi(x-y) =    \Phi(R_1(x)-y)
 .\end{align*}
 we check for singularity at $R_1(x) = y$, since $x \in  H_1^{+} $ then $R_1(x)$ is in $\{x=(x_{1},\ldots ,x_n) \in  \mathbb{R}^{n} \ : \ x_{1}<0 \}  $
 but since $y \in  H^{+}_1 $ the case $R_1(x) = y$ is impossible, and since $\Phi$ is harmonic (without singularity) we get that $G(x,y) - \Phi(x-y)$ extends to a harmonic function.
 For (ii) we check for $x \in  \Omega $ and $y \in  \partial \Omega $
 \begin{align*}
 G(x,y) = \Phi(x-y) - \Phi(R_{1}(x)-y) 
 .\end{align*}
 The fundamental solution only depends on the Length  $\|R_1(x)-y\|$ which is symmetric i.e.
 \begin{align*}
  \|R_1(x) - y\| = \|x-R_1(y)\|
 .\end{align*}
 Since $y \in  \partial \Omega  \coloneqq  H_1^{0} $  we get $R_1(y) = y$ and 
  \begin{align*}
 G(x,y) = \Phi(x-y) - \Phi(R_{1}(x)-y)  = \Phi(x-y) - \Phi(x-R_1(y))  = \Phi(x-y) - \Phi(x-y) = 0
 .\end{align*}
 Thus $G$ is Green's function for $H_1^{+} $
\end{proof}
\begin{exercise}[c]
 Compute the Green's function for $B^{+} $
\end{exercise}
\begin{proof}
 By 3.20 we know that 
 \begin{align*}
   G_{B(0,1)}(x,y) = \Phi(x-y) - \Phi(\abs{x}(\tilde{x}-y ))
 .\end{align*}
 where $\tilde{x} = \frac{x}{\abs{x}^2} $\\
 We know that the greens function $B(0,1)$ must be unique, lets say we have a greens function on $B^{+} $ call it $G^{+} $, we expect 
 \begin{align*}
   G_{B(0,1)(x,y)} \rvert_{B^{+} } \equiv G^{+} 
 .\end{align*}
We consider 
\begin{align*}
  G(x,y) = \Phi(x-y) - \Phi(\abs{x}(R_1(\tilde{x})-y))
.\end{align*}
We check (i)
\begin{align*}
  G(x,y) - \Phi(x-y) = \Phi(\abs{x}(R_1(x)-y))
.\end{align*}abs
By similar argument to (b) we know the singularity is not a problem and (i) is satisfied by properties of the fundamental solution\\
For (ii) we check $x \in  B^{+}  $ and $y \in  \partial B^{+} $ , clearly the boundary $\partial B^{+} $ consists of two parts,
\begin{align*}
  \partial B^{+}   = B^{0} \cup \partial B^{+} \cap H^{+}     
.\end{align*}
We consider the cases separately,for $x \in  B^{+} $ and $y \in  B^{0} $ \\
\begin{align*}
  G(x,y) = \Phi(x-y) - \Phi(\abs{x}(R_1(\tilde{x})-y))  =  \Phi(x-y) - \Phi(\abs{x}(\tilde{x} - y ))
.\end{align*}
And notice it doesn't work out lol , we choose new Green's function such that the above is 0 ,
\begin{align*}
  \tilde{G}(x,y) = \Phi(x-y) - \Phi(R_{1}(x)-y)  - (\Phi(\abs{x}(\tilde{x} - y )) - \Phi(\abs{x}(R_{1}(\tilde{x})-y)))
.\end{align*}
then for $x \in  B^{+} $ and $y \in  B^{0} $
\begin{align*}
 \tilde{G}(x,y) &=   \Phi(x-y) - \Phi(R_{1}(x)-y)  - (\Phi(\abs{x}(\tilde{x} - y )) - \Phi(\abs{x}(R_{1}(\tilde{x})-y)))\\
                &=\Phi(x-y) - \Phi(x-R_1(y))  - (\Phi(\abs{x}(\tilde{x} - y )) - \Phi(\abs{x}(\tilde{x}-y)))\\
                &=\Phi(x-y) - \Phi(x-y)  - (\Phi(\abs{x}(\tilde{x} - y )) - \Phi(\abs{x}(\tilde{x}-y)))\\
                &= 0
.\end{align*}
similar argument to (b), for $y \in \partial B^{+} \cap H^{+} \subset  \partial B(0,1)$  we have by lecture 
\begin{align*}
  \|\abs{x}(\tilde{x}-y )\| = \abs{x-y}
.\end{align*}
and 
\begin{align*}
  \|\abs{x}(R_1(\tilde{x} -y ))\| = \abs{R_1(x)-y}
.\end{align*}
then
\begin{align*} 
 \tilde{G}(x,y) &=   \Phi(x-y) - \Phi(R_{1}(x)-y)  - (\Phi(\abs{x}(\tilde{x} - y )) - \Phi(\abs{x}(R_{1}(\tilde{x})-y)))\\
                &= \Phi(x-y) -  \Phi(\abs{x}(\tilde{x} - y ) +   \Phi(\abs{x}(R_{1}(\tilde{x})-y)) -\Phi(R_{1}(x)-y)\\
                &= 0
.\end{align*}
\end{proof}
\subsection*{29. Teach a man to fish}
\begin{exercise}[a]
 Using the Green's function of $H_1^{+} $  from the previous question, derive the following formal 
 integral representation for a solution of the Dirichlet problem 
 \begin{align*}
   \Delta u  = 0 \qquad u \rvert_{H_1^{0} }   = g
 .\end{align*}
 \begin{align*}
   u(x) = \frac{2x_{1}}{n \omega_n} \int_{H_{1}^{0} } \frac{g(z)}{\abs{x-z}^{n} } d \sigma(z)
 .\end{align*}
\end{exercise}
\begin{proof}
 We assume $g$ has sufficient regularity, by Greens representation we know
 \begin{align*}
   u(x) \coloneqq  \int_{H_1^{+}} G_{H_1^{+} }(x,y) f(y) d^{n}y  - \int_{H_1^{0} } g(z) \nabla_z G_{H_1^{+} } * N d \sigma(z)
 .\end{align*}
 solves the Dirichlet problem in fact  as $f \equiv 0$
\begin{align*}
  u(x) \coloneqq  - \int_{H_1^{0} } g(z) \nabla_z G_{H_1^{+} }(x,z) * N d \sigma(z)
.\end{align*}
We calculate for $n>2$
\begin{align*}
 \nabla G(x,z) = \nabla_z (\Phi(x-z)-\Phi(R_{1}(x)-z))  = \nabla_z (\frac{1}{n(n-2)\omega_n \abs{x-z}^{n-2} }-\frac{1}{n(n-2)\omega_n \abs{x-z}^{n-2} })
.\end{align*}
ja ka rechnen halt,
\end{proof}
\begin{exercise}[b]
 Show that if $g$ is periodic that is, there is some vector $L \in  \mathbb{R}^{n-1} $)  with 
 \begin{align*}
  g(x+L) = x
 .\end{align*}
 for all $x \in  \mathbb{R}^{n-1} $, then so is the solution
\end{exercise}
\begin{proof}
 We have our solution by 
 \begin{align*}
   u(x) = \frac{2x_{1}}{n \omega_n} \int_{H_1^{0} } \frac{g(z)}{\abs{x-z}^{n} } d \sigma(z)
 .\end{align*}
 We naively pick $L \in  \mathbb{R}^{n-1} $ such that $g(x+L) = x$ for all $x \in  \mathbb{R}^{n-1} $ and check
 \begin{align*}
   u(x + L) = \frac{2(x_{1}+L_1)}{n\omega_n} \int_{H_{1}^{0} } \frac{g(z)}{\abs{x+L-z}^{n} }
 .\end{align*}
 Consider 
 \begin{align*}
   y = z-L
 .\end{align*}
 then (we techincally have the jacobian but we may ignore that since this preserves volume and sign stays the same)
 \begin{align*}
   u(x+L) &=   \frac{2(x_{1}+L_1)}{n\omega_n} \int_{H_{1}^{0} } \frac{g(y+L)}{\abs{x-y}^{n} } d\sigma(y)\\
          &= \frac{2(x_{1}+L_1)}{n\omega_n} \int_{H_{1}^{0} } \frac{y}{\abs{x-y}^{n} } d\sigma(y)\\
 .\end{align*}

\end{proof}
