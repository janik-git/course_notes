\begin{lemma}[2.12]
 The convolution $g \star F$ of a test function $g \in  \mathcal{C}_0^{\infty}(\mathbb{R}^{n} ) $ belongs to $C^{\infty}(\mathbb{R}^{n} ) $ : 
 \begin{align*}
  g \star F : \ \mathbb{R}^{n}  \to  \mathbb{R} , \quad x \mapsto F(T_xPg)
 .\end{align*}
\end{lemma}
\begin{proof}
 The continuity argument works by realising that since F is continuous and linear  
 \begin{align*}
   \frac{\partial }{\partial x_i} F(T_xPg) = \lim_{h\to 0} \frac{F(T_{x+h}Pg) - F(T_xPg)}{h} =  F(\lim_{h\to 0} \frac{T_{x+h}Pg - T_xPg}{h})
 .\end{align*}
 The next important argument ist that because we use Riemann integration we are allowed to represent the integral as a limit over sums of  differences, thus we can swap Integral and Distribution as follows 
 \begin{align*}
  F(\int  \phi ) = F(\lim \sum \phi_i) = \lim \sum F(\phi_i ) = \int F(\phi_i )
 .\end{align*}
 This is kinda encrypted in the statement \\[1ex]
 For any $\phi  \in  \mathcal{D}(\mathbb{R}^{n} )$ appropriate Riemann sums define a sequence of finite linear combinations of function in $\{T_xPg \in  \mathcal{C}_0^{\infty}(\mathbb{R}^{n}) \ | \ x \in  \supp(\phi ) \}  $
\end{proof}
