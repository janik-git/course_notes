\section{Sheet 2}
%xsim
\begin{question}[subtitle=Inhomogeneous Transport Equation (a)]
	Find a solution $u : \mathbb{R} \to  \mathbb{R}$:
  \begin{align*}
    \frac{du}{dt} = f(t)
  .\end{align*}
\end{question}
\begin{solution}
  Integrating gives : 
  \begin{align*}
    u(t) = u(0) + \int_0^{t} f(t) dt 
  .\end{align*}
\end{solution}
\begin{question}[subtitle=Inhomogeneous Transport Equation (b)]
  Show that for every c the solution to the PDE with initial value $u(0) = c$ is unique
\end{question}
\begin{solution}
  Let $u,v \in  \mathcal{C}(\mathbb{R};\mathbb{R})$ be two solutions to the IVP then  $u-v$ is a solution to the homogeneous PDE
 \begin{align*}
  \frac{dh}{dt} = 0
 .\end{align*} 
  Meaning $u-v$ is a constant function and since both solve the IVP $u(0)=v(0)=c$  and the functions are equal. 
\end{solution}

\begin{question}[subtitle=Inhomogeneous Transport Equation (c)]
	Show that the integral term itself solves the Inhomogeneous transport equation. What
	initial value problem does it solve
\end{question}
\begin{solution}
	Taking both derivatives of the Integral part :
	\begin{align*}
		\frac{\partial u(x,t)}{\partial t} & = \frac{\partial }{\partial t}  \int_0^t f(x+(s-t)b,s)ds                          \\
		                                   & = f(x+(s-t)b,s) \vert_{s=t}  + \int_0^t \frac{\partial f(x+(s-t)b,s)}{\partial t} \\
		                                   & = f(x,t) + \int_0^t -b \triangledown f(x+(s-t)b,s)                                \\
		.\end{align*}
	Derivative is a result of chain rule and fundamental theorem of calculus (derivative of upper bound gives back original function at that value)
	And :
	\begin{align*}
		\triangledown u \coloneqq \frac{\partial u(x,t)}{\partial x} & = \frac{\partial }{\partial x}  \int_0^t f(x+(s-t)b,s)ds \\
		                                                             & = \int_0^t \triangledown f(x+(s-t)b,s) ds
		.\end{align*}
	Plugging into the PDE :
	\begin{align*}
		\frac{\partial u(x,t)}{\partial t} + b* \triangledown u =  \underbrace{f(x,t) + \int_0^t -b \triangledown f(x+(s-t)b,s)}_{= \frac{\partial u}{\partial t} }  + \underbrace{b*\int_0^t \triangledown f(x+(s-t)b,s) ds}_{=b*\triangledown u} = f(x,t)
		.\end{align*}
  Set $u(x,t) = \int_0^t f(x+(s-t)b,s) ds$ then : 
  \begin{align*}
    u(x,0) =  \int_0^0 f ds =  0
  .\end{align*}
  The Integral solves the initial value problem prob also not wrong \\[1ex]
	\begin{align*}
		\frac{\partial u_s}{\partial t} + b*\triangledown u_s & = 0      \\
		u_s(x,\tau=0)                                         & = f(x,s)
		.\end{align*}
	Where $\tau = t-s$ is a time translation to get an initial value problem (see ODE), alternatively the Cauchy Problem:
	\begin{align*}
		\frac{\partial u_s}{\partial t} + b*\triangledown u_s & = 0      \\
		u_s(x,\tau=0)                                         & = f(x,s)
		.\end{align*}

\end{solution}
\begin{question}[subtitle=Inhomogeneous Transport Equation (d)]
	Prove  that the solution to the initial value problem is unique. (You may assume that the solution to the homogeneous version is unique)
\end{question}
% \begin{solution}
% 	With the result from (c) we know that $f(x+(s-t)b,s)$ is the solution to the homogeneous problem as in (c), which by assumption is unique,
% 	and at $s=t \implies \tau = 0$ :
% 	\begin{align*}
% 		\frac{\partial u_0}{\partial t} + b*\triangledown u_0 & = 0             \\
% 		u_s(x,0)                                              & = f(x,0) = g(x)
% 		.\end{align*}
%     Has the unique solution $u(x,t) = g(x-tb)$, such that the solution to Inhomogeneous initial value problem is unique ?
% \end{solution}
\begin{solution}
  Given two solutions $u_1,u_2$ to the Inhomogeneous transport equation the difference between them : $u_1-u_2$ is a solution to the homogeneous transport equation with initial value 0,
  which by assumption is unique (compare ODE)
\end{solution}
\begin{question}[subtitle=Method of characteristics for an Inhomogeneous PDE]
 Use the method of characteristics to solve the following Inhomogeneous PDE : 
 \begin{align*}
  x \frac{\partial u}{\partial x} + y \frac{\partial u}{\partial y} = 2u  
 .\end{align*}
 for $(x,y) \in \mathbb{R}^{>0}\times \mathbb{R} $ with initial condition $u(1,y) = y$.
\end{question}
\begin{solution}
 Let $z(s) = \ln (u(x(s),y(s)))$ then : 
 \begin{align*}
   z'(s) = \underbrace{\frac{1}{u}}_{\text{outer}}* \underbrace{\left( x \frac{\partial u}{\partial x} + y \frac{\partial u}{\partial y} \right) }_{\text{inner}}
 .\end{align*}
 This means  : 
 \begin{align*}
   x'(s) &= x(s)\\
  y'(s) &= y(s)
 .\end{align*}
 Such that :
 \begin{align*}
   x(s) &= x_0*e^s\\
   y(s) &=y_0*e^s
 .\end{align*}
 Which means :
 \begin{align*}
  z'(s) = 2
 .\end{align*}
 Now given the initial condition $u(1,y) = y$ we want :
 \begin{align*}
  z(0) = \ln(u(1,y)) = \ln(y_0) \implies x_0=1 
 .\end{align*}
  Integrating $z'$ : 
  \begin{align*}
    z(t) = z(0) + \int_0^t z'(s) ds = \ln(y_0) + \int_0^t 2 ds = \ln(y_0)*2t
  .\end{align*}
  Such that : 
  \begin{align*}
    u(x(s),y(s)) = \exp(z(s)) = y_0*e^{2s}  
  .\end{align*}
  Given any point $(\hat{x},\hat{y}) \in  \mathbb{R}^{>0} \times  \mathbb{R} $ we can determine the value $u(\hat{x},\hat{y} )$ by determining
  the characteristic it lies on and the value of s and checking $z(s)$, using our above  : 
  \begin{align*}
    x(s) = \hat{x}  \implies s = \ln(\hat{x}) 
  .\end{align*}
  and : 
  \begin{align*}
    y(\ln(x)) = y_0*\hat{x}  \implies y_0 = \frac{\hat{y} }{\hat{x} }  
  .\end{align*}
  such that for any $(\hat{x},\hat{y}  )$ the value of u is given by 
  \begin{align*}
    u(x(s),y(s)) =  y_0*e^{2s} = \frac{\hat{y} }{\hat{x} }*(\hat{x} )^2 = \hat{y}\hat{x}    
  .\end{align*}
  Disregarding the characteristics : 
  \begin{align*}
    u(x,y) = xy
  .\end{align*}
  Which indeed solves the PDE. \\[1ex]
  The solution is unique as the characteristics for different $(x,y)$ cannot cross as every point in the domain belongs to exactly one characteristic as seen by the above relations
\end{solution}
\begin{question}[subtitle=Duhamel's Principle (1)]
  Consider an Inhomogeneous PDE on $\mathbb{R}^{n} \times  \mathbb{R} $ of the following form : 
  \begin{align*}
    \frac{\partial u}{\partial t}  - Lu = f(x,t), \quad u(x,0) = 0
  .\end{align*}
  where L is a linear differential operator on $\mathbb{R}^{n} $ with constant coefficients.\\
  Show that if $u_s$ is a solution to the following homogeneous equation : 
  \begin{align*}
    \frac{\partial u_s}{\partial t}  - Lu_s = 0 , \quad u_s(x,s) = f(x,s)
  .\end{align*}
  Then : 
  \begin{align*}
    u(x,t) = \int_0^t u_s(x,t) ds 
  .\end{align*}
  solves the Inhomogeneous problem
\end{question}
\begin{solution}
 Let the homogeneous problem be given as : 
  \begin{align*}
    \frac{\partial u_s}{\partial t}  - Lu_s = 0 , \quad u_s(x,s) = f(x,s)
  .\end{align*}
 and define : 
 \begin{align*}
  u(x,t) = \int_0^t u_s(x,t) ds
 .\end{align*}
 Such that : 
 \begin{align*}
   \frac{\partial u}{\partial t}  &= \frac{\partial }{\partial t}  \int_0^t u_s(x,t) ds\\ 
                                  &\myS{Leib.}{=} u_s(x,t)\vert_{s=t} + \int_0^t \frac{\partial u_s}{\partial t} ds  \\
                                  &= \underbrace{f(x,t)}_{=u_t(x,t)} + \int_0^t  \frac{\partial u_s}{\partial t} ds 
 .\end{align*}
And : 
\begin{align*}
  L u_s =  L \int_0^t u_s(x,t) ds  \myS{Descr.}{=} \int_0^t L u_s(x,t) ds
.\end{align*}
Together : 
\begin{align*}
  \frac{\partial u_s}{\partial t} + L u_s =  f(x,t) + \int_0^t \underbrace{\frac{\partial u_s}{\partial t} ds + L u_s{x,t}}_{=0} =  f(x,t)
.\end{align*}
Which shows $u(x,t) = \int_0^t u_s(x,t) ds$ is a solution to the Inhomogeneous PDE
\end{solution}
\begin{question}[subtitle=Duhamel's Principle (2)]
Use the method to solve the Inhomogeneous Transport equation
\end{question}
\begin{solution}
  Let the homogeneous pde be given as : 
  \begin{align*}
    \frac{\partial u_s}{\partial t}  + b \triangledown u_s  = 0  ,\quad u_s(x,s) =f(x,s) 
  .\end{align*}
  With initial value at $s$, then the following time translation : 
  \begin{align*}
    v_s(x,\tau ) = u_s(x,\tau +s) \implies v(x,0) = f(x,s)
  .\end{align*}
  leads to a initial value problem at 0. \\
  Solving by characteristics, let $z(\tau ) = v(x(\tau ),\tau )$:
  \begin{align*}
    z' = \frac{\partial v}{\partial x} x' + \frac{\partial v}{\partial t} 
  .\end{align*}
  Choosing $x'(\tau ) = b \implies x(\tau ) = b*\tau +x_0$: 
  \begin{align*}
    z' = 0
  .\end{align*}
  Meaning $z$ is constant i.e $v_s(x,\tau )$ is constant along the characteristics with value :
  \begin{align*}
    z(0) = v_s(x_0,0) = f(x_0,s)
  .\end{align*}
  Now for any $(x,\tau ) \in  \mathbb{R}^{n}\times \mathbb{R} $ we get $v_s(x,\tau )$ by determining the characteristics it lies on (i.e determine $x_0$):
  \begin{align*}
    x = x_0 + \tau *b \implies x_0 = x - \tau *b
  .\end{align*}
  meaning : 
  \begin{align*}
    v_s(x,\tau ) = f(x-\tau *b)  \implies u_s(x,t) = f(x-(t-s)b,s)
  .\end{align*}
  Then : 
  \begin{align*}
    u(x,t) = \int_0^t f(x+(s-t)b,s) ds 
  .\end{align*}
  solves the Inhomogeneous transport equation with initial value 0 (4 .c )
\end{solution}

