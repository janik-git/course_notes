\chapter{Laplace Equation}
The Laplace equation is given by : 
\begin{align*}
  \triangle u = \frac{\partial ^2 u}{\partial x_1^2} + \ldots  + \frac{\partial ^2 u}{\partial x_n^2}  = 0
.\end{align*}
With the corresponding inhomogeneous PDE (Poissons equation) : 
\begin{align*}
  - \triangle u = f
.\end{align*}
Note :
\begin{align*}
  \triangle u = \triangledown \cdot \triangledown u  
.\end{align*}
\begin{definition}[Harmonic Functions]
  A function is called harmonic if it solves the Laplace equation 
\end{definition}
They describe the potential of an electric field in vacuum (with some distribution of charges f)
\section{Fundamental Solution } % (fold)
\label{sec:Fundamental Solution }
The Laplace equation is invariant with respect to all rotations and translations of the Euclidean space $\mathbb{R}^{n } $
\begin{comment}
 This means that for coordinate translations : 
 \begin{align*}
  x \mapsto x+a \coloneqq x'
 .\end{align*}
 or rotation : 
 \begin{align*}
 x \mapsto R x \coloneqq  x'
 .\end{align*}
 Where R is a rotation matrix it holds that : 
 \begin{align*}
   \triangle u_x  = \triangle u_{x'}
 .\end{align*}
 For translation it is trivial as  a vanishes , and rotation matrices have determinant 1 i.e all eigenvalues are 1
\end{comment}
This means that any harmonic function must be invariant to translations and rotations and only depends on the length of the vector : 
\begin{align*}
  u(x) = v(r) = v(\sqrt{x*x} )
.\end{align*}
Taking the derivative  
\begin{align*}
  \triangledown_x u(x) = v'(\sqrt{x*x} )\triangledown r =  v'(\sqrt{x*x} ) \frac{2x}{2r}
.\end{align*}
Such that the Laplace Equation simplifies to an ODE : 
\begin{align*}
  \triangle u(x) =  \triangledown * \triangledown u = v(r)^{''}\frac{x^2}{r^2} + v'(r)\frac{n}{r} - v'(r)\frac{x^2}{r^2r} = v^{''}(r)+\frac{n-1}{r}v'(r)  = 0 
.\end{align*}
We can solve this ODE by : 
\begin{align*}
  v^{''}(r)  = - \frac{n-1}{r} v'(r)  \quad \implies \quad \frac{v^{''} }{v'} = \frac{1-n}{r} 
.\end{align*}
Using the ln trick : 
\begin{align*}
  \frac{v^{''} }{v'} = \frac{1-n}{r}  \implies \frac{d}{dr} \ln(v'(r)) = \frac{d}{dr} \frac{1-n}{r}
.\end{align*}
Integrating : 
\begin{align*}
  \ln(v'(r)) = (1-n)\ln(r) + C \implies v(r) = \begin{cases}
    C'\ln(r) + C^{''} , &\text{ if }n=2 \\
    \frac{C'}{r^{n-2} } + C^{''} &\text{ if }n\ge 3
  \end{cases}
.\end{align*}
Calculating is just $\exp((1-n)\ln(r) ) = \exp(\ln(r))^{1-n} $ \\ [1ex]
Meaning we get a two dimensional solution space ($C',C^{''} $)
\begin{definition}[Fundamental Solutions]
  Let $\Phi(x)$ be the following solutions of the Laplace equation  : 
  \begin{align*}
    \Phi(x) = \begin{cases}
      -\frac{1 }{2\pi }\ln \abs{x }, &\text{ if }n=2\\
      \frac{1}{n(n-2)\omega_n \abs{x }^{n-2 } } &\text{ if } n\ge 3
    \end{cases}
  .\end{align*}
  Where $\omega_n$ denotes the volume of the unit ball $B(0,1)$ in Euclidean space $\mathbb{R}^{n} $.
\end{definition}
Choosing $C^{''} =0$ shows this solution lies in the space of symmetric solutions.\\
Where as $C'$ is chosen such that the following holds : 
\begin{theorem} 
  For $f \in  \mathcal{C}^{2}_0(\mathbb{R}^{n} ) $ a solution of Poisson's equations $- \triangle u = f$ is given by 
  \begin{align*}
    u(x) = \Phi \star f  = \int_{\mathbb{R}^{n } } \Phi(y)f(x-y) d^{n }y 
  .\end{align*}
  The distribution corresponding to the fundamental solution obeys $- \triangle F_{\Phi } = \delta $
\end{theorem}
\begin{proof}[Proof ]
  Differentiating u twice , noting that integral and Differentiating can be switched as A , f has compact support 
  it vanishes on the boundary and it converges as it can be decomposed into a finite sum of bounded integrals.\\[1ex]
  Next step is splitting the integral at the singularity of $\Phi $ in terms of $\epsilon -$balls around it.
  The two integrals are shown to both converge to 0 for $\epsilon  \to  0$ by upper bounding them (first take max norm as a constant, then limit the integral of $\Phi $ which includes an epsilon term)\\[1ex]
  Distribution step works by using the gradient operation : 
  \begin{align*}
    (\triangle F_{\Phi })(\phi ) = F_{\Phi }(\triangle \phi ) 
  .\end{align*}
  This is the same as setting $\phi(y) = f(0-y)$ such that $-\triangle F_{\Phi }(\phi ) = \phi(0)$ which is the definition of the delta distribution
\end{proof}
\begin{comment}
  In general, a fundamental solution of a constant coefficient linear PDE $Lu = f$ has the property that 
  $L \Phi  = \delta $ in the sense of distribution. : 
  \begin{align*}
    L(\Phi \star  f)  = (L \Phi) \star  f = \delta \star  f = f
  .\end{align*}
\end{comment}
% section Fundamental Solution  (end)
\section{Mean Value Property } % (fold)
\label{sec:Mean Value Property }
Mean value property shows that any harmonic function u is equal to its mean on any ball given the ball is part of the domain by : 
\begin{align*}
  u(x) = M(u,x,r)
.\end{align*}
for any r , this allows us to prove the Maximum principle which says that any harmonic function if it takes a maximum does so on its boundary 
which gives an easy way to show that solutions to the dirichtlet problem are unique by $max(-v) = - min(v) = 0$ where $v = u_1-u_2$ for two solutions.\\[1ex]
Goal is to prove that for any harmonic functions on an open domain $\Omega  \subset  \mathbb{R}^{n } $ the following holds : 
\begin{definition}[Mean Value Property]
  If u is a harmonic function on an open domain $\Omega  \subset  \mathbb{R}^{n } $ then the value  $u(x)$  at the center of any ball 
  $B(x,r)$ with compact closure in $\Omega $ is equal to the mean of u on the boundary of the ball. And the opposite 
  if this holds for all balls with compact closure then u is harmonic.
\end{definition}
\begin{definition}
  Given a function u the spherical mean is given by : 
  \begin{align*}
    S[u](x,r) \coloneqq  \frac{1}{n \omega_n r^{n-1 } }\int_{\partial B(x,r)} u(y) d\sigma(y) =  \frac{1}{n\omega_n} \int_{\partial B(0,1)} u(x+rz)d\sigma(z)
  .\end{align*}
  Where $\omega_n$ denotes the volume of the unit ball in $\mathbb{R}^{n } $
\end{definition}
The mean of $u$ on the Ball $B(x,r)$ is the mean over $r' \in  [0,r]$ of the spherical means of $u$ on $\partial B(x,r')$ (basically taking the ball and unfolding it into $\abs{[0,r]}$ many lines and integrating over it and taking the mean 
this is similar to integrating two dimensional shapes , remember video , https://www.youtube.com/watch?v=jNpKKDekS6k). 
\begin{align*}
  \int_{B(x,r)} u d\mu  = \int_0^r (\int_{\partial B(x,s)} u d \sigma ) ds
.\end{align*}
\begin{corollary}
  Spherical mean and means have several nice properties, the normalisation constant in $S[u]$ ensures that : 
  \begin{align*}
    S[c] = c
  .\end{align*}
  for any constant function c , and the linearity of the integral gives 
  \begin{align*}
    S[au + bv] = a S[u] + bS[v]
  .\end{align*}
  And : 
  \begin{align*}
    u \le  v \implies S[u] \le S[v]
  .\end{align*}
\end{corollary}
\begin{lemma}
  If u is a continuous function then $\lim_{r \downarrow 0 } S[u](x,r) = u(x)$ 
\end{lemma}
\begin{proof}[Proof]
  Just using the constant properties and continuous nature : 
  \begin{align*}
    \abs{S[u] - u(x)} = \abs{S[u]-S[u(x)]} = \abs{S[u-u(x)]} \le S[\abs{u-u(x)}] < S[\epsilon ] = \epsilon 
  .\end{align*}
  Which proves the statement.
\end{proof}
We get the following property using the divergence theorem : 
\begin{align*}
  \frac{\partial }{\partial r} S(r) &= \frac{1}{n \omega_n} \int_{\partial B(0,1)} \frac{d}{dr} u(x+rz) d\sigma(z) = \frac{1}{n \omega_n} \int_{\partial B(0,1)} \triangledown u (x+rz)*z d\sigma(z) \\ 
                                    &= \frac{1}{n \omega_n r^{n-1 } } \int_{\partial B(x,r)} \triangledown u(y) * N d\sigma(y) = \frac{1}{n \omega_n r^{n-1 } }\int_{B(x,r)} \triangle u d\mu  
.\end{align*}


% section Mean Value Property  (end)
