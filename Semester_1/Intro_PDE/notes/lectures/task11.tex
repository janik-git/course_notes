\begin{align*}
  &\text{Leon Fiethen 1728330}\\
 &\text{Janik Sperling 1728567}
.\end{align*}
\section*{Sheet 11}
\subsection*{33 The distribution of heat}
Consider the fundamental solution of the heat equation $\Phi(x,t)$ given in Definition 4.5.
\begin{exercise}[a]
 Show that this extends to a smooth function on $\mathbb{R}^{n} \times  \mathbb{R} \setminus \{(0,0)\}   $ 
\end{exercise}
\begin{proof}
 We recall 
 \begin{align*}
  \Phi(x,t) = \begin{cases}
    \frac{1}{(4\pi t)^{\frac{n}{2}}}e^{-\frac{\abs{x}^2}{4t}}  &\text{ for } x \in  \mathbb{R}^{n} ,t > 0\\
    0  &\text{ for } x \in  \mathbb{R}^{n} ,t\le  0\\
  \end{cases}
 .\end{align*}
 Pick $x \in  \mathbb{R}^{n} $ and handle the cases $t>0$ , $t<0$ and $t=0$,
 for $t=0$ we have that for $n \in  \mathbb{N}$,  $\Phi(x,t)^{(n)} \equiv 0 $ a smooth function,
 for $t<0$ we also see that $\Phi(x,t)^{(n)} \equiv 0 $ and $t\to 0^{-}  $ are also $0$ and we get no problem here.
 \\[1ex]
 We handle $t>0$ by considering that from the last sheet we know that the derivatives of $e^{-x^2} $ are all of the form 
 \begin{align*}
  p(x)e^{-x^2} 
 .\end{align*}
 For some polynomial $p(x)$ in this case we get extra terms including $t$ i.e 
 \begin{align*}
   \Phi^{(n)}(x,t) = q(x,t)e^{-\frac{\abs{x}^2}{4t}}  
 .\end{align*}
 similar to last sheet, the exponential growth for $x\neq 0$ (to 0 since $t\to 0^{+} \implies e^{-\frac{\abs{x}^2}{4t}}\to 0 $) 
 will "outperform" the polynomial growth and force the expression towards 0. For $x=0$ the polynomial is $0$ and thus the whole expression.\\[1ex]
 We get that the fundamental solution extends to a smooth function.
\end{proof}
\newpage
\begin{exercise}[b]
 Verify that this obey the heat equation on $\mathbb{R}^{n}  \times  \mathbb{R} \setminus \{(0,0)\}  $  
\end{exercise}
\begin{solution}
  Let us calculate 
  \begin{align*}
    \partial_t \Phi  &= \frac{1}{(4\pi)^{\frac{n}{2}}  }(-\frac{n}{2} \frac{1}{t^{\frac{n}{2}+1}}*e^{-\frac{\abs{x}^2}{4t}} + \frac{1}{t^{\frac{n}{2}}} \frac{\abs{x}^2}{4t^2}e^{-\frac{\abs{x}^2}{4t}} )\\
                     & \Phi (-\frac{n}{2}\frac{1}{t} + \frac{\abs{x}^2}{4t^2})
  .\end{align*}
  and 
  \begin{align*}
    \partial_{x_i} \Phi  &= - \frac{x_i}{2t}*\Phi \\
    \partial_{x_i^2} \Phi  &=  \frac{x_i^2}{4t^2}\Phi  -  \frac{1}{2t}\Phi \\
                           &= \Phi ( \frac{x_i^2}{4t^2} -  \frac{1}{2t})
  .\end{align*}
  Then 
  \begin{align*}
    \partial_t - \Delta  \Phi  &= \Phi (-\frac{n}{2t}+\frac{\abs{x}^2}{4t^2} -\frac{\abs{x}^2}{4t^2} + \frac{n}{2t} )\\
                               &= \Phi *0
  .\end{align*}
\end{solution}
\begin{exercise}[c]
 Why must there be a constant $T>0$  with 
 \begin{align*}
   H(\phi ) = \int_0^{T}  \int_{\mathbb{R}^{n} } \Phi(x,t)\phi(x,t)dx dt
 .\end{align*}
\end{exercise}
\begin{proof}
 For $t<0$  the Integral vanishes because $\Phi(x,t) = 0$, since $\phi  \in  C_0^{\infty}(K) $ 
 \begin{align*}
   \int_{\mathbb{R}} \phi(x,t) dt = \int_{-T}^{T}  \phi(x,t)
 .\end{align*}
 For some $T>0$ such that we get 
 \begin{align*}
   H(\phi ) &= \int_{\mathbb{R}}  \int_{\mathbb{R}^{n} } \Phi(x,t)\phi(x,t)dx dt\\
            &= \int_{-T}^{T}   \int_{\mathbb{R}^{n} } \Phi(x,t)\phi(x,t)dx dt\\
            &= \int_{0}^{T}   \int_{\mathbb{R}^{n} } \Phi(x,t)\phi(x,t)dx dt\\
 .\end{align*}
\end{proof}
\begin{exercise}[d]
 Conclude with the help of Lemma 4.6 and Theorem 4.7 that 
 \begin{align*}
   \abs{H(\phi )}\le T \|\phi \|_{K,0}
 .\end{align*}
 Hence $H$ is a continuous linear functional
\end{exercise}
\begin{proof}
 \begin{align*}
   \abs{H(\phi )}  &= \abs*{\int_{0}^{T}  \int_{\mathbb{R}^{n} }\Phi(x,t)\phi(x,t) dx dt}\\
                  &\le \int_{0}^{T}  \int_{\mathbb{R}^{n} }\abs{\Phi(x,t)\phi(x,t)} dx dt\\
                  &\le \int_{0}^{T}  \int_{\mathbb{R}^{n} }\abs{\Phi(x,t)}\|\phi \|_{K,0} dx dt\\
                  &\le \int_{0}^{T} \|\phi \|_{K,0} \int_{\mathbb{R}^{n} }\abs{\Phi(x,t)} dx dt\\
 .\end{align*} 
 Now for $t>0$ we have 
 \begin{align*}
   \int_{\mathbb{R}^{n} } \abs{\Phi(x,t)} dx  =  \int_{\mathbb{R}^{n} } \Phi(x,t) dx  = 1
 .\end{align*}
 For $t\to 0$ we have by 4.7. that also
 \begin{align*}
   \int_{\mathbb{R}^{n} } \abs{\Phi(x,t)} dx  =  \underbrace{\int_{\mathbb{R}^{n} } \Phi(x,t)*1 dx }_{u(x,t)} \to 1
 .\end{align*}
 Thus 
 \begin{align*}
   \int_{0}^{T} \|\phi \|_{K,0} \int_{\mathbb{R}^{n} }\abs{\Phi(x,t)} dx dt \le T*\|\phi \|_{K,0} 
 .\end{align*}
\end{proof}
\begin{exercise}[e]
 Extend Theorem 4.7  to show that 
 \begin{align*}
   \int_{\mathbb{R}^{n} }\Phi(x-y,t)h(y,s)dy \to h(x,s)
 .\end{align*}
 as $t\to 0$ uniformly in $s$
\end{exercise}
\begin{proof}
  In the last step of the proof of (iii) we bound 
  \begin{align*}
    \abs{h(y)-h(x)} \le 2 \sup \{\abs{h(y)} , y \in  \mathbb{R}^{n} \}
  .\end{align*}
  By making the necessary assumption that $h \in  \mathcal{C}_b(\mathbb{R}^{n} \times  \mathbb{R})$ we instead bound (for our $h$ now)
    \begin{align*}
    \abs{h(y,s)-h(x,t)} \le 2 \sup \{\abs{h(y,x)} , y,x \in  \mathbb{R}^{n}\times \mathbb{R} \}
  .\end{align*}
  everything else stays the same 
  \end{proof}
  \begin{exercise}[f]
   Hence show that 
   \begin{align*}
     \int_{\epsilon}^{\infty} \int_{\mathbb{R}^{n} }\Phi(-\partial_t \phi  - \Delta \phi )dydt \to \phi(0,0)
   .\end{align*}
   as $\epsilon \to 0$
  \end{exercise}
  \begin{proof}
   Assuming we mean 
   \begin{align*}
     \int_{\epsilon}^{\infty} \int_{\mathbb{R}^{n} }\Phi(y,t)(-\partial_t \phi(y,t)  - \Delta \phi(y,t) )dydt \to \phi(0,0)
   .\end{align*}
   Then 
   \begin{align*}
     \int_{\epsilon}^{\infty} \int_{\mathbb{R}^{n} }\Phi(y,t)(-\partial_t \phi(y,t)  - \Delta \phi(y,t) )dydt &= \int_{\mathbb{R}^{n} }-\Phi(y,\epsilon)\phi(y,\epsilon) dy  \\
                                                                                                              &  \quad -\int_{\epsilon}^{\infty}\int_{\mathbb{R}^{n} }  \partial_t \Phi(y,t) (-\phi - \Delta \phi(y,t) )dy\\
                                                                                                              &=\int_{\mathbb{R}^{n} }-\Phi(y,\epsilon)\phi(y,\epsilon) dy  \\
                                                                                                              & \quad  -\int_{\epsilon}^{\infty}\int_{\mathbb{R}^{n} }  \underbrace{\partial_t \Phi(y,t)-\Delta}_{=0} (-\phi)dy dt\\
                                                                                                              &= \int_{\mathbb{R}^{n} }-\Phi(y,\epsilon)\phi(y,\epsilon) dy  \\
   .\end{align*}
   Some sign is wrong but if we ignore that then in the end we have, since $\Phi $ only depends on the length of the $x$ variable
   \begin{align*}
     u(0,\epsilon) &= \int_{\mathbb{R}^{n} }\Phi(-y,\epsilon)\phi(y,\epsilon) dy  \\
   .\end{align*}
   which by e) tends to
   \begin{align*}
    \phi(0,0)
   .\end{align*}
   as $\epsilon\to 0$
  \end{proof}
  \begin{exercise}[g]
   Prove that as $\epsilon \to 0$  
   \begin{align*}
     \int_0^{\epsilon} \int_{\mathbb{R}^{n} } \Phi(y,t)h(y,t) dy dt \to 0
   .\end{align*}
  \end{exercise}
  \begin{proof}
   We have 
   \begin{align*}
     \abs*{\int_0^{\epsilon} \int_{\mathbb{R}^{n} } \Phi(y,t)h(y,t) dy dt - 0} &\le \abs*{\int_0^{\epsilon} \int_{\mathbb{R}^{n} } \Phi(y,t)h(y,t) dy dt}\\
                                                                               &\le \int_0^{\epsilon} \int_{\mathbb{R}^{n} } \abs*{\Phi(y,t)h(y,t)} dy dt\\
                                                                               &\le  \int_0^{\epsilon}  \sup_{x \in  \mathbb{R}^{n} } \abs{h(x,t)} \int_{\mathbb{R}^{n} }\Phi(y,t) dy dt\\
                                                                               &\le  \int_0^{\epsilon}  \sup_{(x,s) \in  \mathbb{R}^{n} \times \mathbb{R}^{+}  } \abs{h(x,s)} \int_{\mathbb{R}^{n} }\Phi(y,t) dy dt\\
                                                                               &\le  \sup_{(x,s) \in  \mathbb{R}^{n} \times \mathbb{R}^{+}  } \abs{h(x,s)} \int_0^{\epsilon}  \int_{\mathbb{R}^{n} }\Phi(y,t) *1 dy dt\\
                                                                               &\le \epsilon \sup_{(x,s) \in  \mathbb{R}^{n} \times \mathbb{R}^{+}  } \abs{h(x,s)} \\
                                                                               &\to  0
   .\end{align*}
   Where we used similar argument to part d for handling the $\Phi $ integral
  \end{proof}
  \subsection*{34. Heat death of the universe}
  \begin{exercise}[a]
   Suppose that $h \in  \mathcal{C}_b(\mathbb{R}^{n} ) \cap L^{1}(\mathbb{R}^{n} ) $ and u is defined as in Theorem 4.7. Show
   \begin{align*}
     \sup_{x \in  \mathbb{R}^{n} } \abs{u(x,t)} \le \frac{1}{(4\pi t)^{\frac{n}{2}} }\|h\|_{L^{1} }
   .\end{align*}
  \end{exercise}
  \begin{proof}
    Take $t>0$ and check  with 4.7.
    \begin{align*}
      \sup_{x \in  \mathbb{R}^{n} } \abs{u(x,t} &\le  \sup_{x \in \mathbb{R}^{n} } \int_{\mathbb{R}^{n} }\abs{\Phi(x-y,t)h(y)} dy\\
                                                &\le  \sup_{x \in \mathbb{R}^{n} } \int_{\mathbb{R}^{n} }\abs{\Phi(x-y,t)} \abs{h(y)} dy\\
                                                &= \sup_{x \in \mathbb{R}^{n} } \int_{\mathbb{R}^{n} } \frac{1}{(4\pi t)^{\frac{n}{2}} }e^{-\frac{\abs{x-y}^2}{4t}}  \abs{h(y)} dy\\
                                                &= \frac{1}{(4\pi t)^{\frac{n}{2}} } \sup_{x \in \mathbb{R}^{n} } \int_{\mathbb{R}^{n} } \underbrace{e^{-\frac{\abs{x-y}^2}{4t}}}_{\le 1}  \abs{h(y)} dy\\
                                                &= \frac{1}{(4\pi t)^{\frac{n}{2}} } \sup_{x \in \mathbb{R}^{n} } \underbrace{\int_{\mathbb{R}^{n} } \abs{h(y)}}_{\|h\|_{L^{1} }} dy\\
                                                &\le  \frac{1}{(4\pi t)^{\frac{n}{2}} } \|h\|_{L^{1}} 
    .\end{align*}
  \end{proof}
  \begin{exercise}[b]
   Let $l_m$ be the function from Theorem 4.7. that solves the heat equation on $\mathbb{R}^{n} $ with $l_m(x,0) = mk(x)$ for m a constant
   and $k : \mathbb{R}^{n} \to [0,1]$ a smooth function of compact support such that 
   \begin{align*}
     k \rvert_{\Omega } \equiv 1
   .\end{align*}
   why must k exist ? Why does $l_m \to 0$ as $t\to \infty$? What boundary conditions on $\Omega $ does it obey ?
  \end{exercise}
  \begin{proof}
   Let us first assume that we find $l_m$ such that 
   \begin{align*}
    l_m(x,0) = mk(x)
   .\end{align*}
   with $k$ smooth and compact support, then $k \in  \mathcal{C}_b(\mathbb{R}^{n} ) \cap L^{1}(\mathbb{R}^{n} ) $ 
   % \begin{align*}
   %   \int_{\mathbb{R}^{n} } \abs{k(x)} dx  = \int_{K} \abs{k(x)} dx \le \int_{K} \sup_{x \in  K} \abs{k(x)} dx \le  \sup_{x \in  K} \abs{k(x)} * \lambda(K) \le \infty
   % .\end{align*}
   and by part a) we get that 
   \begin{align*}
     \sup_{x \in  \mathbb{R}^{n} }\abs{u(x,t)} \le  \frac{m}{(4\pi t)^{\frac{n}{2}} }\|k\|_{L^{1} } \xrightarrow{t\to \infty} 0
   .\end{align*}
   For the boundary conditions
   \begin{align*}
    \partial \Omega_T = (\partial \Omega  \times  (0,T]) \cup (\overline{\Omega } \times  0 )
   .\end{align*}
   we check individually
   $t = 0$ and $x \in  \Omega $ then we have by (iii) (or rather assumption)
   \begin{align*}
     l_m(x,0)  = mk(x) = m \marginnote{$k\rvert_{\Omega } \equiv 1$}
   .\end{align*}
   For $x \in  \partial \Omega $ and $t \in  (0,T]$
   \begin{align*}
     u(x,t) = \int_{\mathbb{R}^{n} }\Phi(x-y,t)mk(y) dy &=  \int_{K} \Phi(x-y,t)mk(y) dy\\
   .\end{align*}
   We have $k(y) \ge 0$ and for $\Phi(x-y,t) > 0$ , ($t>0$), so the sign of $u$ depends on the sign of $m$\\[1ex]
   For the existence of $k$ we consider that since $\Omega $ is bounded, then it can be contained in a compact set $K$ such that 
   we set $\supp k = K$ and $k \rvert_{\Omega } \equiv 1$ which is smooth, for $\Omega  \setminus K$ let $k \to  0$ in a smooth way.
  \end{proof}
  \begin{exercise}[c]
   Use the monotonicity property to show that u tends to zero. 
  \end{exercise}
  \begin{proof}
   We learned that if we have two problems with 
   \begin{align*}
     h_{1} &\ge h_{2}\\
     g_{1} &\ge g_{2}
   .\end{align*}
   then 
   \begin{align*}
    u_{1} \ge u_{2}
   .\end{align*}
  Let $u$ be a solution to the homogeneous heat equation such that, for some $h \in  \mathcal{C}_b(\mathbb{R}^{n} ) \cap L^{1} $
  \begin{align*}
    u(x,0) = h(x)
  .\end{align*}
  and  on $\partial \Omega  \times  \mathbb{R}^{+} $
  \begin{align*}
    u(x,t) = 0
  .\end{align*}
  we can then construct two solutions by considering
  \begin{align*}
   a\coloneqq  \sup_{x \in  \Omega } \abs{u(x,0)} =  \sup_{x \in  \Omega } \abs{h(x)}
  .\end{align*}
  Then 
  \begin{align*}
    l_{-a}(x,0) \le u(x,0) \le l_{a}(x,0)
  .\end{align*}
  so we must also have
  \begin{align*}
    l_{-a}(x,t) \le u(x,t) \le l_{a}(x,t)
  .\end{align*}
  but as shown above we have 
  \begin{align*}
    \lim_{t \to \infty} l_{-a}(x,t) = 0 = \lim_{t \to \infty} l_a(x,t)
  .\end{align*}
  thus $u(x,t) \to 0$ as well.
  \end{proof}

